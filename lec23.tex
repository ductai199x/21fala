\documentclass[numbers=enddot,12pt,final,onecolumn,notitlepage]{scrartcl}%
\usepackage[headsepline,footsepline,manualmark]{scrlayer-scrpage}
\usepackage[all,cmtip]{xy}
\usepackage{amssymb}
\usepackage{amsmath}
\usepackage{amsthm}
\usepackage{framed}
\usepackage{comment}
\usepackage{color}
\usepackage{hyperref}
\usepackage[sc]{mathpazo}
\usepackage[T1]{fontenc}
\usepackage{tikz}
\usepackage{needspace}
\usepackage{tabls}
\usepackage{wasysym}
\usepackage{easytable}
\usepackage{pythonhighlight}
%TCIDATA{OutputFilter=latex2.dll}
%TCIDATA{Version=5.50.0.2960}
%TCIDATA{LastRevised=Monday, November 29, 2021 11:49:44}
%TCIDATA{SuppressPackageManagement}
%TCIDATA{<META NAME="GraphicsSave" CONTENT="32">}
%TCIDATA{<META NAME="SaveForMode" CONTENT="1">}
%TCIDATA{BibliographyScheme=Manual}
%TCIDATA{Language=American English}
%BeginMSIPreambleData
\providecommand{\U}[1]{\protect\rule{.1in}{.1in}}
%EndMSIPreambleData
\usetikzlibrary{arrows.meta}
\usetikzlibrary{chains}
\newcounter{exer}
\newcounter{exera}
\numberwithin{exer}{subsection}
\theoremstyle{definition}
\newtheorem{theo}{Theorem}[subsection]
\newenvironment{theorem}[1][]
{\begin{theo}[#1]\begin{leftbar}}
{\end{leftbar}\end{theo}}
\newtheorem{lem}[theo]{Lemma}
\newenvironment{lemma}[1][]
{\begin{lem}[#1]\begin{leftbar}}
{\end{leftbar}\end{lem}}
\newtheorem{prop}[theo]{Proposition}
\newenvironment{proposition}[1][]
{\begin{prop}[#1]\begin{leftbar}}
{\end{leftbar}\end{prop}}
\newtheorem{defi}[theo]{Definition}
\newenvironment{definition}[1][]
{\begin{defi}[#1]\begin{leftbar}}
{\end{leftbar}\end{defi}}
\newtheorem{remk}[theo]{Remark}
\newenvironment{remark}[1][]
{\begin{remk}[#1]\begin{leftbar}}
{\end{leftbar}\end{remk}}
\newtheorem{coro}[theo]{Corollary}
\newenvironment{corollary}[1][]
{\begin{coro}[#1]\begin{leftbar}}
{\end{leftbar}\end{coro}}
\newtheorem{conv}[theo]{Convention}
\newenvironment{convention}[1][]
{\begin{conv}[#1]\begin{leftbar}}
{\end{leftbar}\end{conv}}
\newtheorem{quest}[theo]{Question}
\newenvironment{question}[1][]
{\begin{quest}[#1]\begin{leftbar}}
{\end{leftbar}\end{quest}}
\newtheorem{warn}[theo]{Warning}
\newenvironment{warning}[1][]
{\begin{warn}[#1]\begin{leftbar}}
{\end{leftbar}\end{warn}}
\newtheorem{conj}[theo]{Conjecture}
\newenvironment{conjecture}[1][]
{\begin{conj}[#1]\begin{leftbar}}
{\end{leftbar}\end{conj}}
\newtheorem{exam}[theo]{Example}
\newenvironment{example}[1][]
{\begin{exam}[#1]\begin{leftbar}}
{\end{leftbar}\end{exam}}
\newtheorem{exmp}[exer]{Exercise}
\newenvironment{exercise}[1][]
{\begin{exmp}[#1]\begin{leftbar}}
{\end{leftbar}\end{exmp}}
\newenvironment{statement}{\begin{quote}}{\end{quote}}
\newenvironment{fineprint}{\medskip \begin{small}}{\end{small} \medskip}
\iffalse
\newenvironment{proof}[1][Proof]{\noindent\textbf{#1.} }{\ \rule{0.5em}{0.5em}}
\newenvironment{question}[1][Question]{\noindent\textbf{#1.} }{\ \rule{0.5em}{0.5em}}
\newenvironment{warning}[1][Warning]{\noindent\textbf{#1.} }{\ \rule{0.5em}{0.5em}}
\newenvironment{teachingnote}[1][Teaching note]{\noindent\textbf{#1.} }{\ \rule{0.5em}{0.5em}}
\fi
\let\sumnonlimits\sum
\let\prodnonlimits\prod
\let\cupnonlimits\bigcup
\let\capnonlimits\bigcap
\renewcommand{\sum}{\sumnonlimits\limits}
\renewcommand{\prod}{\prodnonlimits\limits}
\renewcommand{\bigcup}{\cupnonlimits\limits}
\renewcommand{\bigcap}{\capnonlimits\limits}
\setlength\tablinesep{3pt}
\setlength\arraylinesep{3pt}
\setlength\extrarulesep{3pt}
\voffset=0cm
\hoffset=-0.7cm
\setlength\textheight{22.5cm}
\setlength\textwidth{15.5cm}
\newcommand\arxiv[1]{\href{http://www.arxiv.org/abs/#1}{\texttt{arXiv:#1}}}
\newenvironment{verlong}{}{}
\newenvironment{vershort}{}{}
\newenvironment{noncompile}{}{}
\newenvironment{teachingnote}{}{}
\excludecomment{verlong}
\includecomment{vershort}
\excludecomment{noncompile}
\excludecomment{teachingnote}
\newcommand{\CC}{\mathbb{C}}
\newcommand{\RR}{\mathbb{R}}
\newcommand{\QQ}{\mathbb{Q}}
\newcommand{\NN}{\mathbb{N}}
\newcommand{\ZZ}{\mathbb{Z}}
\newcommand{\KK}{\mathbb{K}}
\newcommand{\id}{\operatorname{id}}
\newcommand{\lcm}{\operatorname{lcm}}
\newcommand{\rev}{\operatorname{rev}}
\newcommand{\powset}[2][]{\ifthenelse{\equal{#2}{}}{\mathcal{P}\left(#1\right)}{\mathcal{P}_{#1}\left(#2\right)}}
\newcommand{\set}[1]{\left\{ #1 \right\}}
\newcommand{\abs}[1]{\left| #1 \right|}
\newcommand{\tup}[1]{\left( #1 \right)}
\newcommand{\ive}[1]{\left[ #1 \right]}
\newcommand{\floor}[1]{\left\lfloor #1 \right\rfloor}
\newcommand{\lf}[2]{#1^{\underline{#2}}}
\newcommand{\underbrack}[2]{\underbrace{#1}_{\substack{#2}}}
\newcommand{\horrule}[1]{\rule{\linewidth}{#1}}
\newcommand{\are}{\ar@{-}}
\newcommand{\nnn}{\nonumber\\}
\newcommand{\sslash}{\mathbin{/\mkern-6mu/}}
\newcommand{\numboxed}[2]{\underbrace{\boxed{#1}}_{\text{box } #2}}
\newcommand{\ig}[2]{\includegraphics[scale=#1]{#2.png}}
\newcommand{\UNFINISHED}{\begin{center} \Huge{\textbf{Unfinished material begins here.}} \end{center} }
\iffalse
\NOEXPAND{\today}{\today}
\NOEXPAND{\sslash}{\sslash}
\NOEXPAND{\numboxed}[2]{\numboxed}
\NOEXPAND{\UNFINISHED}{\UNFINISHED}
\fi
\ihead{Math 504 notes}
\ohead{page \thepage}
\cfoot{\today}
\begin{document}

\title{Math 504: Advanced Linear Algebra}
\author{Hugo Woerdeman, with edits by Darij Grinberg\thanks{Drexel University, Korman
Center, 15 S 33rd Street, Philadelphia PA, 19104, USA}}
\date{\today\ (unfinished!)}
\maketitle
\tableofcontents

\section*{Math 504 Lecture 23}

\section{Positive and nonnegative matrices ([HorJoh, Chapter 8])}

\subsection{Basics}

Recall the triangle inequality:

\begin{proposition}
[triangle inequality]Let $z_{1},z_{2},\ldots,z_{n}$ be $n$ complex numbers.
Then,%
\[
\left\vert z_{1}\right\vert +\left\vert z_{2}\right\vert +\cdots+\left\vert
z_{n}\right\vert \geq\left\vert z_{1}+z_{2}+\cdots+z_{n}\right\vert .
\]


Equality holds if and only if $z_{1},z_{2},\ldots,z_{n}$ have the same
argument (i.e., there exists some $w\in\mathbb{C}$ such that $z_{1}%
,z_{2},\ldots,z_{n}$ are nonnegative real multiples of $w$).
\end{proposition}

\begin{definition}
Let $A\in\mathbb{C}^{n\times m}$ be a matrix.

\textbf{(a)} We say that $A$ is \textbf{positive} (and write $A>0$) if all
entries of $A$ are positive reals.

\textbf{(b)} We say that $A$ is \textbf{nonnegative} (and write $A\geq0$) if
all entries of $A$ are nonnegative reals.

\textbf{(c)} We let $\left\vert A\right\vert \in\mathbb{R}^{n\times m}$ be the
nonnegative matrix obtained by replacing each entry of $A$ by its absolute
value. In other words,%
\[
\left\vert A\right\vert =\left(
\begin{array}
[c]{cccc}%
\left\vert A_{1,1}\right\vert  & \left\vert A_{1,2}\right\vert  & \cdots &
\left\vert A_{1,m}\right\vert \\
\left\vert A_{2,1}\right\vert  & \left\vert A_{2,2}\right\vert  & \cdots &
\left\vert A_{2,m}\right\vert \\
\vdots & \vdots & \ddots & \vdots\\
\left\vert A_{n,1}\right\vert  & \left\vert A_{n,2}\right\vert  & \cdots &
\left\vert A_{n,m}\right\vert
\end{array}
\right)  .
\]

\end{definition}

\begin{remark}
Recall that row vectors and column vectors are matrices. Thus, $v>0$ and
$v\geq0$ and $\left\vert v\right\vert $ are defined for them as well. If
$v=\left(  v_{1},v_{2},\ldots,v_{k}\right)  ^{T}$, then $\left\vert
v\right\vert =\left(  \left\vert v_{1}\right\vert ,\left\vert v_{2}\right\vert
,\ldots,\left\vert v_{k}\right\vert \right)  ^{T}$.
\end{remark}

\begin{warning}
Do not mistake $\left\vert v\right\vert $ (a vector) for $\left\vert
\left\vert v\right\vert \right\vert $ (a number). Also, for a matrix $A$, do
not mistake $\left\vert A\right\vert $ for (an old notation for) the
determinant of $A$.
\end{warning}

\begin{exercise}
Prove that for any vector $v\in\mathbb{C}^{m}$, we have $\left\vert \left\vert
\left\vert v\right\vert \right\vert \right\vert =\left\vert \left\vert
v\right\vert \right\vert $, where the left hand side means the length of
$\left\vert v\right\vert $.
\end{exercise}

\begin{definition}
Let $A,B\in\mathbb{R}^{n\times m}$ be two matrices with real entries. Then:

\textbf{(a)} We say that $A\geq B$ if and only if $A-B\geq0$ (or,
equivalently, $A_{i,j}\geq B_{i,j}$ for all $i\in\left[  n\right]  $ and
$j\in\left[  m\right]  $).

\textbf{(b)} We say that $A>B$ if and only if $A-B>0$ (or, equivalently,
$A_{i,j}>B_{i,j}$ for all $i\in\left[  n\right]  $ and $j\in\left[  m\right]
$).

\textbf{(c)} We say that $A\leq B$ if and only if $A-B\leq0$ (or,
equivalently, $A_{i,j}\leq B_{i,j}$ for all $i\in\left[  n\right]  $ and
$j\in\left[  m\right]  $).

\textbf{(d)} We say that $A<B$ if and only if $A-B<0$ (or, equivalently,
$A_{i,j}<B_{i,j}$ for all $i\in\left[  n\right]  $ and $j\in\left[  m\right]
$).
\end{definition}

\begin{example}
We have $\left(
\begin{array}
[c]{cc}%
1 & 2\\
3 & 4
\end{array}
\right)  \geq\left(
\begin{array}
[c]{cc}%
0 & 2\\
2 & 4
\end{array}
\right)  $.
\end{example}

\begin{remark}
Again, recall that row vectors and column vectors are matrices too, so this
applies to them as well.
\end{remark}

\begin{warning}
The relations $\geq$ and $\leq$ and not total orders. For instance, $\left(
2,1\right)  $ is neither $\geq$ nor $\leq$ to $\left(  3,0\right)  $.
\end{warning}

\begin{warning}
Do not mistake $\geq$ for $\succcurlyeq$ (majorization).
\end{warning}

\begin{warning}
For $n=0$, the trivial vector $v=\left(  {}\right)  \in\mathbb{R}^{0}$
satisfies $v>v$ and $v<v$ and $v\geq v$ and $v\leq v$, because the
\textquotedblleft for all\textquotedblright\ statements are vacuously true.
\end{warning}

\begin{warning}
Given two matrices $A$ and $B$, the relation $A\geq B$ is \textbf{not}
equivalent to \textquotedblleft$A>B$ or $A=B$\textquotedblright. For example,
$\left(  2,1\right)  \geq\left(  3,1\right)  $, but neither $>$ nor $=$.
\end{warning}

\begin{proposition}
Let $A\in\mathbb{C}^{n\times m}$ and $B\in\mathbb{C}^{m\times p}$ be two
matrices. Then,%
\[
\left\vert A\right\vert \cdot\left\vert B\right\vert \geq\left\vert
AB\right\vert .
\]

\end{proposition}

\begin{proof}
We must prove that $\left(  \left\vert A\right\vert \cdot\left\vert
B\right\vert \right)  _{i,k}\geq\left\vert AB\right\vert _{i,k}$ for all $i$
and $k$.

So let $i\in\left[  n\right]  $ and $k\in\left[  p\right]  $. Then,
\begin{align*}
\left(  \left\vert A\right\vert \cdot\left\vert B\right\vert \right)  _{i,k}
& =\sum_{j=1}^{m}\underbrace{\left\vert A\right\vert _{i,j}}_{=\left\vert
A_{i,j}\right\vert }\cdot\underbrace{\left\vert B\right\vert _{j,k}%
}_{=\left\vert B_{j,k}\right\vert }=\sum_{j=1}^{m}\underbrace{\left\vert
A_{i,j}\right\vert \cdot\left\vert B_{j,k}\right\vert }_{=\left\vert
A_{i,j}B_{j,k}\right\vert }\\
& =\sum_{j=1}^{m}\left\vert A_{i,j}B_{j,k}\right\vert \geq\left\vert
\sum_{j=1}^{m}A_{i,j}B_{j,k}\right\vert .
\end{align*}
In view of%
\[
\left\vert AB\right\vert _{i,k}=\left\vert \left(  AB\right)  _{i,k}%
\right\vert =\left\vert \sum_{j=1}^{m}A_{i,j}B_{j,k}\right\vert ,
\]
we can rewrite this as $\left(  \left\vert A\right\vert \cdot\left\vert
B\right\vert \right)  _{i,k}\geq\left\vert AB\right\vert _{i,k}$, qed.
\end{proof}

\begin{corollary}
Let $A\in\mathbb{C}^{n\times n}$ and $k\in\mathbb{N}$. Then, $\left\vert
A\right\vert ^{k}\geq\left\vert A^{k}\right\vert $.
\end{corollary}

\begin{proof}
Induction on $k$, using the previous proposition (and the fact that
$\left\vert I_{n}\right\vert =I_{n}$).
\end{proof}

\begin{proposition}
Let $A\in\mathbb{C}^{n\times m}$ and $x\in\mathbb{C}^{m}$. Then:

\textbf{(a)} We have $\left\vert A\right\vert \cdot\left\vert x\right\vert
\geq\left\vert Ax\right\vert $.

\textbf{(b)} If at least one row of $A$ is positive and we have $A\geq0$ and
$\left\vert Ax\right\vert =A\cdot\left\vert x\right\vert $, then $\left\vert
x\right\vert =\omega x$ for some $\omega\in\mathbb{C}$ satisfying $\left\vert
\omega\right\vert =1$.

\textbf{(c)} If $x>0$ and $Ax=\left\vert A\right\vert x$, then $A=\left\vert
A\right\vert $ (so that $A\geq0$).
\end{proposition}

\begin{proof}
\textbf{(a)} follows from the previous proposition.

\textbf{(b)} Assume that at least one row of $A$ is positive and we have
$A\geq0$ and $\left\vert Ax\right\vert =A\cdot\left\vert x\right\vert $.

We have assumed that at least one row of $A$ is positive. Let the $i$-th row
of $A$ be positive. Write $x=\left(  x_{1},x_{2},\ldots,x_{n}\right)  ^{T}$.
From $\left\vert Ax\right\vert =A\cdot\left\vert x\right\vert $, we have%
\[
\left(  \text{the }i\text{-th entry of }\left\vert Ax\right\vert \right)
=\left(  \text{the }i\text{-th entry of }A\cdot\left\vert x\right\vert
\right)  =\sum_{j=1}^{n}A_{i,j}\cdot\left\vert x_{j}\right\vert ,
\]
so that
\begin{align*}
\sum_{j=1}^{n}A_{i,j}\cdot\left\vert x_{j}\right\vert  & =\left(  \text{the
}i\text{-th entry of }\left\vert Ax\right\vert \right)  =\left\vert \text{the
}i\text{-th entry of }Ax\right\vert \\
& =\left\vert \sum_{j=1}^{n}A_{i,j}x_{j}\right\vert
\end{align*}
(since the $i$-th entry of $Ax$ is $\sum_{j=1}^{n}A_{i,j}x_{j}$). This is an
equality case of the triangle inequality (since $A_{i,j}=\left\vert
A_{i,j}\right\vert $ for all $j$). Thus, the complex numbers $A_{i,j}x_{j}$
for all $j\in\left[  n\right]  $ have the same argument. In other words, the
$x_{j}$ for all $j\in\left[  n\right]  $ have the same argument. But this
means that we can multiply $x$ by a complex number on the unit circle and
obtain a vector of positive reals. That latter vector, of course, will be
$\left\vert x\right\vert $.

\textbf{(c)} Suppose $x>0$ and $Ax=\left\vert A\right\vert x$. We must show
that $A=\left\vert A\right\vert $ (so that $A\geq0$).

Write $x=\left(  x_{1},x_{2},\ldots,x_{n}\right)  ^{T}$. Fix $i\in\left[
n\right]  $. Then,%
\[
\left(  \text{the }i\text{-th entry of }Ax\right)  =\left(  \text{the
}i\text{-th entry of }\left\vert A\right\vert x\right)  ,
\]
i.e.%
\[
\sum_{j=1}^{n}A_{i,j}x_{j}=\sum_{j=1}^{n}\underbrace{\left\vert A_{i,j}%
\right\vert x_{j}}_{=\left\vert A_{i,j}x_{j}\right\vert }=\sum_{j=1}%
^{n}\left\vert A_{i,j}x_{j}\right\vert \geq\left\vert \sum_{j=1}^{n}%
A_{i,j}x_{j}\right\vert \geq\sum_{j=1}^{n}A_{i,j}x_{j}.
\]
This is a chain of inequalities in which the first and the last side are
equal. Thus, all inequalities in it must be equalities. In particular, we thus
have equality in the triangle inequality $\sum_{j=1}^{n}\left\vert
A_{i,j}x_{j}\right\vert \geq\left\vert \sum_{j=1}^{n}A_{i,j}x_{j}\right\vert
$. Hence, the complex numbers $A_{i,j}x_{j}$ for all $j\in\left[  n\right]  $
have the same argument. Moreover, $\left\vert \sum_{j=1}^{n}A_{i,j}%
x_{j}\right\vert \geq\sum_{j=1}^{n}A_{i,j}x_{j}$ must also become an equality,
so this argument has to be $0$. This shows that the complex numbers
$A_{i,j}x_{j}$ for all $j\in\left[  n\right]  $ are nonnegative reals. Since
$x>0$, this means that the $A_{i,j}$ are nonnegative reals. Since we have
proved this for all $i$, we thus conclude that all entries of $A$ are
nonnegative reals. Hence, $A=\left\vert A\right\vert $.
\end{proof}

\begin{proposition}
Let $A,B,C,D$ be four complex matrices.

\textbf{(a)} If $A\leq B$ and $C\leq D$, then $A+C\leq B+D$ if $A+C$ and $B+D$
are well-defined.

\textbf{(b)} If $A\leq B$ and $0\leq C$, then $AC\leq BC$ if $AC$ and $BC$ are well-defined.

\textbf{(c)} If $A\leq B$ and $0\leq C$, then $CA\leq CB$ if $CA$ and $CB$ are well-defined.

\textbf{(d)} If $0\leq A\leq B$ and $0\leq C\leq D$, then $0\leq AC\leq BD$ if
$AC$ and $BD$ are well-defined.

\textbf{(e)} If $0\leq A\leq B$ and $k\in\mathbb{N}$, then $0\leq A^{k}\leq
B^{k}$.
\end{proposition}

\begin{proof}
\textbf{(a)} Straightforward.

\textbf{(b)} Assume $A\leq B$ and $0\leq C$. Then, compare
\[
\left(  AC\right)  _{i,k}=\sum_{j}A_{i,j}C_{j,k}%
\ \ \ \ \ \ \ \ \ \ \text{with}\ \ \ \ \ \ \ \ \ \ \left(  BC\right)
_{i,k}=\sum_{j}B_{i,j}C_{j,k}.
\]
The right hand side of the first equation is $\leq$ to the right hand side of
the second, because $A_{i,j}\leq B_{i,j}$ and $C_{j,k}\geq0$ for all $j$. So
$\left(  AC\right)  _{i,k}\leq\left(  BC\right)  _{i,k}$. Thus, $AC\leq BC$.

\textbf{(c)} Similar.

\textbf{(d)} Part \textbf{(b)} yields $AC\leq BC$. Part \textbf{(c)} yields
$BC\leq BD$. Since $\leq$ is transitive, this entails $AC\leq BD$.

\textbf{(e)} Follows from \textbf{(d)} by induction on $k$.
\end{proof}

\subsection{The spectral radius}

\begin{definition}
The \textbf{spectral radius} $\rho\left(  A\right)  $ of a matrix
$A\in\mathbb{C}^{n\times n}$ (with $n>0$) is defined to be the largest
absolute value of an eigenvalue of $A$. That is,%
\[
\rho\left(  A\right)  :=\max\left\{  \left\vert \lambda\right\vert
\ \mid\ \lambda\in\sigma\left(  A\right)  \right\}  .
\]


Note that $\rho\left(  A\right)  $ is always a nonnegative real.
\end{definition}

By Exercise 3.4.2 (equivalence $\mathcal{A}\Longleftrightarrow\mathcal{C}$), a
square matrix $A$ satisfies $\rho\left(  A\right)  =0$ if and only if $A$ is nilpotent.

If $A=\operatorname*{diag}\left(  \lambda_{1},\lambda_{2},\ldots,\lambda
_{n}\right)  $, then $\rho\left(  A\right)  =\max\left\{  \left\vert
\lambda_{1}\right\vert ,\left\vert \lambda_{2}\right\vert ,\ldots,\left\vert
\lambda_{n}\right\vert \right\}  $.

\begin{theorem}
Let $A\in\mathbb{C}^{n\times n}$ and $B\in\mathbb{R}^{n\times n}$ be such that
$B\geq\left\vert A\right\vert $. Then, $\rho\left(  A\right)  \leq\rho\left(
B\right)  $.
\end{theorem}

\begin{proof}
If $\rho\left(  A\right)  =0$, then this is obvious. So, WLOG assume that
$\rho\left(  A\right)  >0$.

We can thus scale both matrices $A$ and $B$ by $\dfrac{1}{\rho\left(
A\right)  }$. This does not break the inequality $B\geq\left\vert A\right\vert
$, and also does not break the claim $\rho\left(  A\right)  \leq\rho\left(
B\right)  $ (since $\rho\left(  \lambda A\right)  =\left\vert \lambda
\right\vert \rho\left(  A\right)  $ for any $\lambda\in\mathbb{C}$).

So we WLOG assume that $\rho\left(  A\right)  =1$. (This is achieved by the
scaling we just mentioned.)

This yields that $A$ has an eigenvalue $\lambda$ with $\left\vert
\lambda\right\vert =1$. Let $v$ be a nonzero eigenvector at this eigenvalue
$\lambda$. Thus,%
\[
A^{m}v=\lambda^{m}v\ \ \ \ \ \ \ \ \ \ \text{for any }m\in\mathbb{N}.
\]


Now, we must prove $\rho\left(  A\right)  \leq\rho\left(  B\right)  $. In
other words, we must prove that $1\leq\rho\left(  B\right)  $. Assume the
contrary. Thus, $\rho\left(  B\right)  <1$. Hence, all eigenvalues of $B$ have
absolute value $<1$. Hence, Corollary 3.5.2 shows that $\lim
\limits_{m\rightarrow\infty}B^{m}=0$. Therefore, $\lim\limits_{m\rightarrow
\infty}B^{m}\left\vert v\right\vert =0$.

However, $B\geq\left\vert A\right\vert \geq0$ entails $B^{m}\geq\left\vert
A\right\vert ^{m}$, so that%
\begin{align*}
B^{m}\left\vert v\right\vert  & \geq\left\vert A\right\vert ^{m}\left\vert
v\right\vert \geq\left\vert A^{m}\right\vert \cdot\left\vert v\right\vert
\ \ \ \ \ \ \ \ \ \ \left(  \text{since }\left\vert A\right\vert ^{m}%
\geq\left\vert A^{m}\right\vert \right)  \\
& \geq\left\vert A^{m}v\right\vert =\left\vert \lambda^{m}v\right\vert
=\underbrace{\left\vert \lambda^{m}\right\vert }_{\substack{=1\\\text{(since
}\left\vert \lambda\right\vert =1\text{)}}}\cdot\left\vert v\right\vert
=\left\vert v\right\vert .
\end{align*}
In view of $\lim\limits_{m\rightarrow\infty}B^{m}\left\vert v\right\vert =0$,
this is only possible if $\left\vert v\right\vert =0$. However, $\left\vert
v\right\vert \neq0$ (since $v$ is nonzero). Contradiction, qed.
\end{proof}

\begin{corollary}
Let $A\in\mathbb{C}^{n\times n}$ and $B\in\mathbb{R}^{n\times n}$ be such that
$B\geq\left\vert A\right\vert $. Then, $\rho\left(  A\right)  \leq\rho\left(
\left\vert A\right\vert \right)  \leq\rho\left(  B\right)  $.
\end{corollary}

\begin{proof}
Applying the above theorem to $\left\vert A\right\vert $ instead of $B$, we
get $\rho\left(  A\right)  \leq\rho\left(  \left\vert A\right\vert \right)  $.

Applying the above theorem to $\left\vert A\right\vert $ instead of $A$, we
get $\rho\left(  \left\vert A\right\vert \right)  \leq\rho\left(  B\right)  $
(since $\left\vert \left\vert A\right\vert \right\vert =\left\vert
A\right\vert $).
\end{proof}

\begin{corollary}
Let $A\in\mathbb{R}^{n\times n}$ and $B\in\mathbb{R}^{n\times n}$ satisfy
$B\geq A\geq0$, then $\rho\left(  A\right)  \leq\rho\left(  B\right)  $.
\end{corollary}

\begin{proof}
Apply the theorem, noticing that $\left\vert A\right\vert =A$.
\end{proof}

\begin{corollary}
Let $A\in\mathbb{R}^{n\times n}$ satisfy $A\geq0$.

\textbf{(a)} If $\widetilde{A}$ is a principal submatrix of $A$ (that is, a
matrix obtained from $A$ by removing a bunch of rows along with the
corresponding columns), then $\rho\left(  \widetilde{A}\right)  \leq
\rho\left(  A\right)  $.

\textbf{(b)} We have $\max\left\{  A_{i,i}\ \mid\ i\in\left[  n\right]
\right\}  \leq\rho\left(  A\right)  $.

\textbf{(c)} If $A_{i,i}>0$ for some $i\in\left[  n\right]  $, then
$\rho\left(  A\right)  >0$.
\end{corollary}

\begin{proof}
\textbf{(a)} Let $\widetilde{A}$ be a principal submatrix of $A$. For
simplicity, I assume that $\widetilde{A}$ is $A$ without the $n$-th row and
the $n$-th column. Thus,%
\[
A=\left(
\begin{array}
[c]{cc}%
\widetilde{A} & y\\
x & \lambda
\end{array}
\right)  \ \ \ \ \ \ \ \ \ \ \left(  \text{in block-matrix notation}\right)
\]
for some nonnegative $x$, $y$ and $\lambda$. Now, it is easy to see that%
\[
\rho\left(  \widetilde{A}\right)  =\rho\left(  \left(
\begin{array}
[c]{cc}%
\widetilde{A} & 0\\
0 & 0
\end{array}
\right)  \right)
\]
(since $\sigma\left(  \left(
\begin{array}
[c]{cc}%
\widetilde{A} & 0\\
0 & 0
\end{array}
\right)  \right)  =\sigma\left(  \widetilde{A}\right)  \cup\left\{  0\right\}
$). However, $0\leq\left(
\begin{array}
[c]{cc}%
\widetilde{A} & 0\\
0 & 0
\end{array}
\right)  \leq\left(
\begin{array}
[c]{cc}%
\widetilde{A} & y\\
x & \lambda
\end{array}
\right)  =A$, so the previous corollary yields%
\[
\rho\left(  \left(
\begin{array}
[c]{cc}%
\widetilde{A} & 0\\
0 & 0
\end{array}
\right)  \right)  \leq\rho\left(  A\right)  .
\]
So $\rho\left(  \widetilde{A}\right)  \leq\rho\left(  A\right)  $. This proves
part \textbf{(a)}.

\textbf{(b)} We must show that $A_{i,i}\leq\rho\left(  A\right)  $ for all
$i\in\left[  n\right]  $.

So let $i\in\left[  n\right]  $. Then, the $1\times1$-matrix $\left(
\begin{array}
[c]{c}%
A_{i,i}%
\end{array}
\right)  $ is a principal submatrix of $A$ (obtained by removing all rows of
$A$ other than the $i$-th one, and all columns of $A$ other than the $i$-th
one). Hence, part \textbf{(a)} yields $\rho\left(  \left(
\begin{array}
[c]{c}%
A_{i,i}%
\end{array}
\right)  \right)  \leq\rho\left(  A\right)  $. However, $\rho\left(  \left(
\begin{array}
[c]{c}%
A_{i,i}%
\end{array}
\right)  \right)  =\left\vert A_{i,i}\right\vert =A_{i,i}$ (since $A\geq0$).
Therefore, $A_{i,i}\leq\rho\left(  A\right)  $, qed.

\textbf{(c)} Follows from \textbf{(b)}.
\end{proof}


\end{document}