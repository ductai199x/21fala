\documentclass[numbers=enddot,12pt,final,onecolumn,notitlepage]{scrartcl}%
\usepackage[headsepline,footsepline,manualmark]{scrlayer-scrpage}
\usepackage[all,cmtip]{xy}
\usepackage{amssymb}
\usepackage{amsmath}
\usepackage{amsthm}
\usepackage{framed}
\usepackage{comment}
\usepackage{color}
\usepackage{hyperref}
\usepackage[sc]{mathpazo}
\usepackage[T1]{fontenc}
\usepackage{tikz}
\usepackage{needspace}
\usepackage{tabls}
\usepackage{wasysym}
\usepackage{easytable}
\usepackage{pythonhighlight}
%TCIDATA{OutputFilter=latex2.dll}
%TCIDATA{Version=5.50.0.2960}
%TCIDATA{LastRevised=Friday, September 24, 2021 11:51:58}
%TCIDATA{SuppressPackageManagement}
%TCIDATA{<META NAME="GraphicsSave" CONTENT="32">}
%TCIDATA{<META NAME="SaveForMode" CONTENT="1">}
%TCIDATA{BibliographyScheme=Manual}
%TCIDATA{Language=American English}
%BeginMSIPreambleData
\providecommand{\U}[1]{\protect\rule{.1in}{.1in}}
%EndMSIPreambleData
\usetikzlibrary{arrows.meta}
\usetikzlibrary{chains}
\newcounter{exer}
\newcounter{exera}
\numberwithin{exer}{subsection}
\theoremstyle{definition}
\newtheorem{theo}{Theorem}[subsection]
\newenvironment{theorem}[1][]
{\begin{theo}[#1]\begin{leftbar}}
{\end{leftbar}\end{theo}}
\newtheorem{lem}[theo]{Lemma}
\newenvironment{lemma}[1][]
{\begin{lem}[#1]\begin{leftbar}}
{\end{leftbar}\end{lem}}
\newtheorem{prop}[theo]{Proposition}
\newenvironment{proposition}[1][]
{\begin{prop}[#1]\begin{leftbar}}
{\end{leftbar}\end{prop}}
\newtheorem{defi}[theo]{Definition}
\newenvironment{definition}[1][]
{\begin{defi}[#1]\begin{leftbar}}
{\end{leftbar}\end{defi}}
\newtheorem{remk}[theo]{Remark}
\newenvironment{remark}[1][]
{\begin{remk}[#1]\begin{leftbar}}
{\end{leftbar}\end{remk}}
\newtheorem{coro}[theo]{Corollary}
\newenvironment{corollary}[1][]
{\begin{coro}[#1]\begin{leftbar}}
{\end{leftbar}\end{coro}}
\newtheorem{conv}[theo]{Convention}
\newenvironment{convention}[1][]
{\begin{conv}[#1]\begin{leftbar}}
{\end{leftbar}\end{conv}}
\newtheorem{quest}[theo]{Question}
\newenvironment{question}[1][]
{\begin{quest}[#1]\begin{leftbar}}
{\end{leftbar}\end{quest}}
\newtheorem{warn}[theo]{Warning}
\newenvironment{warning}[1][]
{\begin{warn}[#1]\begin{leftbar}}
{\end{leftbar}\end{warn}}
\newtheorem{conj}[theo]{Conjecture}
\newenvironment{conjecture}[1][]
{\begin{conj}[#1]\begin{leftbar}}
{\end{leftbar}\end{conj}}
\newtheorem{exam}[theo]{Example}
\newenvironment{example}[1][]
{\begin{exam}[#1]\begin{leftbar}}
{\end{leftbar}\end{exam}}
\newtheorem{exmp}[exer]{Exercise}
\newenvironment{exercise}[1][]
{\begin{exmp}[#1]\begin{leftbar}}
{\end{leftbar}\end{exmp}}
\newenvironment{statement}{\begin{quote}}{\end{quote}}
\newenvironment{fineprint}{\medskip \begin{small}}{\end{small} \medskip}
\iffalse
\newenvironment{proof}[1][Proof]{\noindent\textbf{#1.} }{\ \rule{0.5em}{0.5em}}
\newenvironment{question}[1][Question]{\noindent\textbf{#1.} }{\ \rule{0.5em}{0.5em}}
\newenvironment{warning}[1][Warning]{\noindent\textbf{#1.} }{\ \rule{0.5em}{0.5em}}
\newenvironment{teachingnote}[1][Teaching note]{\noindent\textbf{#1.} }{\ \rule{0.5em}{0.5em}}
\fi
\let\sumnonlimits\sum
\let\prodnonlimits\prod
\let\cupnonlimits\bigcup
\let\capnonlimits\bigcap
\renewcommand{\sum}{\sumnonlimits\limits}
\renewcommand{\prod}{\prodnonlimits\limits}
\renewcommand{\bigcup}{\cupnonlimits\limits}
\renewcommand{\bigcap}{\capnonlimits\limits}
\setlength\tablinesep{3pt}
\setlength\arraylinesep{3pt}
\setlength\extrarulesep{3pt}
\voffset=0cm
\hoffset=-0.7cm
\setlength\textheight{22.5cm}
\setlength\textwidth{15.5cm}
\newcommand\arxiv[1]{\href{http://www.arxiv.org/abs/#1}{\texttt{arXiv:#1}}}
\newenvironment{verlong}{}{}
\newenvironment{vershort}{}{}
\newenvironment{noncompile}{}{}
\newenvironment{teachingnote}{}{}
\excludecomment{verlong}
\includecomment{vershort}
\excludecomment{noncompile}
\excludecomment{teachingnote}
\newcommand{\CC}{\mathbb{C}}
\newcommand{\RR}{\mathbb{R}}
\newcommand{\QQ}{\mathbb{Q}}
\newcommand{\NN}{\mathbb{N}}
\newcommand{\ZZ}{\mathbb{Z}}
\newcommand{\KK}{\mathbb{K}}
\newcommand{\id}{\operatorname{id}}
\newcommand{\lcm}{\operatorname{lcm}}
\newcommand{\rev}{\operatorname{rev}}
\newcommand{\powset}[2][]{\ifthenelse{\equal{#2}{}}{\mathcal{P}\left(#1\right)}{\mathcal{P}_{#1}\left(#2\right)}}
\newcommand{\set}[1]{\left\{ #1 \right\}}
\newcommand{\abs}[1]{\left| #1 \right|}
\newcommand{\tup}[1]{\left( #1 \right)}
\newcommand{\ive}[1]{\left[ #1 \right]}
\newcommand{\floor}[1]{\left\lfloor #1 \right\rfloor}
\newcommand{\lf}[2]{#1^{\underline{#2}}}
\newcommand{\underbrack}[2]{\underbrace{#1}_{\substack{#2}}}
\newcommand{\horrule}[1]{\rule{\linewidth}{#1}}
\newcommand{\are}{\ar@{-}}
\newcommand{\nnn}{\nonumber\\}
\newcommand{\sslash}{\mathbin{/\mkern-6mu/}}
\newcommand{\numboxed}[2]{\underbrace{\boxed{#1}}_{\text{box } #2}}
\newcommand{\ig}[2]{\includegraphics[scale=#1]{#2.png}}
\newcommand{\UNFINISHED}{\begin{center} \Huge{\textbf{Unfinished material begins here.}} \end{center} }
\iffalse
\NOEXPAND{\today}{\today}
\NOEXPAND{\sslash}{\sslash}
\NOEXPAND{\numboxed}[2]{\numboxed}
\NOEXPAND{\UNFINISHED}{\UNFINISHED}
\fi
\ihead{Math 504 notes}
\ohead{page \thepage}
\cfoot{\today}
\begin{document}

\title{Math 504: Advanced Linear Algebra}
\author{Hugo Woerdeman, with edits by Darij Grinberg\thanks{Drexel University, Korman
Center, 15 S 33rd Street, Philadelphia PA, 19104, USA}}
\date{\today\ (unfinished!)}
\maketitle
\tableofcontents

\section*{Math 504 Lecture 3}

\section{Unitary matrices ([HorJoh13, \S 2.1]) (cont'd)}

\subsection{Administrativa}

Next topics after today:

\begin{itemize}
\item Normal matrices and the spectral thm for them (lec 4)
%TCIMACRO{\TEXTsymbol{<}}%
%BeginExpansion
$<$%
%EndExpansion
-- Scribe: Phil (on Monday)

\item The Cayley--Hamilton theorem (lec 5)
%TCIMACRO{\TEXTsymbol{<}}%
%BeginExpansion
$<$%
%EndExpansion
-- Scribe: Atharv (on Wednesday)

\item Sylvester's equation $AX-XB=C$ (lec 5--6)
%TCIMACRO{\TEXTsymbol{<}}%
%BeginExpansion
$<$%
%EndExpansion
-- Scribe: Hunter (on Friday)

\item Jordan canonical form (lec 6--7)
%TCIMACRO{\TEXTsymbol{<}}%
%BeginExpansion
$<$%
%EndExpansion
-- Scribe: Juliane 
\end{itemize}

Who scribes?

\subsection{The Gram--Schmidt process (cont'd)}

\begin{theorem}
[Gram--Schmidt process]Let $\left(  v_{1},v_{2},\ldots,v_{m}\right)  $ be a
linearly independent tuple of vectors in $\mathbb{C}^{n}$.

Then, there is an orthogonal tuple $\left(  z_{1},z_{2},\ldots,z_{m}\right)  $
of vectors in $\mathbb{C}^{n}$ that satisfies%
\[
\operatorname*{span}\left\{  v_{1},v_{2},\ldots,v_{j}\right\}
=\operatorname*{span}\left\{  z_{1},z_{2},\ldots,z_{j}\right\}
\ \ \ \ \ \ \ \ \ \ \text{for all }j\in\left[  m\right]  .
\]


Furthermore, such a tuple $\left(  z_{1},z_{2},\ldots,z_{m}\right)  $ can be
constructed by the following recursive process:

\begin{itemize}
\item For each $p\in\left[  m\right]  $, if the first $p-1$ entries
$z_{1},z_{2},\ldots,z_{p-1}$ of this tuple have already been constructed, then
we define the $p$-th entry $z_{p}$ by%
\[
z_{p}=v_{p}-\sum_{k=1}^{p-1}\dfrac{\left\langle v_{p},z_{k}\right\rangle
}{\left\langle z_{k},z_{k}\right\rangle }z_{k}.
\]
(Note that when $p=1$, the sum on the RHS is an empty sum, so this equality
simply becomes $z_{1}=v_{1}$.)
\end{itemize}
\end{theorem}

Roughly speaking, the claim of this theorem is that if we start with any
linearly independent tuple $\left(  v_{1},v_{2},\ldots,v_{m}\right)  $ of
vectors in $\mathbb{C}^{n}$, then we can make this tuple orthogonal by
tweaking it as follows:

\begin{itemize}
\item we leave $v_{1}$ unchanged;

\item we modify $v_{2}$ by subtracting an appropriate scalar multiple of
$v_{1}$;

\item we modify $v_{3}$ by subtracting an appropriate linear combination of
$v_{1}$ and $v_{2}$;

\item and so on.
\end{itemize}

The above recursive formula tells us which scalar multiples / linear
combinations to take.

\begin{example}%
\begin{align*}
z_{1}  &  =v_{1};\\
z_{2}  &  =v_{2}-\dfrac{\left\langle v_{2},z_{1}\right\rangle }{\left\langle
z_{1},z_{1}\right\rangle }z_{1};\\
z_{3}  &  =v_{3}-\dfrac{\left\langle v_{3},z_{1}\right\rangle }{\left\langle
z_{1},z_{1}\right\rangle }z_{1}-\dfrac{\left\langle v_{3},z_{2}\right\rangle
}{\left\langle z_{2},z_{2}\right\rangle }z_{2};\\
&  \ldots
\end{align*}


See the notes for an actual example (with numbers).
\end{example}

\begin{proof}
We must show three things:

\begin{enumerate}
\item The $z_{1},z_{2},\ldots,z_{m}$ constructed by the recursive process exist.

\item They satisfy $\operatorname*{span}\left\{  v_{1},v_{2},\ldots
,v_{j}\right\}  =\operatorname*{span}\left\{  z_{1},z_{2},\ldots
,z_{j}\right\}  $ for all $j\in\left[  m\right]  $.

\item The tuple $\left(  z_{1},z_{2},\ldots,z_{m}\right)  $ is orthogonal.
\end{enumerate}

We will first prove statements 1 and 2 in lockstep:

\begin{statement}
\textit{Claim 1:} For each $p\in\left\{  0,1,\ldots,m\right\}  $, the vectors
$z_{1},z_{2},\ldots,z_{p}$ are well-defined and satisfy%
\[
\operatorname*{span}\left\{  v_{1},v_{2},\ldots,v_{p}\right\}
=\operatorname*{span}\left\{  z_{1},z_{2},\ldots,z_{p}\right\}  .
\]

\end{statement}

[\textit{Proof of Claim 1:} We induct on $p$. The \textit{base case} ($p=0$)
is obvious.

For the \textit{induction step}, we fix $p\in\left[  m\right]  $, and we
assume that the claim holds for $p-1$. In other words, we assume that the
vectors $z_{1},z_{2},\ldots,z_{p-1}$ are well-defined and satisfy%
\begin{equation}
\operatorname*{span}\left\{  v_{1},v_{2},\ldots,v_{p-1}\right\}
=\operatorname*{span}\left\{  z_{1},z_{2},\ldots,z_{p-1}\right\}  . \label{1}%
\end{equation}


We must show that the vectors $z_{1},z_{2},\ldots,z_{p}$ are well-defined and
satisfy%
\begin{equation}
\operatorname*{span}\left\{  v_{1},v_{2},\ldots,v_{p}\right\}
=\operatorname*{span}\left\{  z_{1},z_{2},\ldots,z_{p}\right\}  . \label{2}%
\end{equation}


First, we use (1) to conclude that $\left(  z_{1},z_{2},\ldots,z_{p-1}\right)
$ is linearly independent, since their span has dimension $p-1$. Thus, in
particular, for each $k\in\left[  p-1\right]  $, we have $z_{k}\neq0$, so that
$\left\langle z_{k},z_{k}\right\rangle >0$. Thus, in the equality%
\begin{equation}
z_{p}=v_{p}-\sum_{k=1}^{p-1}\dfrac{\left\langle v_{p},z_{k}\right\rangle
}{\left\langle z_{k},z_{k}\right\rangle }z_{k}, \label{3}%
\end{equation}
the denominators are nonzero, so that $z_{p}$ is well-defined. Thus,
$z_{1},z_{2},\ldots,z_{p}$ are well-defined.

Now, (2) follows from (1) and (3). (See the notes for details.) So the
induction is complete, and Claim 1 is proven.]

Now it remains to prove statement 3:

\begin{statement}
\textit{Claim 2:} For any $j\in\left\{  0,1,\ldots,m\right\}  $, the tuple
$\left(  z_{1},z_{2},\ldots,z_{j}\right)  $ is orthogonal.
\end{statement}

[\textit{Proof of Claim 2:} Induct on $j$. The \textit{base case} is trivial.

\textit{Induction step:} Let $p\in\left[  m\right]  $. Assume (as the IH) that
Claim 2 holds for $j=p-1$. We must show that it holds for $j=p$.

By our IH, the tuple $\left(  z_{1},z_{2},\ldots,z_{p-1}\right)  $ is
orthogonal. It thus remains to show that $z_{p}$ is orthogonal to each $z_{a}$
with $a\in\left[  p-1\right]  $. To that purpose, we fix some $a\in\left[
p-1\right]  $, and we shall show that $\left\langle z_{p},z_{a}\right\rangle
=0$. To wit:%
\begin{align*}
\left\langle z_{p},z_{a}\right\rangle  &  =\left\langle v_{p}-\sum_{k=1}%
^{p-1}\dfrac{\left\langle v_{p},z_{k}\right\rangle }{\left\langle z_{k}%
,z_{k}\right\rangle }z_{k},z_{a}\right\rangle \ \ \ \ \ \ \ \ \ \ \left(
\text{since }z_{p}=v_{p}-\sum_{k=1}^{p-1}\dfrac{\left\langle v_{p}%
,z_{k}\right\rangle }{\left\langle z_{k},z_{k}\right\rangle }z_{k}\right) \\
&  =\left\langle v_{p},z_{a}\right\rangle -\sum_{k=1}^{p-1}\dfrac{\left\langle
v_{p},z_{k}\right\rangle }{\left\langle z_{k},z_{k}\right\rangle
}\underbrace{\left\langle z_{k},z_{a}\right\rangle }_{\substack{=0\text{
unless }k=a\\\text{(because the tuple }\left(  z_{1},z_{2},\ldots
,z_{p-1}\right)  \\\text{is orthogonal)}}}\\
&  =\left\langle v_{p},z_{a}\right\rangle -\dfrac{\left\langle v_{p}%
,z_{a}\right\rangle }{\left\langle z_{a},z_{a}\right\rangle }\left\langle
z_{a},z_{a}\right\rangle =0,
\end{align*}
so that $z_{p}\perp z_{a}$, as desired. This completes the induction.]

With Claim 1 and Claim 2 proved, the proof of the theorem is complete.
\end{proof}

What if the original tuple $\left(  v_{1},v_{2},\ldots,v_{m}\right)  $ is not
linearly independent? We can adapt our above theorem to this case:

\begin{theorem}
[Gram--Schmidt process, take 2]Let $\left(  v_{1},v_{2},\ldots,v_{m}\right)  $
be any tuple of vectors in $\mathbb{C}^{n}$ with $m\leq n$.

Then, there is an orthogonal tuple $\left(  z_{1},z_{2},\ldots,z_{m}\right)  $
of nonzero vectors in $\mathbb{C}^{n}$ that satisfies%
\[
\operatorname*{span}\left\{  v_{1},v_{2},\ldots,v_{j}\right\}  \subseteq
\operatorname*{span}\left\{  z_{1},z_{2},\ldots,z_{j}\right\}
\ \ \ \ \ \ \ \ \ \ \text{for all }j\in\left[  m\right]  .
\]


Furthermore, such a tuple $\left(  z_{1},z_{2},\ldots,z_{m}\right)  $ can be
constructed by the following recursive process:

\begin{itemize}
\item For each $p\in\left[  m\right]  $, if the first $p-1$ entries
$z_{1},z_{2},\ldots,z_{p-1}$ of this tuple have already been constructed, then
we define the $p$-th entry $z_{p}$ as follows:

\begin{itemize}
\item If $v_{p}-\sum\limits_{k=1}^{p-1}\dfrac{\left\langle v_{p}%
,z_{k}\right\rangle }{\left\langle z_{k},z_{k}\right\rangle }z_{k}\neq0$, then
we set%
\[
z_{p}:=v_{p}-\sum_{k=1}^{p-1}\dfrac{\left\langle v_{p},z_{k}\right\rangle
}{\left\langle z_{k},z_{k}\right\rangle }z_{k}.
\]


\item If $v_{p}-\sum\limits_{k=1}^{p-1}\dfrac{\left\langle v_{p}%
,z_{k}\right\rangle }{\left\langle z_{k},z_{k}\right\rangle }z_{k}=0$, then we
pick an arbitrary nonzero vector $b$ that is orthogonal to all of $z_{1}%
,z_{2},\ldots,z_{p-1}$ (such a $b$ exists, as previously shown), and we set%
\[
z_{p}:=b.
\]

\end{itemize}
\end{itemize}
\end{theorem}

\begin{proof}
This requires a variant of the proof of the previous theorem; see the notes.
\end{proof}

\begin{corollary}
Let $\left(  v_{1},v_{2},\ldots,v_{m}\right)  $ be any tuple of vectors in
$\mathbb{C}^{n}$ with $m\leq n$.

Then, there is an orthonormal tuple $\left(  z_{1},z_{2},\ldots,z_{m}\right)
$ of nonzero vectors in $\mathbb{C}^{n}$ that satisfies%
\[
\operatorname*{span}\left\{  v_{1},v_{2},\ldots,v_{j}\right\}  \subseteq
\operatorname*{span}\left\{  z_{1},z_{2},\ldots,z_{j}\right\}
\ \ \ \ \ \ \ \ \ \ \text{for all }j\in\left[  m\right]  .
\]

\end{corollary}

\begin{proof}
Use the previous theorem to obtain an orthogonal tuple $\left(  z_{1}%
,z_{2},\ldots,z_{m}\right)  $ with this property. Then, replace it by $\left(
\dfrac{1}{\left\vert \left\vert z_{1}\right\vert \right\vert }z_{1}%
,\ \ \dfrac{1}{\left\vert \left\vert z_{2}\right\vert \right\vert }%
z_{2},\ \ \ldots,\ \ \dfrac{1}{\left\vert \left\vert z_{m}\right\vert
\right\vert }z_{m}\right)  $.
\end{proof}

\subsection{QR factorization}

Recall that an isometry is a matrix whose columns form an orthonormal tuple.

\begin{theorem}
[QR factorization, isometry version]Let $A\in\mathbb{C}^{n\times m}$ satisfy
$n\geq m$. Then, there exists an isometry $Q\in\mathbb{C}^{n\times m}$ and an
upper-triangular matrix $R\in\mathbb{C}^{m\times m}$ such that $A=QR$.
\end{theorem}

The pair $\left(  Q,R\right)  $ in this theorem is called a \textbf{QR
factorization} of $A$.

\begin{example}
Let%
\[
A=\left(
\begin{array}
[c]{cccc}%
1 & 0 & 1 & 2\\
1 & -2 & 0 & 2\\
1 & 0 & 1 & 0\\
1 & -2 & 0 & 0
\end{array}
\right)  \in\mathbb{C}^{4\times4}.
\]
Then, one QR factorization of $A$ is%
\[
A=\underbrace{\left(
\begin{array}
[c]{cccc}%
1/2 & 1/2 & 1/2 & 1/2\\
1/2 & -1/2 & 1/2 & -1/2\\
1/2 & 1/2 & -1/2 & -1/2\\
1/2 & -1/2 & -1/2 & 1/2
\end{array}
\right)  }_{=Q}\underbrace{\left(
\begin{array}
[c]{cccc}%
2 & -2 & 1 & 0\\
0 & 2 & 1 & 2\\
0 & 0 & 0 & 0\\
0 & 0 & 0 & 2
\end{array}
\right)  }_{=R}.
\]
There are others.
\end{example}

\begin{proof}
[Proof of Theorem.]Let $A_{\bullet,1},A_{\bullet,2},\ldots,A_{\bullet,m}$
denote the $m$ columns of $A$.

Applying the previous corollary to $\left(  v_{1},v_{2},\ldots,v_{m}\right)
=\left(  A_{\bullet,1},A_{\bullet,2},\ldots,A_{\bullet,m}\right)  $, we
conclude that there exists an orthonormal tuple $\left(  q_{1},q_{2}%
,\ldots,q_{m}\right)  $ of vectors in $\mathbb{C}^{n}$ such that%
\[
\operatorname*{span}\left\{  A_{\bullet,1},A_{\bullet,2},\ldots,A_{\bullet
,j}\right\}  \subseteq\operatorname*{span}\left\{  q_{1},q_{2},\ldots
,q_{j}\right\}  \ \ \ \ \ \ \ \ \ \ \text{for all }j\in\left[  m\right]  .
\]
Consider this tuple. Let $Q\in\mathbb{C}^{n\times m}$ be the matrix whose
columns are $q_{1},q_{2},\ldots,q_{m}$. Then, $Q$ is an isometry.

Now, I claim that $A=QR$ for some upper-triangular $R$. Indeed, for each
$j\in\left[  m\right]  $, we have%
\[
A_{\bullet,j}\in\operatorname*{span}\left\{  A_{\bullet,1},A_{\bullet
,2},\ldots,A_{\bullet,j}\right\}  \subseteq\operatorname*{span}\left\{
q_{1},q_{2},\ldots,q_{j}\right\}  ,
\]
so that%
\[
A_{\bullet,j}=r_{1,j}q_{1}+r_{2,j}q_{2}+\cdots+r_{j,j}q_{j}%
\ \ \ \ \ \ \ \ \ \ \text{for some }r_{1,j},r_{2,j},\ldots,r_{j,j}%
\in\mathbb{C}.
\]
This shows that $A=QR$ for the upper-triangular matrix%
\[
R=\left(
\begin{array}
[c]{cccc}%
r_{1,1} & r_{1,2} & \cdots & r_{1,m}\\
0 & r_{2,2} & \cdots & r_{2,m}\\
\vdots & \vdots & \ddots & \vdots\\
0 & 0 & \cdots & r_{m,m}%
\end{array}
\right)  .
\]

\end{proof}

\section{Schur triangularization}

\subsection{Similarity of matrices}

\begin{definition}
Let $\mathbb{F}$ be a field. Let $A$ and $B$ be two $n\times n$-matrices over
$\mathbb{F}$. We say that $A$ is \textbf{similar} to $B$ if there exists an
invertible matrix $W\in\mathbb{F}^{n\times n}$ such that $B=WAW^{-1}$.
\end{definition}

\begin{example}
The matrix $\left(
\begin{array}
[c]{cc}%
1 & 1\\
1 & 1
\end{array}
\right)  $ is similar to the matrix $\left(
\begin{array}
[c]{cc}%
2 & 0\\
0 & 0
\end{array}
\right)  $, since
\[
\left(
\begin{array}
[c]{cc}%
2 & 0\\
0 & 0
\end{array}
\right)  =W\left(
\begin{array}
[c]{cc}%
1 & 1\\
1 & 1
\end{array}
\right)  W^{-1}\ \ \ \ \ \ \ \ \ \ \text{for }W=\left(
\begin{array}
[c]{cc}%
1 & 1\\
-1 & 1
\end{array}
\right)  .
\]

\end{example}

\begin{remark}
If you think of matrices as representing linear maps, then similarity has a
much more fundamental meaning: A matrix is similar to another if and only if
the two matrices represent the same endomorphism of $\mathbb{F}^{n}$ (i.e.,
linear map $\mathbb{F}^{n}\rightarrow\mathbb{F}^{n}$) with respect to two bases.
\end{remark}

The relation \textquotedblleft similar\textquotedblright\ is an equivalence
relation: i.e., it is

\begin{itemize}
\item \textbf{reflexive:} Any matrix $A\in\mathbb{F}^{n\times n}$ is similar
to itself.

\item \textbf{symmetric:} If $A$ is similar to $B$, then $B$ is similar to $A$.

\item \textbf{transitive:} If $A$ is similar to $B$, and $B$ is similar to
$C$, then $A$ is similar to $C$.
\end{itemize}

To check these, recall that $\left(  UV\right)  ^{-1}=V^{-1}U^{-1}$.

Because of the symmetry of the relation \textquotedblleft
similar\textquotedblright, we often say \textquotedblleft$A$ and $B$ are
similar\textquotedblright\ instead of saying \textquotedblleft$A$ is similar
to $B$\textquotedblright.

Similar matrices have a lot in common:

\begin{proposition}
Let $A$ and $B$ be two similar matrices. Then:

\textbf{(a)} $A$ and $B$ have the same rank.

\textbf{(b)} $A$ and $B$ have the same nullity.

\textbf{(c)} $A$ and $B$ have the same determinant.

\textbf{(d)} $A$ and $B$ have the same characteristic polynomial.

\textbf{(e)} $A$ and $B$ have the same eigenvalues, with the same algebraic
multiplicities and the same geometric multiplicities.

\textbf{(f)} For any $k\in\mathbb{N}$, the matrices $A^{k}$ and $B^{k}$ are similar.
\end{proposition}

\begin{proof}
See notes.
\end{proof}

There is a notation $A\sim B$ for \textquotedblleft$A$ and $B$ are
similar\textquotedblright.

\subsection{Unitary similarity}

\begin{definition}
Let $A$ and $B$ be two matrices in $\mathbb{C}^{n\times n}$. We say that $A$
is \textbf{unitary similar} to $B$ if there exists a \textbf{unitary} matrix
$W\in\operatorname*{U}\nolimits_{n}\left(  \mathbb{C}\right)  $ such that
$B=WA\underbrace{W^{\ast}}_{=W^{-1}}$.

We write \textquotedblleft$A\overset{\text{us}}{\sim}B$\textquotedblright\ for
\textquotedblleft$A$ is unitary similar to $B$\textquotedblright.
\end{definition}

\begin{example}
The matrix $\left(
\begin{array}
[c]{cc}%
1 & 1\\
1 & 1
\end{array}
\right)  $ is unitary similar to the matrix $\left(
\begin{array}
[c]{cc}%
2 & 0\\
0 & 0
\end{array}
\right)  $, since
\[
\left(
\begin{array}
[c]{cc}%
2 & 0\\
0 & 0
\end{array}
\right)  =W\left(
\begin{array}
[c]{cc}%
1 & 1\\
1 & 1
\end{array}
\right)  W^{\ast}\ \ \ \ \ \ \ \ \ \ \text{for }W=\dfrac{1}{\sqrt{2}}\left(
\begin{array}
[c]{cc}%
1 & 1\\
-1 & 1
\end{array}
\right)  \in\operatorname*{U}\nolimits_{2}\left(  \mathbb{C}\right)  .
\]

\end{example}

It is clear that any two unitary similar matrices are similar. The converse is
not true (exercise).

\textquotedblleft Unitary similar\textquotedblright, just like
\textquotedblleft similar\textquotedblright, is an equivalence relation.

\subsection{Schur triangularization}

We are now ready for one more matrix decomposition, the so-called
\textbf{Schur triangularization} (aka \textbf{Schur decomposition}):

\begin{theorem}
[Schur triangularization theorem]Let $A\in\mathbb{C}^{n\times n}$. Then, there
exists a unitary matrix $U\in\operatorname*{U}\nolimits_{n}\left(
\mathbb{C}\right)  $ and an upper-triangular matrix $T\in\mathbb{C}^{n\times
n}$ such that $A=UTU^{\ast}$.

In other words, $A$ is unitary similar to some upper-triangular matrix.
\end{theorem}

\begin{example}
Let $A=\left(
\begin{array}
[c]{cc}%
1 & 3\\
-3 & 7
\end{array}
\right)  $. Then, a Schur triangularization of $A$ is $A=UTU^{\ast}$, where%
\[
U=\dfrac{1}{\sqrt{2}}\left(
\begin{array}
[c]{cc}%
1 & -1\\
1 & 1
\end{array}
\right)  \ \ \ \ \ \ \ \ \ \ \text{and}\ \ \ \ \ \ \ \ \ \ T=\left(
\begin{array}
[c]{cc}%
4 & 6\\
0 & 4
\end{array}
\right)  .
\]

\end{example}

\begin{proof}
Next time.
\end{proof}


\end{document}