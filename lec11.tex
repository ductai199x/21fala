\documentclass[numbers=enddot,12pt,final,onecolumn,notitlepage]{scrartcl}%
\usepackage[headsepline,footsepline,manualmark]{scrlayer-scrpage}
\usepackage[all,cmtip]{xy}
\usepackage{amssymb}
\usepackage{amsmath}
\usepackage{amsthm}
\usepackage{framed}
\usepackage{comment}
\usepackage{color}
\usepackage{hyperref}
\usepackage[sc]{mathpazo}
\usepackage[T1]{fontenc}
\usepackage{tikz}
\usepackage{needspace}
\usepackage{tabls}
\usepackage{wasysym}
\usepackage{easytable}
\usepackage{pythonhighlight}
%TCIDATA{OutputFilter=latex2.dll}
%TCIDATA{Version=5.50.0.2960}
%TCIDATA{LastRevised=Friday, October 15, 2021 11:50:39}
%TCIDATA{SuppressPackageManagement}
%TCIDATA{<META NAME="GraphicsSave" CONTENT="32">}
%TCIDATA{<META NAME="SaveForMode" CONTENT="1">}
%TCIDATA{BibliographyScheme=Manual}
%TCIDATA{Language=American English}
%BeginMSIPreambleData
\providecommand{\U}[1]{\protect\rule{.1in}{.1in}}
%EndMSIPreambleData
\usetikzlibrary{arrows.meta}
\usetikzlibrary{chains}
\newcounter{exer}
\newcounter{exera}
\numberwithin{exer}{subsection}
\theoremstyle{definition}
\newtheorem{theo}{Theorem}[subsection]
\newenvironment{theorem}[1][]
{\begin{theo}[#1]\begin{leftbar}}
{\end{leftbar}\end{theo}}
\newtheorem{lem}[theo]{Lemma}
\newenvironment{lemma}[1][]
{\begin{lem}[#1]\begin{leftbar}}
{\end{leftbar}\end{lem}}
\newtheorem{prop}[theo]{Proposition}
\newenvironment{proposition}[1][]
{\begin{prop}[#1]\begin{leftbar}}
{\end{leftbar}\end{prop}}
\newtheorem{defi}[theo]{Definition}
\newenvironment{definition}[1][]
{\begin{defi}[#1]\begin{leftbar}}
{\end{leftbar}\end{defi}}
\newtheorem{remk}[theo]{Remark}
\newenvironment{remark}[1][]
{\begin{remk}[#1]\begin{leftbar}}
{\end{leftbar}\end{remk}}
\newtheorem{coro}[theo]{Corollary}
\newenvironment{corollary}[1][]
{\begin{coro}[#1]\begin{leftbar}}
{\end{leftbar}\end{coro}}
\newtheorem{conv}[theo]{Convention}
\newenvironment{convention}[1][]
{\begin{conv}[#1]\begin{leftbar}}
{\end{leftbar}\end{conv}}
\newtheorem{quest}[theo]{Question}
\newenvironment{question}[1][]
{\begin{quest}[#1]\begin{leftbar}}
{\end{leftbar}\end{quest}}
\newtheorem{warn}[theo]{Warning}
\newenvironment{warning}[1][]
{\begin{warn}[#1]\begin{leftbar}}
{\end{leftbar}\end{warn}}
\newtheorem{conj}[theo]{Conjecture}
\newenvironment{conjecture}[1][]
{\begin{conj}[#1]\begin{leftbar}}
{\end{leftbar}\end{conj}}
\newtheorem{exam}[theo]{Example}
\newenvironment{example}[1][]
{\begin{exam}[#1]\begin{leftbar}}
{\end{leftbar}\end{exam}}
\newtheorem{exmp}[exer]{Exercise}
\newenvironment{exercise}[1][]
{\begin{exmp}[#1]\begin{leftbar}}
{\end{leftbar}\end{exmp}}
\newenvironment{statement}{\begin{quote}}{\end{quote}}
\newenvironment{fineprint}{\medskip \begin{small}}{\end{small} \medskip}
\iffalse
\newenvironment{proof}[1][Proof]{\noindent\textbf{#1.} }{\ \rule{0.5em}{0.5em}}
\newenvironment{question}[1][Question]{\noindent\textbf{#1.} }{\ \rule{0.5em}{0.5em}}
\newenvironment{warning}[1][Warning]{\noindent\textbf{#1.} }{\ \rule{0.5em}{0.5em}}
\newenvironment{teachingnote}[1][Teaching note]{\noindent\textbf{#1.} }{\ \rule{0.5em}{0.5em}}
\fi
\let\sumnonlimits\sum
\let\prodnonlimits\prod
\let\cupnonlimits\bigcup
\let\capnonlimits\bigcap
\renewcommand{\sum}{\sumnonlimits\limits}
\renewcommand{\prod}{\prodnonlimits\limits}
\renewcommand{\bigcup}{\cupnonlimits\limits}
\renewcommand{\bigcap}{\capnonlimits\limits}
\setlength\tablinesep{3pt}
\setlength\arraylinesep{3pt}
\setlength\extrarulesep{3pt}
\voffset=0cm
\hoffset=-0.7cm
\setlength\textheight{22.5cm}
\setlength\textwidth{15.5cm}
\newcommand\arxiv[1]{\href{http://www.arxiv.org/abs/#1}{\texttt{arXiv:#1}}}
\newenvironment{verlong}{}{}
\newenvironment{vershort}{}{}
\newenvironment{noncompile}{}{}
\newenvironment{teachingnote}{}{}
\excludecomment{verlong}
\includecomment{vershort}
\excludecomment{noncompile}
\excludecomment{teachingnote}
\newcommand{\CC}{\mathbb{C}}
\newcommand{\RR}{\mathbb{R}}
\newcommand{\QQ}{\mathbb{Q}}
\newcommand{\NN}{\mathbb{N}}
\newcommand{\ZZ}{\mathbb{Z}}
\newcommand{\KK}{\mathbb{K}}
\newcommand{\id}{\operatorname{id}}
\newcommand{\lcm}{\operatorname{lcm}}
\newcommand{\rev}{\operatorname{rev}}
\newcommand{\powset}[2][]{\ifthenelse{\equal{#2}{}}{\mathcal{P}\left(#1\right)}{\mathcal{P}_{#1}\left(#2\right)}}
\newcommand{\set}[1]{\left\{ #1 \right\}}
\newcommand{\abs}[1]{\left| #1 \right|}
\newcommand{\tup}[1]{\left( #1 \right)}
\newcommand{\ive}[1]{\left[ #1 \right]}
\newcommand{\floor}[1]{\left\lfloor #1 \right\rfloor}
\newcommand{\lf}[2]{#1^{\underline{#2}}}
\newcommand{\underbrack}[2]{\underbrace{#1}_{\substack{#2}}}
\newcommand{\horrule}[1]{\rule{\linewidth}{#1}}
\newcommand{\are}{\ar@{-}}
\newcommand{\nnn}{\nonumber\\}
\newcommand{\sslash}{\mathbin{/\mkern-6mu/}}
\newcommand{\numboxed}[2]{\underbrace{\boxed{#1}}_{\text{box } #2}}
\newcommand{\ig}[2]{\includegraphics[scale=#1]{#2.png}}
\newcommand{\UNFINISHED}{\begin{center} \Huge{\textbf{Unfinished material begins here.}} \end{center} }
\iffalse
\NOEXPAND{\today}{\today}
\NOEXPAND{\sslash}{\sslash}
\NOEXPAND{\numboxed}[2]{\numboxed}
\NOEXPAND{\UNFINISHED}{\UNFINISHED}
\fi
\ihead{Math 504 notes}
\ohead{page \thepage}
\cfoot{\today}
\begin{document}

\title{Math 504: Advanced Linear Algebra}
\author{Hugo Woerdeman, with edits by Darij Grinberg\thanks{Drexel University, Korman
Center, 15 S 33rd Street, Philadelphia PA, 19104, USA}}
\date{\today\ (unfinished!)}
\maketitle
\tableofcontents

\section*{Math 504 Lecture 11}

\section{Jordan canonical (aka normal) form (cont'd)}

\subsection{Powers and the Jordan canonical form}

Let $n\in\mathbb{N}$ and $A\in\mathbb{C}^{n\times n}$. Assume that we know the
JCF $J$ of $A$ and an invertible matrix $S$ such that
\[
A=SJS^{-1}.
\]
Then, it is fairly easy to compute all powers $A^{m}$ of $A$. Indeed, recall that

\begin{itemize}
\item $\left(  SJS^{-1}\right)  ^{m}=SJ^{m}S^{-1}$ for any $m\in\mathbb{N}$.

\item $\left(
\begin{array}
[c]{cccc}%
A_{1} &  &  & \\
& A_{2} &  & \\
&  & \ddots & \\
&  &  & A_{k}%
\end{array}
\right)  ^{m}=\left(
\begin{array}
[c]{cccc}%
A_{1}^{m} &  &  & \\
& A_{2}^{m} &  & \\
&  & \ddots & \\
&  &  & A_{k}^{m}%
\end{array}
\right)  $ for any $m\in\mathbb{N}$.
\end{itemize}

Thus, it suffices to compute the $m$-th power of any Jordan cell $J_{k}\left(
\lambda\right)  $.

So let us consider a Jordan cell
\[
C:=J_{5}\left(  \lambda\right)  =\left(
\begin{array}
[c]{ccccc}%
\lambda & 1 &  &  & \\
& \lambda & 1 &  & \\
&  & \lambda & 1 & \\
&  &  & \lambda & 1\\
&  &  &  & \lambda
\end{array}
\right)  .
\]
Then,%
\begin{align*}
C^{2}  & =\left(
\begin{array}
[c]{ccccc}%
\lambda^{2} & 2\lambda & 1 &  & \\
& \lambda^{2} & 2\lambda & 1 & \\
&  & \lambda^{2} & 2\lambda & 1\\
&  &  & \lambda^{2} & 2\lambda\\
&  &  &  & \lambda^{2}%
\end{array}
\right)  ;\ \ \ \ \ \ \ \ \ \ C^{3}=\left(
\begin{array}
[c]{ccccc}%
\lambda^{3} & 3\lambda^{2} & 3\lambda & 1 & \\
& \lambda^{3} & 3\lambda^{2} & 3\lambda & 1\\
&  & \lambda^{3} & 3\lambda^{2} & 3\lambda\\
&  &  & \lambda^{3} & 3\lambda^{2}\\
&  &  &  & \lambda^{3}%
\end{array}
\right)  ;\\
C^{4}  & =\left(
\begin{array}
[c]{ccccc}%
\lambda^{4} & 4\lambda^{3} & 6\lambda^{2} & 4\lambda & 1\\
& \lambda^{4} & 4\lambda^{3} & 6\lambda^{2} & 4\lambda\\
&  & \lambda^{4} & 4\lambda^{3} & 6\lambda^{2}\\
&  &  & \lambda^{4} & 4\lambda^{3}\\
&  &  &  & \lambda^{4}%
\end{array}
\right)  ;\ \ \ \ \ \ \ \ \ \ C^{5}=\left(
\begin{array}
[c]{ccccc}%
\lambda^{5} & 5\lambda^{4} & 10\lambda^{3} & 10\lambda^{2} & 5\lambda\\
& \lambda^{5} & 5\lambda^{4} & 10\lambda^{3} & 10\lambda^{2}\\
&  & \lambda^{5} & 5\lambda^{4} & 10\lambda^{3}\\
&  &  & \lambda^{5} & 5\lambda^{4}\\
&  &  &  & \lambda^{5}%
\end{array}
\right)  .
\end{align*}
In general:

\begin{theorem}
Let $k>0$ and $\lambda\in\mathbb{C}$. Let $C=J_{k}\left(  \lambda\right)  $.
Let $m\in\mathbb{N}$. Then, $C^{m}$ is the upper-triangular $k\times k$-matrix
whose $\left(  i,j\right)  $-th entry is $\dbinom{m}{j-i}\lambda^{m-j+i}$.
(Here, we follow the convention that $\dbinom{n}{\ell}:=0$ when $\ell
\notin\mathbb{N}$. Also, recall that $\dbinom{n}{\ell}=0$ when $n\in
\mathbb{N}$ and $\ell>n$.)
\end{theorem}

\begin{proof}
[First proof.] Induct on $m$ and use $C^{m}=CC^{m-1}$ as well as Pascal's
recursion%
\[
\dbinom{n}{\ell}=\dbinom{n-1}{\ell}+\dbinom{n-1}{\ell-1}.
\]

\end{proof}

\begin{proof}
[Second proof.] Set $B:=J_{k}\left(  0\right)  =\left(
\begin{array}
[c]{ccccc}
& 1 &  &  & \\
&  & 1 &  & \\
&  &  & 1 & \\
&  &  &  & 1\\
&  &  &  &
\end{array}
\right)  $. We know all powers of $B$ already: $B^{i}$ has $1$s $i$ steps
above the main diagonal, and $0$s everywhere else.

However, $C=B+\lambda I_{k}$. The matrices $\lambda I_{k}$ and $B$ commute
(i.e., we have $B\cdot\lambda I_{k}=\lambda I_{k}\cdot B$). It is a general
fact that if $X$ and $Y$ are two commuting $n\times n$-matrices, then the
binomial formula%
\[
\left(  X+Y\right)  ^{m}=\sum_{i=0}^{m}\dbinom{m}{i}X^{i}Y^{m-i}.
\]
(This can be proved in the same way as for numbers, because the commutativity
of $X$ and $Y$ lets you move any $X$es past any $Y$s.) Applying this formula
to $X=B$ and $Y=\lambda I_{k}$, we obtain%
\begin{align*}
\left(  B+\lambda I_{k}\right)  ^{m}  & =\sum_{i=0}^{m}\dbinom{m}{i}%
B^{i}\underbrace{\left(  \lambda I_{k}\right)  ^{m-i}}_{=\lambda^{m-i}I_{k}%
}=\sum_{i=0}^{m}\dbinom{m}{i}\lambda^{m-i}B^{i}\\
& =\left(
\begin{array}
[c]{ccccccc}%
\lambda^{m} & \dbinom{m}{1}\lambda^{m-1} & \dbinom{m}{2}\lambda^{m-2} & \cdots
& \cdots & \cdots & \cdots\\
& \lambda^{m} & \dbinom{m}{1}\lambda^{m-1} & \dbinom{m}{2}\lambda^{m-2} &
\cdots & \cdots & \cdots\\
&  & \lambda^{m} & \dbinom{m}{1}\lambda^{m-1} & \dbinom{m}{2}\lambda^{m-2} &
\cdots & \cdots\\
&  &  & \lambda^{m} & \dbinom{m}{1}\lambda^{m-1} & \cdots & \cdots\\
&  &  &  & \lambda^{m} & \cdots & \vdots\\
&  &  &  &  & \ddots & \vdots\\
&  &  &  &  &  & \lambda^{m}%
\end{array}
\right)  ,
\end{align*}
which is precisely the matrix claimed in the theorem.
\end{proof}

Now we know how to take powers of Jordan cells, and therefore how to take
powers of any matrix that we know how to bring to a Jordan canonical form.

\begin{corollary}
Let $A\in\mathbb{C}^{n\times n}$. Then, $\lim\limits_{m\rightarrow\infty}%
A^{m}=0$ if and only if all eigenvalues of $A$ have absolute value $<1$.
\end{corollary}

\begin{proof}
$\Longrightarrow:$ Suppose that $\lim\limits_{m\rightarrow\infty}A^{m}=0$, but
$A$ has an eigenvalue $\lambda$ of absolute value $\geq1$. We want a contradiction.

Consider a nonzero eigenvector $x$ for eigenvalue $\lambda$. Thus, $Ax=\lambda
x$. Then, $A^{2}x=\lambda^{2}x$ (since $A^{2}x=A\underbrace{Ax}_{=\lambda
x}=\lambda\underbrace{Ax}_{=\lambda x}=\lambda\lambda x=\lambda^{2}x$) and
similarly $A^{3}x=\lambda^{3}x$ and $A^{4}x=\lambda^{4}x$ and so on. Thus,
\[
A^{m}x=\lambda^{m}x\ \ \ \ \ \ \ \ \ \ \text{for each }m\in\mathbb{N}.
\]


Now, as $m\rightarrow\infty$, the vector $A^{m}x$ goes to $0$ (since
$A^{m}\rightarrow0$), but the vector $\lambda^{m}x$ does not (since $x\neq0$
and $\left\vert \lambda\right\vert \geq1$). Contradiction.

$\Longleftarrow:$ Suppose that all eigenvalues of $A$ have absolute value $<1$.

Let $A=SJS^{-1}$ be the Jordan canonical form of $A$. Write $J$ as $\left(
\begin{array}
[c]{ccc}%
J_{1} &  & \\
& \ddots & \\
&  & J_{p}%
\end{array}
\right)  $, where each $J_{i}$ is a Jordan cell.

It suffices to show that $\lim\limits_{n\rightarrow\infty}J_{i}^{m}=0$.

Write $J_{i}$ as $J_{k}\left(  \lambda\right)  $, with $\left\vert
\lambda\right\vert <1$. The preceding theorem then tells us that%
\[
J_{i}^{m}=\left(
\begin{array}
[c]{ccccccc}%
\lambda^{m} & \dbinom{m}{1}\lambda^{m-1} & \dbinom{m}{2}\lambda^{m-2} & \cdots
& \cdots & \cdots & \cdots\\
& \lambda^{m} & \dbinom{m}{1}\lambda^{m-1} & \dbinom{m}{2}\lambda^{m-2} &
\cdots & \cdots & \cdots\\
&  & \lambda^{m} & \dbinom{m}{1}\lambda^{m-1} & \dbinom{m}{2}\lambda^{m-2} &
\cdots & \cdots\\
&  &  & \lambda^{m} & \dbinom{m}{1}\lambda^{m-1} & \cdots & \cdots\\
&  &  &  & \lambda^{m} & \cdots & \vdots\\
&  &  &  &  & \ddots & \vdots\\
&  &  &  &  &  & \lambda^{m}%
\end{array}
\right)  .
\]
Look closely at this matrix. We need to show that for each $i,j\in\left[
k\right]  $, we have%
\[
\lim\limits_{m\rightarrow\infty}\dbinom{m}{j-i}\lambda^{m-j+i}=0.
\]
However, this is a standard asymptotics argument:%
\[
\underbrace{\dbinom{m}{j-i}}_{\substack{=\dfrac{m\left(  m-1\right)  \left(
m-2\right)  \cdots\left(  m-j+i\right)  }{\left(  j-i\right)  !}\\\text{(for
}i\leq j\text{; otherwise the claim is trivial)}}}\underbrace{\lambda^{m-j+i}%
}_{\substack{\text{exponential in }m\text{,}\\\text{with quotient }%
\lambda\text{ having absolute value }\left\vert \lambda\right\vert
<1}}\rightarrow0
\]
because exponential functions with a quotient of absolute value $<1$ converge
to $0$ faster than polynomials can go to $\infty$. 
\end{proof}

\subsection{The minimal polynomial}

\textbf{Recall:} A polynomial $p\left(  t\right)  \in\mathbb{F}\left[
t\right]  $ (where $\mathbb{F}$ is any field, and $t$ is an indeterminate) is
said to be \textbf{monic} if its leading coefficient is $1$ -- that is, if it
can be written in the form%
\[
p\left(  t\right)  =t^{m}+p_{m-1}t^{m-1}+p_{m-2}t^{m-2}+\cdots+p_{0}%
t^{0}\ \ \ \ \ \ \ \ \ \ \text{for some }m\in\mathbb{N}\text{ and }p_{0}%
,p_{1},\ldots,p_{m-1}\in\mathbb{F}.
\]


\begin{definition}
Given a matrix $A\in\mathbb{F}^{n\times n}$ and a polynomial $p\left(
t\right)  \in\mathbb{F}\left[  t\right]  $, we saay that $p\left(  t\right)  $
\textbf{annihilates} $A$ if $p\left(  A\right)  =0$.
\end{definition}

The Cayley--Hamilton theorem says that the characteristic polynomial $p_{A}$
of a square matrix $A$ always annihilates $A$. However, often there are
matrices that are annihilated by other -- sometimes simpler -- polynomials.

\begin{example}
The identity matrix $I_{n}$ is annihilated by the polynomial $p\left(
t\right)  :=t-1$, because%
\[
p\left(  I_{n}\right)  =I_{n}-I_{n}=0.
\]

\end{example}

\begin{example}
The matrix $\left(
\begin{array}
[c]{cc}%
0 & 1\\
0 & 0
\end{array}
\right)  $ is annihilated by the polynomial $p\left(  t\right)  :=t^{2}$,
since its square is $0$.
\end{example}

\begin{example}
The matrix $\left(
\begin{array}
[c]{ccc}%
0 & 1 & 0\\
0 & 0 & 0\\
0 & 0 & 0
\end{array}
\right)  $ is annihilated by the polynomial $p\left(  t\right)  :=t^{2}$,
since its square is $0$.
\end{example}

\begin{example}
The diagonal matrix $\left(
\begin{array}
[c]{ccc}%
2 &  & \\
& 2 & \\
&  & 3
\end{array}
\right)  $ is annihilated by the polynomial $p\left(  t\right)  :=\left(
t-2\right)  \left(  t-3\right)  $, since%
\begin{align*}
& \left(  \left(
\begin{array}
[c]{ccc}%
2 &  & \\
& 2 & \\
&  & 3
\end{array}
\right)  -2I_{n}\right)  \left(  \left(
\begin{array}
[c]{ccc}%
2 &  & \\
& 2 & \\
&  & 3
\end{array}
\right)  -3I_{n}\right)  \\
& =\left(
\begin{array}
[c]{ccc}
&  & \\
&  & \\
&  & 1
\end{array}
\right)  \left(
\begin{array}
[c]{ccc}%
-1 &  & \\
& -1 & \\
&  &
\end{array}
\right)  =0.
\end{align*}

\end{example}

\begin{theorem}
Given an $n\times n$-matrix $A\in\mathbb{C}^{n\times n}$. Then, there is a
\textbf{unique} monic polynomial $q_{A}\left(  t\right)  $ of minimum degree
that annihilates $A$.
\end{theorem}

\begin{proof}
The Cayley--Hamilton theorem shows that $p_{A}$ annihilates $A$. Since $p_{A}$
is monic, we thus conclude that there exists \textbf{some} monic polynomial
that annihilates $A$. Hence, there exists such a polynomial of minimum degree.

It remains to show that it is unique. To do so, we let $q_{A}$ and
$\widetilde{q}_{A}$ be two monic polynomials of minimum degree that annihilate
$A$. Our goal then is to show that $q_{A}=\widetilde{q}_{A}$.

Indeed, if $q_{A}\neq\widetilde{q}_{A}$, then the difference $q_{A}%
-\widetilde{q}_{A}$ is a polynomial of smaller degree that annihilates $A$
(indeed, it is of smaller degree because $q_{A}$ and $\widetilde{q}_{A}$ are
monic of the same degree and thus lose their leading terms upon subtraction;
it annihilates $A$ because $\left(  q_{A}-\widetilde{q}_{A}\right)  \left(
A\right)  =q_{A}\left(  A\right)  -\widetilde{q}_{A}\left(  A\right)
=0-0=0$). By scaling this difference appropriately, we can make it monic
(since it is nonzero), and then get a contradiction to the minimality of
$q_{A}$'s degree. This concludes the proof.
\end{proof}

\begin{definition}
Let $A\in\mathbb{C}^{n\times n}$ be an $n\times n$-matrix. The preceding
theorem shows that there is a \textbf{unique} monic polynomial $q_{A}\left(
t\right)  $ of minimum degree that annihilates $A$. This unique polynomial
will be denoted $q_{A}\left(  t\right)  $ and will be called the
\textbf{minimal polynomial} of $A$.
\end{definition}

\begin{example}
Let $A$ be the diagonal matrix $\left(
\begin{array}
[c]{ccc}%
2 &  & \\
& 2 & \\
&  & 3
\end{array}
\right)  $. Then,%
\[
q_{A}\left(  t\right)  =\left(  t-2\right)  \left(  t-3\right)  .
\]
Indeed, we already know that the monic polynomial $\left(  t-2\right)  \left(
t-3\right)  $ annihilates $A$. If there was any monic polynomial of smaller
degree that would annihilate $A$, then it would have the form $t-\lambda$ for
some $\lambda\in\mathbb{C}$, but $\lambda$ cannot be $2$ and $3$ at the same time.

For comparison: $p_{A}\left(  t\right)  =\left(  t-2\right)  ^{2}\left(
t-3\right)  $.
\end{example}

\begin{theorem}
Let $A\in\mathbb{C}^{n\times n}$ be an $n\times n$-matrix. Let $f\left(
t\right)  \in\mathbb{C}\left[  t\right]  $ be any polynomial. Then, $f$
annihilates $A$ if and only if $f$ is a multiple of $q_{A}$ (that is,
$f\left(  t\right)  =q_{A}\left(  t\right)  \cdot g\left(  t\right)  $ for
some polynomial $g\left(  t\right)  \in\mathbb{C}\left[  t\right]  $).
\end{theorem}

\begin{proof}
$\Longrightarrow:$ Assume that $f$ annihilates $A$. Thus, $f\left(  A\right)
=0$. WLOG, assume that $f\neq0$. Thus, we can make $f$ monic by scaling it.
Thus, $\deg f\geq\deg q_{A}$ (since $q_{A}$ had minimum degree). Hence, we can
divide $f$ by $q_{A}$ with remainder, obtaining%
\[
f\left(  t\right)  =q_{A}\left(  t\right)  \cdot g\left(  t\right)  +r\left(
t\right)  ,
\]
where $g\left(  t\right)  $ and $r\left(  t\right)  $ are two polynomials with
$\deg r<\deg q_{A}$. Substituting $A$ for $t$ in this equality, we obtain%
\[
f\left(  A\right)  =\underbrace{q_{A}\left(  A\right)  }%
_{\substack{=0\\\text{(since }q_{A}\text{ annihilates }A\text{)}}}\cdot
g\left(  A\right)  +r\left(  A\right)  =r\left(  A\right)  ,
\]
so that $r\left(  A\right)  =f\left(  A\right)  =0$. In other words, $r$
annihilates $A$. Since $\deg r<\deg q_{A}$, this entails that $r=0$ (since
otherwise, scaling $r$ to make it monic would contradict the minimality of
$\deg q_{A}$). Thus,%
\[
f\left(  t\right)  =q_{A}\left(  t\right)  \cdot g\left(  t\right)
+\underbrace{r\left(  t\right)  }_{=0}=q_{A}\left(  t\right)  \cdot g\left(
t\right)  .
\]
Thus, $f$ is a multiple of $q_{A}$.

$\Longleftarrow:$ Easy and LTTR.
\end{proof}

\begin{corollary}
Let $A\in\mathbb{C}^{n\times n}$ be a matrix. Then, $q_{A}\left(  t\right)
\mid p_{A}\left(  t\right)  $.
\end{corollary}

\begin{proof}
Apply the previous theorem to $f=p_{A}$, recalling that $p_{A}$ annihilates
$A$.
\end{proof}

The corollary yields that any root of $q_{A}$ must be a root of $p_{A}$, that
is, an eigenvalue of $A$. Conversely, we can show that any eigenvalue of $A$
is a root of $q_{A}$ (but we don't know with which multiplicity):

\begin{proposition}
If $\lambda\in\sigma\left(  A\right)  $, then $q_{A}\left(  \lambda\right)
=0$.
\end{proposition}

\begin{proof}
Let $\lambda\in\sigma\left(  A\right)  $. Thus, there exists a nonzero
eigenvector $x$ for $\lambda$.

Then, $Ax=\lambda x$. As we have seen above, this entails $A^{m}x=\lambda
^{m}x$ for each $m\in\mathbb{N}$. Therefore, $f\left(  A\right)  x=f\left(
\lambda\right)  x$ for each polynomial $f\left(  t\right)  \in\mathbb{C}%
\left[  t\right]  $ (because you can write $f\left(  t\right)  $ as
$f_{0}t^{0}+f_{1}t^{1}+\cdots+f_{p}t^{p}$, and then apply $A^{m}x=\lambda
^{m}x$ to each of $m=0,1,\ldots,p$). Hence, $q_{A}\left(  A\right)
x=q_{A}\left(  \lambda\right)  x$, so that%
\[
q_{A}\left(  \lambda\right)  x=\underbrace{q_{A}\left(  A\right)
}_{\substack{=0\\\text{(since }q_{A}\text{ annihilates }A\text{)}}}x=0.
\]
Since $x\neq0$, this entails $q_{A}\left(  \lambda\right)  =0$, qed.
\end{proof}


\end{document}