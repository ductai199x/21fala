\documentclass[numbers=enddot,12pt,final,onecolumn,notitlepage]{scrartcl}%
\usepackage[headsepline,footsepline,manualmark]{scrlayer-scrpage}
\usepackage[all,cmtip]{xy}
\usepackage{amssymb}
\usepackage{amsmath}
\usepackage{amsthm}
\usepackage{framed}
\usepackage{comment}
\usepackage{color}
\usepackage{hyperref}
\usepackage[sc]{mathpazo}
\usepackage[T1]{fontenc}
\usepackage{tikz}
\usepackage{needspace}
\usepackage{tabls}
\usepackage{wasysym}
\usepackage{easytable}
\usepackage{pythonhighlight}
%TCIDATA{OutputFilter=latex2.dll}
%TCIDATA{Version=5.50.0.2960}
%TCIDATA{LastRevised=Friday, October 29, 2021 11:51:08}
%TCIDATA{SuppressPackageManagement}
%TCIDATA{<META NAME="GraphicsSave" CONTENT="32">}
%TCIDATA{<META NAME="SaveForMode" CONTENT="1">}
%TCIDATA{BibliographyScheme=Manual}
%TCIDATA{Language=American English}
%BeginMSIPreambleData
\providecommand{\U}[1]{\protect\rule{.1in}{.1in}}
%EndMSIPreambleData
\usetikzlibrary{arrows.meta}
\usetikzlibrary{chains}
\newcounter{exer}
\newcounter{exera}
\numberwithin{exer}{subsection}
\theoremstyle{definition}
\newtheorem{theo}{Theorem}[subsection]
\newenvironment{theorem}[1][]
{\begin{theo}[#1]\begin{leftbar}}
{\end{leftbar}\end{theo}}
\newtheorem{lem}[theo]{Lemma}
\newenvironment{lemma}[1][]
{\begin{lem}[#1]\begin{leftbar}}
{\end{leftbar}\end{lem}}
\newtheorem{prop}[theo]{Proposition}
\newenvironment{proposition}[1][]
{\begin{prop}[#1]\begin{leftbar}}
{\end{leftbar}\end{prop}}
\newtheorem{defi}[theo]{Definition}
\newenvironment{definition}[1][]
{\begin{defi}[#1]\begin{leftbar}}
{\end{leftbar}\end{defi}}
\newtheorem{remk}[theo]{Remark}
\newenvironment{remark}[1][]
{\begin{remk}[#1]\begin{leftbar}}
{\end{leftbar}\end{remk}}
\newtheorem{coro}[theo]{Corollary}
\newenvironment{corollary}[1][]
{\begin{coro}[#1]\begin{leftbar}}
{\end{leftbar}\end{coro}}
\newtheorem{conv}[theo]{Convention}
\newenvironment{convention}[1][]
{\begin{conv}[#1]\begin{leftbar}}
{\end{leftbar}\end{conv}}
\newtheorem{quest}[theo]{Question}
\newenvironment{question}[1][]
{\begin{quest}[#1]\begin{leftbar}}
{\end{leftbar}\end{quest}}
\newtheorem{warn}[theo]{Warning}
\newenvironment{warning}[1][]
{\begin{warn}[#1]\begin{leftbar}}
{\end{leftbar}\end{warn}}
\newtheorem{conj}[theo]{Conjecture}
\newenvironment{conjecture}[1][]
{\begin{conj}[#1]\begin{leftbar}}
{\end{leftbar}\end{conj}}
\newtheorem{exam}[theo]{Example}
\newenvironment{example}[1][]
{\begin{exam}[#1]\begin{leftbar}}
{\end{leftbar}\end{exam}}
\newtheorem{exmp}[exer]{Exercise}
\newenvironment{exercise}[1][]
{\begin{exmp}[#1]\begin{leftbar}}
{\end{leftbar}\end{exmp}}
\newenvironment{statement}{\begin{quote}}{\end{quote}}
\newenvironment{fineprint}{\medskip \begin{small}}{\end{small} \medskip}
\iffalse
\newenvironment{proof}[1][Proof]{\noindent\textbf{#1.} }{\ \rule{0.5em}{0.5em}}
\newenvironment{question}[1][Question]{\noindent\textbf{#1.} }{\ \rule{0.5em}{0.5em}}
\newenvironment{warning}[1][Warning]{\noindent\textbf{#1.} }{\ \rule{0.5em}{0.5em}}
\newenvironment{teachingnote}[1][Teaching note]{\noindent\textbf{#1.} }{\ \rule{0.5em}{0.5em}}
\fi
\let\sumnonlimits\sum
\let\prodnonlimits\prod
\let\cupnonlimits\bigcup
\let\capnonlimits\bigcap
\renewcommand{\sum}{\sumnonlimits\limits}
\renewcommand{\prod}{\prodnonlimits\limits}
\renewcommand{\bigcup}{\cupnonlimits\limits}
\renewcommand{\bigcap}{\capnonlimits\limits}
\setlength\tablinesep{3pt}
\setlength\arraylinesep{3pt}
\setlength\extrarulesep{3pt}
\voffset=0cm
\hoffset=-0.7cm
\setlength\textheight{22.5cm}
\setlength\textwidth{15.5cm}
\newcommand\arxiv[1]{\href{http://www.arxiv.org/abs/#1}{\texttt{arXiv:#1}}}
\newenvironment{verlong}{}{}
\newenvironment{vershort}{}{}
\newenvironment{noncompile}{}{}
\newenvironment{teachingnote}{}{}
\excludecomment{verlong}
\includecomment{vershort}
\excludecomment{noncompile}
\excludecomment{teachingnote}
\newcommand{\CC}{\mathbb{C}}
\newcommand{\RR}{\mathbb{R}}
\newcommand{\QQ}{\mathbb{Q}}
\newcommand{\NN}{\mathbb{N}}
\newcommand{\ZZ}{\mathbb{Z}}
\newcommand{\KK}{\mathbb{K}}
\newcommand{\id}{\operatorname{id}}
\newcommand{\lcm}{\operatorname{lcm}}
\newcommand{\rev}{\operatorname{rev}}
\newcommand{\powset}[2][]{\ifthenelse{\equal{#2}{}}{\mathcal{P}\left(#1\right)}{\mathcal{P}_{#1}\left(#2\right)}}
\newcommand{\set}[1]{\left\{ #1 \right\}}
\newcommand{\abs}[1]{\left| #1 \right|}
\newcommand{\tup}[1]{\left( #1 \right)}
\newcommand{\ive}[1]{\left[ #1 \right]}
\newcommand{\floor}[1]{\left\lfloor #1 \right\rfloor}
\newcommand{\lf}[2]{#1^{\underline{#2}}}
\newcommand{\underbrack}[2]{\underbrace{#1}_{\substack{#2}}}
\newcommand{\horrule}[1]{\rule{\linewidth}{#1}}
\newcommand{\are}{\ar@{-}}
\newcommand{\nnn}{\nonumber\\}
\newcommand{\sslash}{\mathbin{/\mkern-6mu/}}
\newcommand{\numboxed}[2]{\underbrace{\boxed{#1}}_{\text{box } #2}}
\newcommand{\ig}[2]{\includegraphics[scale=#1]{#2.png}}
\newcommand{\UNFINISHED}{\begin{center} \Huge{\textbf{Unfinished material begins here.}} \end{center} }
\iffalse
\NOEXPAND{\today}{\today}
\NOEXPAND{\sslash}{\sslash}
\NOEXPAND{\numboxed}[2]{\numboxed}
\NOEXPAND{\UNFINISHED}{\UNFINISHED}
\fi
\ihead{Math 504 notes}
\ohead{page \thepage}
\cfoot{\today}
\begin{document}

\title{Math 504: Advanced Linear Algebra}
\author{Hugo Woerdeman, with edits by Darij Grinberg\thanks{Drexel University, Korman
Center, 15 S 33rd Street, Philadelphia PA, 19104, USA}}
\date{\today\ (unfinished!)}
\maketitle
\tableofcontents

\section*{Math 504 Lecture 17}

\section{Hermitian matrices (cont'd)}

\subsection{Rayleigh quotients (cont'd)}

Recall:

\begin{definition}
Let $A\in\mathbb{C}^{n\times n}$ be a Hermitian matrix, and $x\in
\mathbb{C}^{n}$ be a nonzero vector. Then, the \textbf{Rayleigh quotient} for
$A$ and $x$ is defined to be the real number%
\[
R\left(  A,x\right)  :=\dfrac{\left\langle Ax,x\right\rangle }{\left\langle
x,x\right\rangle }=\dfrac{x^{\ast}Ax}{x^{\ast}x}=\dfrac{x^{\ast}Ax}{\left\vert
\left\vert x\right\vert \right\vert ^{2}}=y^{\ast}Ay,
\]
where $y=\dfrac{x}{\left\vert \left\vert x\right\vert \right\vert }$.
\end{definition}

We need to show:

\begin{theorem}
[Courant--Fisher theorem]Let $A\in\mathbb{C}^{n\times n}$ be a Hermitian
matrix. Let $\lambda_{1},\lambda_{2},\ldots,\lambda_{n}$ be the eigenvalues of
$A$, with $\lambda_{1}\leq\lambda_{2}\leq\cdots\leq\lambda_{n}$. Then, for
each $k\in\left[  n\right]  $, we have%
\[
\lambda_{k}=\min\limits_{\substack{S\subseteq\mathbb{C}^{n}\text{ is a
subspace;}\\\dim S=k}}\ \ \max\limits_{\substack{x\in S;\\x\neq0}}\ \ R\left(
A,x\right)
\]
and%
\[
\lambda_{k}=\max\limits_{\substack{S\subseteq\mathbb{C}^{n}\text{ is a
subspace;}\\\dim S=n-k+1}}\ \ \min\limits_{\substack{x\in S;\\x\neq
0}}\ \ R\left(  A,x\right)  .
\]

\end{theorem}

To prove this theorem, we will use some elementary facts about subspaces of
finite-dimensional vector spaces. We begin by recalling the following definition:

\begin{definition}
Let $S_{1}$ and $S_{2}$ be two subspaces of a vector space $V$. Then,
\[
S_{1}+S_{2}:=\left\{  s_{1}+s_{2}\ \mid\ s_{1}\in S_{1}\text{ and }s_{2}\in
S_{2}\right\}  .
\]
This is again a subspace of $V$. (This is the smallest subspace of $V$ that
contains both $S_{1}$ and $S_{2}$ as subspaces.)
\end{definition}

\begin{proposition}
Let $\mathbb{F}$ be a field. Let $V$ be a finite-dimensional $\mathbb{F}%
$-vector space. Let $S_{1}$ and $S_{2}$ be two subspaces of $V$. Then,%
\[
\dim\left(  S_{1}\cap S_{2}\right)  +\dim\left(  S_{1}+S_{2}\right)  =\dim
S_{1}+\dim S_{2}.
\]

\end{proposition}

\begin{proof}
Pick any basis $\left(  x_{1},x_{2},\ldots,x_{k}\right)  $ of the vector space
$S_{1}\cap S_{2}$.

Then, $\left(  x_{1},x_{2},\ldots,x_{k}\right)  $ is a linearly independent
list of vectors in $S_{1}$. Thus, we can extend it to a basis of $S_{1}$ by
inserting some new vectors $y_{1},y_{2},\ldots,y_{p}$. Thus,%
\[
\left(  x_{1},x_{2},\ldots,x_{k},y_{1},y_{2},\ldots,y_{p}\right)  \text{ is a
basis of }S_{1}\text{.}%
\]


On the other hand, $\left(  x_{1},x_{2},\ldots,x_{k}\right)  $ is a linearly
independent list of vectors in $S_{2}$. Thus, we can extend it to a basis of
$S_{2}$ by inserting some new vectors $z_{1},z_{2},\ldots,z_{q}$. Thus,%
\[
\left(  x_{1},x_{2},\ldots,x_{k},z_{1},z_{2},\ldots,z_{q}\right)  \text{ is a
basis of }S_{2}\text{.}%
\]


The above three bases yield $\dim\left(  S_{1}\cap S_{2}\right)  =k$ and $\dim
S_{1}=k+p$ and $\dim S_{2}=k+q$.

Now, we claim that
\[
\mathbf{w}:=\left(  x_{1},x_{2},\ldots,x_{k},y_{1},y_{2},\ldots,y_{p}%
,z_{1},z_{2},\ldots,z_{q}\right)  \text{ is a basis of }S_{1}+S_{2}.
\]
Once this is proved, we will conclude that $\dim\left(  S_{1}+S_{2}\right)
=k+p+q$, and the proposition will follow by a simple computation ($k+\left(
k+p+q\right)  =\left(  k+p\right)  +\left(  k+q\right)  $).

So let us prove our claim. To prove that $\mathbf{w}$ is a basis of
$S_{1}+S_{2}$, we need to check the following two statements:

\begin{enumerate}
\item The list $\mathbf{w}$ is linearly independent.

\item The list $\mathbf{w}$ spans $S_{1}+S_{2}$.
\end{enumerate}

Proving statement 2 is easy: Any element of $S_{1}+S_{2}$ is an element of
$S_{1}$ plus an element of $S_{2}$, and thus can be written as
\begin{align*}
& \left(  \text{a linear combination of }x_{1},x_{2},\ldots,x_{k},y_{1}%
,y_{2},\ldots,y_{p}\right)  \\
& \ \ \ \ \ \ \ \ \ \ +\left(  \text{a linear combination of }x_{1}%
,x_{2},\ldots,x_{k},z_{1},z_{2},\ldots,z_{q}\right)  \\
& =\lambda_{1}x_{1}+\lambda_{2}x_{2}+\cdots+\lambda_{k}x_{k}+\alpha_{1}%
y_{1}+\alpha_{2}y_{2}+\cdots+\alpha_{p}y_{p}\\
& \ \ \ \ \ \ \ \ \ \ +\mu_{1}x_{1}+\mu_{2}x_{2}+\cdots+\mu_{k}x_{k}+\beta
_{1}z_{1}+\beta_{2}z_{2}+\cdots+\beta_{q}z_{q}\\
& =\left(  \lambda_{1}+\mu_{1}\right)  x_{1}+\left(  \lambda_{2}+\mu
_{2}\right)  x_{2}+\cdots+\left(  \lambda_{k}+\mu_{k}\right)  x_{k}\\
& \ \ \ \ \ \ \ \ \ \ +\alpha_{1}y_{1}+\alpha_{2}y_{2}+\cdots+\alpha_{p}%
y_{p}+\beta_{1}z_{1}+\beta_{2}z_{2}+\cdots+\beta_{q}z_{q}\\
& =\left(  \text{a linear combination of }x_{1},x_{2},\ldots,x_{k},y_{1}%
,y_{2},\ldots,y_{p},z_{1},z_{2},\ldots,z_{q}\right)  ;
\end{align*}
thus it belongs to the span of $\mathbf{w}$.

Let us now prove statement 1. We need to show that $\mathbf{w}$ is linearly
independent. So let us assume that%
\[
\lambda_{1}x_{1}+\lambda_{2}x_{2}+\cdots+\lambda_{k}x_{k}+\alpha_{1}%
y_{1}+\alpha_{2}y_{2}+\cdots+\alpha_{p}y_{p}+\beta_{1}z_{1}+\beta_{2}%
z_{2}+\cdots+\beta_{q}z_{q}=0
\]
for some coefficients $\lambda_{m},\alpha_{i},\beta_{j}$ that are not all
equal to $0$. We want a contradiction.

Let
\[
v:=\lambda_{1}x_{1}+\lambda_{2}x_{2}+\cdots+\lambda_{k}x_{k}+\alpha_{1}%
y_{1}+\alpha_{2}y_{2}+\cdots+\alpha_{p}y_{p}.
\]
Then,%
\begin{align*}
v  & =-\left(  \beta_{1}z_{1}+\beta_{2}z_{2}+\cdots+\beta_{q}z_{q}\right)
\ \ \ \ \ \ \ \ \ \ \left(  \text{by the above equation}\right)  \\
& \in S_{2}\ \ \ \ \ \ \ \ \ \ \left(  \text{since the }z_{j}\text{'s lie in
}S_{2}\right)  .
\end{align*}
On the other hand, the definition of $v$ yields $v\in S_{1}$ (since the
$x_{m}$'s and the $y_{i}$'s lie in $S_{1}$). Thus, $v$ lies in both $S_{1}$
and $S_{2}$. This entails that $v\in S_{1}\cap S_{2}$. Since $\left(
x_{1},x_{2},\ldots,x_{k}\right)  $ is a basis of $S_{1}\cap S_{2}$, this
entails that%
\[
v=\xi_{1}x_{1}+\xi_{2}x_{2}+\cdots+\xi_{k}x_{k}\ \ \ \ \ \ \ \ \ \ \text{for
some }\xi_{1},\xi_{2},\ldots,\xi_{k}\in\mathbb{F}.
\]
Comparing this with%
\[
v=-\left(  \beta_{1}z_{1}+\beta_{2}z_{2}+\cdots+\beta_{q}z_{q}\right)  ,
\]
we obtain%
\[
\xi_{1}x_{1}+\xi_{2}x_{2}+\cdots+\xi_{k}x_{k}=-\left(  \beta_{1}z_{1}%
+\beta_{2}z_{2}+\cdots+\beta_{q}z_{q}\right)  .
\]
In other words,%
\[
\xi_{1}x_{1}+\xi_{2}x_{2}+\cdots+\xi_{k}x_{k}+\beta_{1}z_{1}+\beta_{2}%
z_{2}+\cdots+\beta_{q}z_{q}=0.
\]
Since the list $\left(  x_{1},x_{2},\ldots,x_{k},z_{1},z_{2},\ldots
,z_{q}\right)  $ is linearly independent (being a basis of $S_{2}$), this
entails that all coefficients $\xi_{m}$ and $\beta_{j}$ are $0$. Thus, $v=0$
(since $v=\xi_{1}x_{1}+\xi_{2}x_{2}+\cdots+\xi_{k}x_{k}$). Now, recalling that%
\[
v=\lambda_{1}x_{1}+\lambda_{2}x_{2}+\cdots+\lambda_{k}x_{k}+\alpha_{1}%
y_{1}+\alpha_{2}y_{2}+\cdots+\alpha_{p}y_{p},
\]
we obtain%
\[
\lambda_{1}x_{1}+\lambda_{2}x_{2}+\cdots+\lambda_{k}x_{k}+\alpha_{1}%
y_{1}+\alpha_{2}y_{2}+\cdots+\alpha_{p}y_{p}=0.
\]
Since the list $\left(  x_{1},x_{2},\ldots,x_{k},y_{1},y_{2},\ldots
,y_{p}\right)  $ is linearly independent (being a basis of $S_{1}$), this
entails that all coefficients $\lambda_{m}$ and $\alpha_{i}$ are $0$.

Now we know that all $\lambda_{m}$ and $\alpha_{i}$ and $\beta_{j}$ are $0$,
which contradicts our assumption that some of them are nonzero. This completes
the proof of Statement 1.

As we said, we now conclude that $\dim\left(  S_{1}+S_{2}\right)  =k+p+q$, so
that%
\begin{align*}
\underbrace{\dim\left(  S_{1}\cap S_{2}\right)  }_{=k}+\underbrace{\dim\left(
S_{1}+S_{2}\right)  }_{=k+p+q}  & =k+\left(  k+p+q\right)  \\
& =\underbrace{\left(  k+p\right)  }_{=\dim S_{1}}+\underbrace{\left(
k+q\right)  }_{=\dim S_{2}}\\
& =\dim S_{1}+\dim S_{2}.
\end{align*}

\end{proof}

\begin{remark}
A well-known fact in elementary set theory says that if $A_{1}$ and $A_{2}$
are two finite sets, then%
\[
\left\vert A_{1}\cap A_{2}\right\vert +\left\vert A_{1}\cup A_{2}\right\vert
=\left\vert A_{1}\right\vert +\left\vert A_{2}\right\vert .
\]
The above theorem is an analogue of this fact for vector spaces (noticing that
the sum $S_{1}+S_{2}$ is a vector-space analogue of the union).

Note, however, that the \textquotedblleft next level\textquotedblright\ of the
above formula has no vector space analogue. We do have%
\begin{align*}
& \left\vert A_{1}\cup A_{2}\cup A_{3}\right\vert +\left\vert A_{1}\cap
A_{2}\right\vert +\left\vert A_{1}\cap A_{3}\right\vert +\left\vert A_{2}\cap
A_{3}\right\vert \\
& =\left\vert A_{1}\right\vert +\left\vert A_{2}\right\vert +\left\vert
A_{3}\right\vert +\left\vert A_{1}\cap A_{2}\cap A_{3}\right\vert
\end{align*}
for any three finite sets $A_{1},A_{2},A_{3}$, but no such relation holds for
three subspaces of a vector space.
\end{remark}

\begin{corollary}
Let $\mathbb{F}$ be a field, and let $n\in\mathbb{N}$. Let $V$ be an
$n$-dimensional $\mathbb{F}$-vector space. Let $S_{1},S_{2},\ldots,S_{k}$ be
subspaces of $V$ (with $k\geq1$). Let
\[
\delta:=\dim\left(  S_{1}\right)  +\dim\left(  S_{2}\right)  +\cdots
+\dim\left(  S_{k}\right)  -\left(  k-1\right)  n.
\]


\textbf{(a)} Then, $\dim\left(  S_{1}\cap S_{2}\cap\cdots\cap S_{k}\right)
\geq\delta$.

\textbf{(b)} If $\mathbb{F}=\mathbb{C}$ and $V=\mathbb{C}^{n}$ and $\delta>0$,
then there exists a vector $x\in S_{1}\cap S_{2}\cap\cdots\cap S_{k}$ with
$\left\vert \left\vert x\right\vert \right\vert =1$.
\end{corollary}

\begin{proof}
\textbf{(a)} We induct on $k$. The \textit{base case} ($k=1$) is obvious
(since $\dim\left(  S_{1}\cap S_{2}\cap\cdots\cap S_{k}\right)  =\dim\left(
S_{1}\right)  =\delta$ in this case).

\textit{Induction step:} Suppose the statement holds for some $k$. Now
consider $k+1$ subspaces $S_{1},S_{2},\ldots,S_{k+1}$ of $V$, and let%
\[
\delta_{k+1}:=\dim\left(  S_{1}\right)  +\dim\left(  S_{2}\right)
+\cdots+\dim\left(  S_{k+1}\right)  -kn.
\]
We want to prove that $\dim\left(  S_{1}\cap S_{2}\cap\cdots\cap S_{k}\cap
S_{k+1}\right)  \geq\delta_{k+1}$.

Then,%
\begin{align*}
& \dim\left(  S_{1}\cap S_{2}\cap\cdots\cap S_{k}\cap S_{k+1}\right)  \\
& =\dim\left(  S_{1}\cap S_{2}\cap\cdots\cap S_{k-1}\cap\left(  S_{k}\cap
S_{k+1}\right)  \right)  .
\end{align*}
Now, set
\[
\delta_{k}:=\dim\left(  S_{1}\right)  +\dim\left(  S_{2}\right)  +\cdots
+\dim\left(  S_{k-1}\right)  +\dim\left(  S_{k}\cap S_{k+1}\right)  -\left(
k-1\right)  n.
\]
By the induction hypothesis, we have%
\[
\dim\left(  S_{1}\cap S_{2}\cap\cdots\cap S_{k-1}\cap\left(  S_{k}\cap
S_{k+1}\right)  \right)  \geq\delta_{k}.
\]
What remains is to show that $\delta_{k}\geq\delta_{k+1}$. Equivalently, we
need to show that%
\[
\dim\left(  S_{k}\cap S_{k+1}\right)  -\left(  k-1\right)  n\geq\dim\left(
S_{k}\right)  +\dim\left(  S_{k+1}\right)  -kn.
\]
In other words, we need to show that%
\[
\dim\left(  S_{k}\cap S_{k+1}\right)  +n\geq\dim\left(  S_{k}\right)
+\dim\left(  S_{k+1}\right)  .
\]
However, $S_{k}+S_{k+1}$ is a subspace of $V$, so its dimension is
$\dim\left(  S_{k}+S_{k+1}\right)  \leq\dim V=n$. Therefore,
\[
\dim\left(  S_{k}\cap S_{k+1}\right)  +\underbrace{n}_{\geq\dim\left(
S_{k}+S_{k+1}\right)  }\geq\dim\left(  S_{k}\cap S_{k+1}\right)  +\dim\left(
S_{k}+S_{k+1}\right)  =\dim\left(  S_{k}\right)  +\dim\left(  S_{k+1}\right)
\]
(by the previous proposition). So the induction step is complete, and part
\textbf{(a)} of the corollary is proved.

\textbf{(b)} Assume that $\mathbb{F}=\mathbb{C}$ and $V=\mathbb{C}^{n}$ and
$\delta>0$. Then, part \textbf{(a)} yields%
\[
\dim\left(  S_{1}\cap S_{2}\cap\cdots\cap S_{k}\right)  \geq\delta>0.
\]
Thus, the subspace $S_{1}\cap S_{2}\cap\cdots\cap S_{k}$ is not just $\left\{
0\right\}  $. Therefore, it contains a nonzero vector. Scaling this vector by
the reciprocal of its length, we obtain a vector of length $1$. This proves
part \textbf{(b)}.
\end{proof}

Now, we get to the proof of the Courant--Fisher theorem:

\begin{proof}
[Proof of the Courant--Fisher theorem.] The spectral theorem says that
$A=UDU^{\ast}$ for some unitary $U$ and some real diagonal matrix $D$.
Consider these $U$ and $D$. The columns of $U$ form an orthonormal basis of
$\mathbb{C}^{n}$ (since $U$ is unitary); let $\left(  u_{1},u_{2},\ldots
,u_{n}\right)  $ be this basis. Then, $u_{1},u_{2},\ldots,u_{n}$ are
eigenvectors of $A$. We WLOG assume that the corresponding eigenvalues are
$\lambda_{1},\lambda_{2},\ldots,\lambda_{n}$ (otherwise, permute the diagonal
entries of $D$ and correspondingly permute the columns of $U$).

Let $k\in\left[  n\right]  $.

Let $S$ be a vector subspace of $\mathbb{C}^{n}$ with $\dim S=k$. Let
$S^{\prime}=\operatorname*{span}\left(  u_{k},u_{k+1},\ldots,u_{n}\right)  $.
Then, by the proposition, we have%
\[
\dim\left(  S\cap S^{\prime}\right)  +\dim\left(  S+S^{\prime}\right)
=\underbrace{\dim S}_{=k}+\underbrace{\dim S^{\prime}}_{=n-k+1}=n+1>n\geq
\dim\left(  S+S^{\prime}\right)
\]
(since $S+S^{\prime}$ is a subspace of $\mathbb{C}^{n}$). Subtracting
$\dim\left(  S+S^{\prime}\right)  $ from this inequality, we obtain
$\dim\left(  S\cap S^{\prime}\right)  >0$. Thus, $S\cap S^{\prime}$ contains a
nonzero vector. Thus, $\sup\limits_{\substack{x\in S\cap S^{\prime};\\x\neq
0}}\ \ R\left(  A,x\right)  $ and $\inf\limits_{\substack{x\in S\cap
S^{\prime};\\x\neq0}}\ \ R\left(  A,x\right)  $ are well-defined.

Now,%
\[
\sup\limits_{\substack{x\in S;\\x\neq0}}\ \ R\left(  A,x\right)  \geq
\sup\limits_{\substack{x\in S\cap S^{\prime};\\x\neq0}}\ \ R\left(
A,x\right)  \geq\inf\limits_{\substack{x\in S\cap S^{\prime};\\x\neq
0}}\ \ R\left(  A,x\right)  \geq\inf\limits_{\substack{x\in S^{\prime}%
;\\x\neq0}}\ \ R\left(  A,x\right)  .
\]
However, I claim that $\inf\limits_{\substack{x\in S^{\prime};\\x\neq
0}}\ \ R\left(  A,x\right)  =\lambda_{k}$. Indeed, any $x\in S^{\prime}$ is a
linear combination $\alpha_{k}u_{k}+\alpha_{k+1}u_{k+1}+\cdots+\alpha_{n}%
u_{n}$ and therefore satisfies%
\begin{align*}
\left\langle Ax,x\right\rangle  & =\left\langle A\left(  \alpha_{k}%
u_{k}+\alpha_{k+1}u_{k+1}+\cdots+\alpha_{n}u_{n}\right)  ,\ \alpha_{k}%
u_{k}+\alpha_{k+1}u_{k+1}+\cdots+\alpha_{n}u_{n}\right\rangle \\
& =\left\langle \alpha_{k}Au_{k}+\alpha_{k+1}Au_{k+1}+\cdots+\alpha_{n}%
Au_{n},\ \alpha_{k}u_{k}+\alpha_{k+1}u_{k+1}+\cdots+\alpha_{n}u_{n}%
\right\rangle \\
& =\left\langle \alpha_{k}\lambda_{k}u_{k}+\alpha_{k+1}\lambda_{k+1}%
u_{k+1}+\cdots+\alpha_{n}\lambda_{n}u_{n},\ \alpha_{k}u_{k}+\alpha
_{k+1}u_{k+1}+\cdots+\alpha_{n}u_{n}\right\rangle \\
& =\sum_{i=k}^{n}\underbrace{\alpha_{i}\overline{\alpha_{i}}}_{=\left\vert
\alpha_{i}\right\vert ^{2}}\lambda_{i}\ \ \ \ \ \ \ \ \ \ \left(  \text{since
}u_{k},u_{k+1},\ldots,u_{n}\text{ are orthonormal}\right)  \\
& =\sum_{i=k}^{n}\left\vert \alpha_{i}\right\vert ^{2}\underbrace{\lambda_{i}%
}_{\substack{\geq\lambda_{k}\\\text{(since }\lambda_{1}\leq\lambda_{2}%
\leq\cdots\leq\lambda_{n}\text{)}}}\geq\lambda_{k}\underbrace{\sum_{i=k}%
^{n}\left\vert \alpha_{i}\right\vert ^{2}}_{=\left\langle x,x\right\rangle
}=\lambda_{k}\left\langle x,x\right\rangle
\end{align*}
and thus $R\left(  A,x\right)  =\dfrac{\left\langle Ax,x\right\rangle
}{\left\langle x,x\right\rangle }\geq\lambda_{k}$. So, altogether, we find%
\[
\sup\limits_{\substack{x\in S;\\x\neq0}}\ \ R\left(  A,x\right)  \geq
\lambda_{k}.
\]
Furthermore, this supremum is a maximum, because
\begin{align*}
\sup\limits_{\substack{x\in S;\\x\neq0}}\ \ R\left(  A,x\right)    &
=\sup\limits_{\substack{y\in S;\\\left\vert \left\vert y\right\vert
\right\vert =1}}\ \ R\left(  A,y\right)  \ \ \ \ \ \ \ \ \ \ \left(
\text{since }R\left(  A,x\right)  =R\left(  A,y\right)  \text{ where }%
y=\dfrac{x}{\left\vert \left\vert x\right\vert \right\vert }\right)  \\
& =\max\limits_{\substack{y\in S;\\\left\vert \left\vert y\right\vert
\right\vert =1}}\ \ R\left(  A,y\right)  \ \ \ \ \ \ \ \ \ \ \left(
\begin{array}
[c]{c}%
\text{since the set of all }y\in S\text{ satisfying }\left\vert \left\vert
y\right\vert \right\vert =1\\
\text{is compact, and since a continuous function}\\
\text{on a compact set always has a maximum}%
\end{array}
\right)  \\
& =\max\limits_{\substack{x\in S;\\x\neq0}}\ \ R\left(  A,x\right)  .
\end{align*}
So we conclude%
\[
\max\limits_{\substack{x\in S;\\x\neq0}}\ \ R\left(  A,x\right)
=\sup\limits_{\substack{x\in S;\\x\neq0}}\ \ R\left(  A,x\right)  \geq
\lambda_{k}.
\]


Forget that we fixed $S$. We thus have shown that if $S$ is any $k$%
-dimensional subspace of $\mathbb{C}^{n}$, then $\max\limits_{\substack{x\in
S;\\x\neq0}}\ \ R\left(  A,x\right)  $ exists and satisfies
\[
\max\limits_{\substack{x\in S;\\x\neq0}}\ \ R\left(  A,x\right)  \geq
\lambda_{k}.
\]
However, by choosing $S$ appropriately, we can achieve equality here; indeed,
we have to choose $S=\operatorname*{span}\left(  u_{1},u_{2},\ldots
,u_{k}\right)  $ for this. (Why? Because each $x\in\operatorname*{span}\left(
u_{1},u_{2},\ldots,u_{k}\right)  $ can easily be seen to satisfy $\left\langle
Ax,x\right\rangle \leq\lambda_{k}\left\langle x,x\right\rangle $ by a similar
argument to the one we used above.)

So $\max\limits_{\substack{x\in S;\\x\neq0}}\ \ R\left(  A,x\right)  $ is
$\geq\lambda_{k}$ for each $S$, but is $=\lambda_{k}$ for a certain $S$.
Therefore, $\lambda_{k}$ is the smallest possible value of $\max
\limits_{\substack{x\in S;\\x\neq0}}\ \ R\left(  A,x\right)  $. In other
words,%
\[
\lambda_{k}=\min\limits_{\substack{S\subseteq\mathbb{C}^{n}\text{ is a
subspace;}\\\dim S=k}}\ \ \max\limits_{\substack{x\in S;\\x\neq0}}\ \ R\left(
A,x\right)  .
\]


It remains to prove the other part of the theorem -- i.e., the equality%
\[
\lambda_{k}=\max\limits_{\substack{S\subseteq\mathbb{C}^{n}\text{ is a
subspace;}\\\dim S=n-k+1}}\ \ \min\limits_{\substack{x\in S;\\x\neq
0}}\ \ R\left(  A,x\right)  .
\]
One way to prove this is by arguing similarly to the above proof.
Alternatively, we can simply apply the already proved equality%
\[
\lambda_{k}=\min\limits_{\substack{S\subseteq\mathbb{C}^{n}\text{ is a
subspace;}\\\dim S=k}}\ \ \max\limits_{\substack{x\in S;\\x\neq0}}\ \ R\left(
A,x\right)
\]
to $-A$ instead of $A$, which is a Hermitian matrix with eigenvalues%
\[
-\lambda_{n}\leq-\lambda_{n-1}\leq\cdots\leq-\lambda_{1}.
\]
Keep in mind that $-\lambda_{k}$ is not the $k$-th smallest eigenvalue of
$-A$, but it is the $k$-th largest eigenvalue of $-A$, and thus the $\left(
n-k+1\right)  $-st smallest eigenvalue of $-A$. Thus, we have to apply the
equality to $-A$ and $n-k+1$ instead of $A$ and $k$. Taking negatives turns
minima into maxima and vice versa.
\end{proof}

The Courant--Fisher theorem can be used to connect the eigenvalues of $A+B$
with the eigenvalues of $A$ and $B$.

\begin{theorem}
[Weyl's theorem]Let $A$ and $B$ be two Hermitian matrices in $\mathbb{C}%
^{n\times n}$. Let $i\in\left[  n\right]  $ and $j\in\left\{  0,1,\ldots
,n-i\right\}  $.

\textbf{(a)} Then,%
\[
\lambda_{i}\left(  A+B\right)  \leq\lambda_{i+j}\left(  A\right)
+\lambda_{n-j}\left(  B\right)  .
\]
Here, $\lambda_{k}\left(  C\right)  $ means the $k$-th smallest eigenvalue of
a Hermitian matrix $C$.

Moreover, this inequality becomes an equality if and only if there exists a
nonzero vector $x\in\mathbb{C}^{n}$ satisfying
\[
Ax=\lambda_{i+j}\left(  A\right)  x,\ \ \ \ \ \ \ \ \ \ Bx=\lambda
_{n-j}\left(  B\right)  x,\ \ \ \ \ \ \ \ \ \ \left(  A+B\right)
x=\lambda_{i}\left(  A+B\right)  x
\]
(at the same time).

\textbf{(b)} Furthermore,%
\[
\lambda_{i-k+1}\left(  A\right)  +\lambda_{k}\left(  B\right)  \leq\lambda
_{i}\left(  A+B\right)  \ \ \ \ \ \ \ \ \ \ \text{for any }k\in\left[
i\right]  .
\]

\end{theorem}

\begin{proof}
Let $\left(  x_{1},x_{2},\ldots,x_{n}\right)  $, $\left(  y_{1},y_{2}%
,\ldots,y_{n}\right)  $ and $\left(  z_{1},z_{2},\ldots,z_{n}\right)  $ be
three orthonormal bases of $\mathbb{C}^{n}$ with%
\[
Ax_{i}=\lambda_{i}\left(  A\right)  x_{i},\ \ \ \ \ \ \ \ \ \ By_{i}%
=\lambda_{i}\left(  B\right)  y_{i},\ \ \ \ \ \ \ \ \ \ \left(  A+B\right)
z_{i}=\lambda_{i}\left(  A+B\right)  z_{i}%
\]
for all $i\in\left[  n\right]  $. (As above, we can find such bases by using
the spectral decompositions of $A$, $B$ and $A+B$.)

Let
\begin{align*}
S_{1}  & =\operatorname*{span}\left(  x_{1},x_{2},\ldots,x_{i+j}\right)  ;\\
S_{2}  & =\operatorname*{span}\left(  y_{1},y_{2},\ldots,y_{n-j}\right)  ;\\
S_{3}  & =\operatorname*{span}\left(  z_{i},z_{i+1},\ldots,z_{n}\right)  .
\end{align*}
Then,%
\[
\delta:=\underbrace{\dim\left(  S_{1}\right)  }_{=i+j}+\underbrace{\dim\left(
S_{2}\right)  }_{=n-j}+\underbrace{\dim\left(  S_{3}\right)  }_{=n-i+1}%
-2n=1>0.
\]
Hence, part \textbf{(b)} of the corollary yields that there is a length-$1$
vector $x$ in $S_{1}\cap S_{2}\cap S_{3}$. This $x$ satisfies%
\begin{align*}
\lambda_{i}\left(  A+B\right)    & \leq\left\langle \left(  A+B\right)
x,\ x\right\rangle =\left\langle Ax+Bx,\ x\right\rangle
=\underbrace{\left\langle Ax,\ x\right\rangle }_{\leq\lambda_{i+j}\left(
A\right)  }+\underbrace{\left\langle Bx,\ x\right\rangle }_{\leq\lambda
_{n-j}\left(  B\right)  }\\
& \leq\lambda_{i+j}\left(  A\right)  +\lambda_{n-j}\left(  B\right)  .
\end{align*}
This proves \textbf{(a)}.

We leave \textbf{(b)} and \textbf{(c)} to the reader.
\end{proof}


\end{document}