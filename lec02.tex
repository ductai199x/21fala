\documentclass[numbers=enddot,12pt,final,onecolumn,notitlepage]{scrartcl}%
\usepackage[headsepline,footsepline,manualmark]{scrlayer-scrpage}
\usepackage[all,cmtip]{xy}
\usepackage{amssymb}
\usepackage{amsmath}
\usepackage{amsthm}
\usepackage{framed}
\usepackage{comment}
\usepackage{color}
\usepackage{hyperref}
\usepackage[sc]{mathpazo}
\usepackage[T1]{fontenc}
\usepackage{tikz}
\usepackage{needspace}
\usepackage{tabls}
\usepackage{wasysym}
\usepackage{easytable}
\usepackage{pythonhighlight}
%TCIDATA{OutputFilter=latex2.dll}
%TCIDATA{Version=5.50.0.2960}
%TCIDATA{LastRevised=Wednesday, September 22, 2021 11:50:51}
%TCIDATA{SuppressPackageManagement}
%TCIDATA{<META NAME="GraphicsSave" CONTENT="32">}
%TCIDATA{<META NAME="SaveForMode" CONTENT="1">}
%TCIDATA{BibliographyScheme=Manual}
%TCIDATA{Language=American English}
%BeginMSIPreambleData
\providecommand{\U}[1]{\protect\rule{.1in}{.1in}}
%EndMSIPreambleData
\usetikzlibrary{arrows.meta}
\usetikzlibrary{chains}
\newcounter{exer}
\newcounter{exera}
\numberwithin{exer}{subsection}
\theoremstyle{definition}
\newtheorem{theo}{Theorem}[subsection]
\newenvironment{theorem}[1][]
{\begin{theo}[#1]\begin{leftbar}}
{\end{leftbar}\end{theo}}
\newtheorem{lem}[theo]{Lemma}
\newenvironment{lemma}[1][]
{\begin{lem}[#1]\begin{leftbar}}
{\end{leftbar}\end{lem}}
\newtheorem{prop}[theo]{Proposition}
\newenvironment{proposition}[1][]
{\begin{prop}[#1]\begin{leftbar}}
{\end{leftbar}\end{prop}}
\newtheorem{defi}[theo]{Definition}
\newenvironment{definition}[1][]
{\begin{defi}[#1]\begin{leftbar}}
{\end{leftbar}\end{defi}}
\newtheorem{remk}[theo]{Remark}
\newenvironment{remark}[1][]
{\begin{remk}[#1]\begin{leftbar}}
{\end{leftbar}\end{remk}}
\newtheorem{coro}[theo]{Corollary}
\newenvironment{corollary}[1][]
{\begin{coro}[#1]\begin{leftbar}}
{\end{leftbar}\end{coro}}
\newtheorem{conv}[theo]{Convention}
\newenvironment{convention}[1][]
{\begin{conv}[#1]\begin{leftbar}}
{\end{leftbar}\end{conv}}
\newtheorem{quest}[theo]{Question}
\newenvironment{question}[1][]
{\begin{quest}[#1]\begin{leftbar}}
{\end{leftbar}\end{quest}}
\newtheorem{warn}[theo]{Warning}
\newenvironment{warning}[1][]
{\begin{warn}[#1]\begin{leftbar}}
{\end{leftbar}\end{warn}}
\newtheorem{conj}[theo]{Conjecture}
\newenvironment{conjecture}[1][]
{\begin{conj}[#1]\begin{leftbar}}
{\end{leftbar}\end{conj}}
\newtheorem{exam}[theo]{Example}
\newenvironment{example}[1][]
{\begin{exam}[#1]\begin{leftbar}}
{\end{leftbar}\end{exam}}
\newtheorem{exmp}[exer]{Exercise}
\newenvironment{exercise}[1][]
{\begin{exmp}[#1]\begin{leftbar}}
{\end{leftbar}\end{exmp}}
\newenvironment{statement}{\begin{quote}}{\end{quote}}
\newenvironment{fineprint}{\medskip \begin{small}}{\end{small} \medskip}
\iffalse
\newenvironment{proof}[1][Proof]{\noindent\textbf{#1.} }{\ \rule{0.5em}{0.5em}}
\newenvironment{question}[1][Question]{\noindent\textbf{#1.} }{\ \rule{0.5em}{0.5em}}
\newenvironment{warning}[1][Warning]{\noindent\textbf{#1.} }{\ \rule{0.5em}{0.5em}}
\newenvironment{teachingnote}[1][Teaching note]{\noindent\textbf{#1.} }{\ \rule{0.5em}{0.5em}}
\fi
\let\sumnonlimits\sum
\let\prodnonlimits\prod
\let\cupnonlimits\bigcup
\let\capnonlimits\bigcap
\renewcommand{\sum}{\sumnonlimits\limits}
\renewcommand{\prod}{\prodnonlimits\limits}
\renewcommand{\bigcup}{\cupnonlimits\limits}
\renewcommand{\bigcap}{\capnonlimits\limits}
\setlength\tablinesep{3pt}
\setlength\arraylinesep{3pt}
\setlength\extrarulesep{3pt}
\voffset=0cm
\hoffset=-0.7cm
\setlength\textheight{22.5cm}
\setlength\textwidth{15.5cm}
\newcommand\arxiv[1]{\href{http://www.arxiv.org/abs/#1}{\texttt{arXiv:#1}}}
\newenvironment{verlong}{}{}
\newenvironment{vershort}{}{}
\newenvironment{noncompile}{}{}
\newenvironment{teachingnote}{}{}
\excludecomment{verlong}
\includecomment{vershort}
\excludecomment{noncompile}
\excludecomment{teachingnote}
\newcommand{\CC}{\mathbb{C}}
\newcommand{\RR}{\mathbb{R}}
\newcommand{\QQ}{\mathbb{Q}}
\newcommand{\NN}{\mathbb{N}}
\newcommand{\ZZ}{\mathbb{Z}}
\newcommand{\KK}{\mathbb{K}}
\newcommand{\id}{\operatorname{id}}
\newcommand{\lcm}{\operatorname{lcm}}
\newcommand{\rev}{\operatorname{rev}}
\newcommand{\powset}[2][]{\ifthenelse{\equal{#2}{}}{\mathcal{P}\left(#1\right)}{\mathcal{P}_{#1}\left(#2\right)}}
\newcommand{\set}[1]{\left\{ #1 \right\}}
\newcommand{\abs}[1]{\left| #1 \right|}
\newcommand{\tup}[1]{\left( #1 \right)}
\newcommand{\ive}[1]{\left[ #1 \right]}
\newcommand{\floor}[1]{\left\lfloor #1 \right\rfloor}
\newcommand{\lf}[2]{#1^{\underline{#2}}}
\newcommand{\underbrack}[2]{\underbrace{#1}_{\substack{#2}}}
\newcommand{\horrule}[1]{\rule{\linewidth}{#1}}
\newcommand{\are}{\ar@{-}}
\newcommand{\nnn}{\nonumber\\}
\newcommand{\sslash}{\mathbin{/\mkern-6mu/}}
\newcommand{\numboxed}[2]{\underbrace{\boxed{#1}}_{\text{box } #2}}
\newcommand{\ig}[2]{\includegraphics[scale=#1]{#2.png}}
\newcommand{\UNFINISHED}{\begin{center} \Huge{\textbf{Unfinished material begins here.}} \end{center} }
\iffalse
\NOEXPAND{\today}{\today}
\NOEXPAND{\sslash}{\sslash}
\NOEXPAND{\numboxed}[2]{\numboxed}
\NOEXPAND{\UNFINISHED}{\UNFINISHED}
\fi
\ihead{Math 504 notes}
\ohead{page \thepage}
\cfoot{\today}
\begin{document}

\title{Math 504: Advanced Linear Algebra}
\author{Hugo Woerdeman, with edits by Darij Grinberg\thanks{Drexel University, Korman
Center, 15 S 33rd Street, Philadelphia PA, 19104, USA}}
\date{\today\ (unfinished!)}
\maketitle
\tableofcontents

\section*{Math 504 Lecture 2}

\section{Unitary matrices ([HorJoh13, \S 2.1]) (cont'd)}

\textbf{Recall:}

\begin{itemize}
\item An $n\times k$-matrix $A$ is said to be an \textbf{isometry} if
$A^{\ast}A=I_{k}$. (The notation $I_{k}$ means the $k\times k$ identity matrix.)

\item An $n\times k$-matrix $A$ is an isometry if and only if its columns form
an orthonormal tuple of vectors.
\end{itemize}

\subsection{Unitary matrices}

\begin{definition}
A matrix $U\in\mathbb{C}^{n\times k}$ is said to be \textbf{unitary} if both
$U$ and $U^{\ast}$ are isometries.
\end{definition}

\begin{example}
\textbf{(a)} The matrix $A=\dfrac{1}{\sqrt{2}}\left(
\begin{array}
[c]{cc}%
1 & 1\\
1 & -1
\end{array}
\right)  $ is unitary. Indeed, it is easy to see that $A$ is an isometry, but
$A^{\ast}$ is therefore also an isometry, since $A^{\ast}=A$. Thus, $A$ is unitary.

\textbf{(b)} A $1\times1$-matrix $\left(
\begin{array}
[c]{c}%
\lambda
\end{array}
\right)  $ is unitary if and only if $\left\vert \lambda\right\vert =1$.

\textbf{(c)} For any $n\in\mathbb{N}$, the identity matrix $I_{n}$ is an
isometry and thus unitary.

\textbf{(d)} Let $n\in\mathbb{N}$. Let $\sigma$ be a permutation of $\left[
n\right]  =\left\{  1,2,\ldots,n\right\}  $. That is, $\sigma$ is a bijective
map from $\left[  n\right]  $ to $\left[  n\right]  $.

Let $P_{\sigma}$ be the \textbf{permutation matrix} of $\sigma$; this is the
$n\times n$-matrix whose $\left(  \sigma\left(  j\right)  ,j\right)  $-th
entry is $1$ for all $j\in\left[  n\right]  $, and whose all other entries are
$0$.

For instance, if $n=3$ and $\sigma=\left(
\begin{array}
[c]{ccc}%
1 & 2 & 3\\
2 & 3 & 1
\end{array}
\right)  $, then%
\[
P_{\sigma}=\left(
\begin{array}
[c]{ccc}%
0 & 0 & 1\\
1 & 0 & 0\\
0 & 1 & 0
\end{array}
\right)  .
\]


The permutation matrix $P_{\sigma}$ is always unitary (for any $n$ and any
$\sigma$). Indeed, the inverse of $P_{\sigma}$ is $P_{\sigma^{-1}}$, but the
conjugate transpose of $P_{\sigma}$ is also $P_{\sigma^{-1}}$. Thus,
$P_{\sigma}^{\ast}=P_{\sigma}^{-1}$, so $P_{\sigma}^{\ast}P_{\sigma}=I_{n}$.
This shows that $P_{\sigma}$ is an isometry. Similarly, $P_{\sigma}P_{\sigma
}^{\ast}=I_{n}$, so that $P_{\sigma}^{\ast}$ is an isometry, and thus
$P_{\sigma}$ is unitary\textbf{.}

\textbf{(e)} A diagonal matrix $\operatorname*{diag}\left(  \lambda
_{1},\lambda_{2},\ldots,\lambda_{n}\right)  =\left(
\begin{array}
[c]{cccc}%
\lambda_{1} & 0 & \cdots & 0\\
0 & \lambda_{2} & \cdots & 0\\
\vdots & \vdots & \ddots & \vdots\\
0 & 0 & \cdots & \lambda_{n}%
\end{array}
\right)  $ is unitary if and only if%
\[
\left\vert \lambda_{1}\right\vert =\left\vert \lambda_{2}\right\vert
=\cdots=\left\vert \lambda_{n}\right\vert =1.
\]
Indeed,
\begin{align*}
& \left(
\begin{array}
[c]{cccc}%
\lambda_{1} & 0 & \cdots & 0\\
0 & \lambda_{2} & \cdots & 0\\
\vdots & \vdots & \ddots & \vdots\\
0 & 0 & \cdots & \lambda_{n}%
\end{array}
\right)  ^{\ast}\left(
\begin{array}
[c]{cccc}%
\lambda_{1} & 0 & \cdots & 0\\
0 & \lambda_{2} & \cdots & 0\\
\vdots & \vdots & \ddots & \vdots\\
0 & 0 & \cdots & \lambda_{n}%
\end{array}
\right)  \\
& =\left(
\begin{array}
[c]{cccc}%
\overline{\lambda_{1}} & 0 & \cdots & 0\\
0 & \overline{\lambda_{2}} & \cdots & 0\\
\vdots & \vdots & \ddots & \vdots\\
0 & 0 & \cdots & \overline{\lambda_{n}}%
\end{array}
\right)  \left(
\begin{array}
[c]{cccc}%
\lambda_{1} & 0 & \cdots & 0\\
0 & \lambda_{2} & \cdots & 0\\
\vdots & \vdots & \ddots & \vdots\\
0 & 0 & \cdots & \lambda_{n}%
\end{array}
\right)  \\
& =\left(
\begin{array}
[c]{cccc}%
\overline{\lambda_{1}}\lambda_{1} & 0 & \cdots & 0\\
0 & \overline{\lambda_{2}}\lambda_{2} & \cdots & 0\\
\vdots & \vdots & \ddots & \vdots\\
0 & 0 & \cdots & \overline{\lambda_{n}}\lambda_{n}%
\end{array}
\right)  =\left(
\begin{array}
[c]{cccc}%
\left\vert \lambda_{1}\right\vert ^{2} & 0 & \cdots & 0\\
0 & \left\vert \lambda_{2}\right\vert ^{2} & \cdots & 0\\
\vdots & \vdots & \ddots & \vdots\\
0 & 0 & \cdots & \left\vert \lambda_{n}\right\vert ^{2}%
\end{array}
\right)  .
\end{align*}

\end{example}

\begin{theorem}
Let $U\in\mathbb{C}^{n\times k}$ be a matrix. The following six statements are equivalent:

\begin{itemize}
\item $\mathcal{A}$: The matrix $U$ is unitary.

\item $\mathcal{B}$: The matrices $U$ and $U^{\ast}$ are isometries.

\item $\mathcal{C}$: We have $UU^{\ast}=I_{n}$ and $U^{\ast}U=I_{k}$.

\item $\mathcal{D}$: The matrix $U$ is square (that is, $n=k$) and invertible
and satisfies $U^{-1}=U^{\ast}$.

\item $\mathcal{E}$: The columns of $U$ form an orthonormal basis of
$\mathbb{C}^{n}$.

\item $\mathcal{F}$: The matrix $U$ is square (that is, $n=k$) and is an isometry.
\end{itemize}
\end{theorem}

\begin{proof}
$\mathcal{A}\Longleftrightarrow\mathcal{B}$ follows from the def of
\textquotedblleft unitary\textquotedblright.

$\mathcal{B}\Longleftrightarrow\mathcal{C}$ follows from the def of
\textquotedblleft isometry\textquotedblright, since $\left(  U^{\ast}\right)
^{\ast}=U$.

$\mathcal{D}\Longrightarrow\mathcal{C}$ is obvious.

$\mathcal{C}\Longrightarrow\mathcal{D}$ follows from the fact that any
invertible matrix is square.

The other implications are a bit harder:

\begin{itemize}
\item $\mathcal{D}\Longrightarrow\mathcal{E}$: Assume that $\mathcal{D}$
holds. Thus, $U$ is an isometry (since $U^{\ast}U=I_{k}$). So, from a result
from last time, we see that the columns of $U$ form an orthonormal tuple of
vectors. But they also form a basis of $\mathbb{C}^{n}$, because $U$ is
invertible. So $\mathcal{E}$ holds.

\item $\mathcal{E}\Longrightarrow\mathcal{D}$: Assume that $\mathcal{E}$
holds. Then, the columns of $U$ form an orthonormal basis, hence an
orthonormal tuple. Thus, $U$ is an isometry (by what we know from last
lecture). Furthermore, the columns of $U$ form a basis, so there are precisely
$n$ of them (since any basis of $\mathbb{C}^{n}$ has $n$ vectors). This shows
that $U$ is square. Finally, $U$ is invertible, since its columns form a
basis. So $\mathcal{D}$ holds.

\item $\mathcal{D}\Longrightarrow\mathcal{F}$: Easy.

\item $\mathcal{F}\Longrightarrow\mathcal{D}$: It is known that a square
matrix $A$ that is left invertible (i.e., there is a matrix $B$ such that
$BA=I$) is always invertible. Since $U$ is an isometry, we have $U^{\ast
}U=I_{k}$, so that $U$ is left-invertible. Thus, $U$ is invertible, and its
inverse is $U^{-1}=U^{\ast}$. So $\mathcal{D}$ holds.
\end{itemize}

This proves $\mathcal{A}\Longleftrightarrow\mathcal{B}\Longleftrightarrow
\mathcal{C}\Longleftrightarrow\mathcal{D}\Longleftrightarrow\mathcal{E}%
\Longleftrightarrow\mathcal{F}$.
\end{proof}

This theorem shows that any unitary matrix is square. In contrast, an isometry
can be a tall matrix (but not a wide matrix).

The set of all unitary $n\times n$-matrices is called the $n$\textbf{-th
unitary group}, and is denoted $\operatorname*{U}\nolimits_{n}\left(
\mathbb{C}\right)  $. It is a group under multiplication. 

If $U$ is a unitary matrix, then $\left\vert \det U\right\vert =1$ and any
eigenvalue $\lambda$ of $U$ has $\left\vert \lambda\right\vert =1$.

\subsubsection{Block matrices}

\begin{definition}
Let $A\in\mathbb{C}^{n\times p}$, $B\in\mathbb{C}^{n\times q}$, $C\in
\mathbb{C}^{m\times p}$, $D\in\mathbb{C}^{m\times q}$ be four matrices. Then,
$\left(
\begin{array}
[c]{cc}%
A & B\\
C & D
\end{array}
\right)  $ means the matrix%
\[
\left(
\begin{array}
[c]{cccccc}%
A_{1,1} & A_{1,2} & \cdots & B_{1,1} & B_{1,2} & \cdots\\
A_{2,1} & A_{2,2} & \cdots & B_{2,1} & B_{2,2} & \cdots\\
\vdots & \vdots & \ddots & \vdots & \vdots & \ddots\\
C_{1,1} & C_{1,2} & \cdots & D_{1,1} & D_{1,2} & \cdots\\
C_{2,1} & C_{2,2} & \cdots & D_{2,1} & D_{2,2} & \cdots\\
\vdots & \vdots & \ddots & \vdots & \vdots & \ddots
\end{array}
\right)  \in\mathbb{C}^{\left(  n+m\right)  \times\left(  p+q\right)  }%
\]
(where $M_{i,j}$ means the $\left(  i,j\right)  $-th entry of a matrix $M$).

This matrix is called the \textbf{block matrix} formed of $A$, $B$, $C$ and
$D$.
\end{definition}

Similar notations will be used to glue together more than $4$ matrices.

This block matrix notation is more than just a convenient notation. Indeed,
multiplication of matrices plays along with it:

\begin{proposition}
For any matrices $A,B,C,D,A^{\prime},B^{\prime},C^{\prime},D^{\prime}$, we
have%
\[
\left(
\begin{array}
[c]{cc}%
A & B\\
C & D
\end{array}
\right)  \left(
\begin{array}
[c]{cc}%
A^{\prime} & B^{\prime}\\
C^{\prime} & D^{\prime}%
\end{array}
\right)  =\left(
\begin{array}
[c]{cc}%
AA^{\prime}+BC^{\prime} & AB^{\prime}+BD^{\prime}\\
CA^{\prime}+DC^{\prime} & CB^{\prime}+DD^{\prime}%
\end{array}
\right)  ,
\]
provided that the block matrices and the products make sense.
\end{proposition}

Again, a similar rule holds for larger block matrices.

A particularly well-behaved family of block matrices are the
\textbf{block-diagonal matrices}. These are the block matrices of the form%
\[
\left(
\begin{array}
[c]{cccc}%
A\left(  1,1\right)   & 0 & \cdots & 0\\
0 & A\left(  2,2\right)   & \cdots & 0\\
\vdots & \vdots & \ddots & \vdots\\
0 & 0 & \cdots & A\left(  n,n\right)
\end{array}
\right)  ,
\]
where each $A\left(  i,i\right)  $ is a square matrix, and where the $0$'s
mean zero matrices of appropriate dimension.

These block-diagonal matrices can be muliplied \textquotedblleft diagonal
block by diagonal block\textquotedblright:%
\begin{align*}
& \left(
\begin{array}
[c]{cccc}%
A\left(  1,1\right)   & 0 & \cdots & 0\\
0 & A\left(  2,2\right)   & \cdots & 0\\
\vdots & \vdots & \ddots & \vdots\\
0 & 0 & \cdots & A\left(  n,n\right)
\end{array}
\right)  \left(
\begin{array}
[c]{cccc}%
B\left(  1,1\right)   & 0 & \cdots & 0\\
0 & B\left(  2,2\right)   & \cdots & 0\\
\vdots & \vdots & \ddots & \vdots\\
0 & 0 & \cdots & B\left(  n,n\right)
\end{array}
\right)  \\
& =\left(
\begin{array}
[c]{cccc}%
A\left(  1,1\right)  B\left(  1,1\right)   & 0 & \cdots & 0\\
0 & A\left(  2,2\right)  B\left(  2,2\right)   & \cdots & 0\\
\vdots & \vdots & \ddots & \vdots\\
0 & 0 & \cdots & A\left(  n,n\right)  B\left(  n,n\right)
\end{array}
\right)  ,
\end{align*}
provided that
\[
\text{size of }A\left(  i,i\right)  =\text{size of }B\left(  i,i\right)
\text{ for each }i.
\]


\textbf{Example:} $\left(
\begin{array}
[c]{ccc}%
a & 0 & 0\\
0 & b & c\\
0 & d & e
\end{array}
\right)  \left(
\begin{array}
[c]{ccc}%
a^{\prime} & 0 & 0\\
0 & b^{\prime} & c^{\prime}\\
0 & d^{\prime} & e^{\prime}%
\end{array}
\right)  =\allowbreak\left(
\begin{array}
[c]{ccc}%
aa^{\prime} & 0 & 0\\
0 & cd^{\prime}+bb^{\prime} & ce^{\prime}+bc^{\prime}\\
0 & d^{\prime}e+db^{\prime} & e^{\prime}e+dc^{\prime}%
\end{array}
\right)  \allowbreak$

\textbf{Non-example:} $\left(
\begin{array}
[c]{ccc}%
1 & 0 & 0\\
0 & 2 & 3\\
0 & 1 & 4
\end{array}
\right)  \left(
\begin{array}
[c]{ccc}%
1 & 2 & 0\\
4 & 3 & 0\\
0 & 0 & 1
\end{array}
\right)  =\allowbreak\left(
\begin{array}
[c]{ccc}%
1 & 2 & 0\\
8 & 6 & 3\\
4 & 3 & 4
\end{array}
\right)  $

\begin{proposition}
Let $A_{1},A_{2},\ldots,A_{u}$ be a bunch of square matrices. Then, the
block-diagonal matrix $\left(
\begin{array}
[c]{cccc}%
A_{1} & 0 & \cdots & 0\\
0 & A_{2} & \cdots & 0\\
\vdots & \vdots & \ddots & \vdots\\
0 & 0 & \cdots & A_{u}%
\end{array}
\right)  $ is unitary if and only $A_{1},A_{2},\ldots,A_{u}$ are unitary.
\end{proposition}

\subsection{The Gram--Schmidt process}

\begin{theorem}
[Gram--Schmidt process]Let $\left(  v_{1},v_{2},\ldots,v_{m}\right)  $ be a
linearly independent tuple of vectors in $\mathbb{C}^{n}$.

Then, there is an orthogonal tuple $\left(  z_{1},z_{2},\ldots,z_{m}\right)  $
of vectors in $\mathbb{C}^{n}$ that satisfies%
\[
\operatorname*{span}\left\{  v_{1},v_{2},\ldots,v_{j}\right\}
=\operatorname*{span}\left\{  z_{1},z_{2},\ldots,z_{j}\right\}
\ \ \ \ \ \ \ \ \ \ \text{for all }j\in\left[  m\right]  .
\]


Furthermore, such a tuple $\left(  z_{1},z_{2},\ldots,z_{m}\right)  $ can be
constructed by the following recursive process:

\begin{itemize}
\item For each $p\in\left[  m\right]  $, if the first $p-1$ entries
$z_{1},z_{2},\ldots,z_{p-1}$ of this tuple have already been constructed, then
we define the $p$-th entry $z_{p}$ by%
\[
z_{p}=v_{p}-\sum_{k=1}^{p-1}\dfrac{\left\langle v_{p},z_{k}\right\rangle
}{\left\langle z_{k},z_{k}\right\rangle }z_{k}.
\]
(Note that when $p=1$, the sum on the RHS is an empty sum, so this equality
simply becomes $z_{1}=v_{1}$.)
\end{itemize}
\end{theorem}

Roughly speaking, the claim of this theorem is that if we start with any
linearly independent tuple $\left(  v_{1},v_{2},\ldots,v_{m}\right)  $ of
vectors in $\mathbb{C}^{n}$, then we can make this tuple orthogonal by
tweaking it as follows:

\begin{itemize}
\item we leave $v_{1}$ unchanged;

\item we modify $v_{2}$ by subtracting an appropriate scalar multiple of
$v_{1}$;

\item we modify $v_{3}$ by subtracting an appropriate linear combination of
$v_{1}$ and $v_{2}$;

\item and so on.
\end{itemize}

The above recursive formula tells us which scalar multiples / linear
combinations to take.

\begin{example}%
\begin{align*}
z_{1}  & =v_{1};\\
z_{2}  & =v_{2}-\dfrac{\left\langle v_{2},z_{1}\right\rangle }{\left\langle
z_{1},z_{1}\right\rangle }z_{1};\\
z_{3}  & =v_{3}-\dfrac{\left\langle v_{3},z_{1}\right\rangle }{\left\langle
z_{1},z_{1}\right\rangle }z_{1}-\dfrac{\left\langle v_{3},z_{2}\right\rangle
}{\left\langle z_{2},z_{2}\right\rangle }z_{2};\\
& \ldots
\end{align*}

\end{example}


\end{document}