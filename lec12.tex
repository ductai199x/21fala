\documentclass[numbers=enddot,12pt,final,onecolumn,notitlepage]{scrartcl}%
\usepackage[headsepline,footsepline,manualmark]{scrlayer-scrpage}
\usepackage[all,cmtip]{xy}
\usepackage{amssymb}
\usepackage{amsmath}
\usepackage{amsthm}
\usepackage{framed}
\usepackage{comment}
\usepackage{color}
\usepackage{hyperref}
\usepackage[sc]{mathpazo}
\usepackage[T1]{fontenc}
\usepackage{tikz}
\usepackage{needspace}
\usepackage{tabls}
\usepackage{wasysym}
\usepackage{easytable}
\usepackage{pythonhighlight}
%TCIDATA{OutputFilter=latex2.dll}
%TCIDATA{Version=5.50.0.2960}
%TCIDATA{LastRevised=Monday, October 18, 2021 11:49:08}
%TCIDATA{SuppressPackageManagement}
%TCIDATA{<META NAME="GraphicsSave" CONTENT="32">}
%TCIDATA{<META NAME="SaveForMode" CONTENT="1">}
%TCIDATA{BibliographyScheme=Manual}
%TCIDATA{Language=American English}
%BeginMSIPreambleData
\providecommand{\U}[1]{\protect\rule{.1in}{.1in}}
%EndMSIPreambleData
\usetikzlibrary{arrows.meta}
\usetikzlibrary{chains}
\newcounter{exer}
\newcounter{exera}
\numberwithin{exer}{subsection}
\theoremstyle{definition}
\newtheorem{theo}{Theorem}[subsection]
\newenvironment{theorem}[1][]
{\begin{theo}[#1]\begin{leftbar}}
{\end{leftbar}\end{theo}}
\newtheorem{lem}[theo]{Lemma}
\newenvironment{lemma}[1][]
{\begin{lem}[#1]\begin{leftbar}}
{\end{leftbar}\end{lem}}
\newtheorem{prop}[theo]{Proposition}
\newenvironment{proposition}[1][]
{\begin{prop}[#1]\begin{leftbar}}
{\end{leftbar}\end{prop}}
\newtheorem{defi}[theo]{Definition}
\newenvironment{definition}[1][]
{\begin{defi}[#1]\begin{leftbar}}
{\end{leftbar}\end{defi}}
\newtheorem{remk}[theo]{Remark}
\newenvironment{remark}[1][]
{\begin{remk}[#1]\begin{leftbar}}
{\end{leftbar}\end{remk}}
\newtheorem{coro}[theo]{Corollary}
\newenvironment{corollary}[1][]
{\begin{coro}[#1]\begin{leftbar}}
{\end{leftbar}\end{coro}}
\newtheorem{conv}[theo]{Convention}
\newenvironment{convention}[1][]
{\begin{conv}[#1]\begin{leftbar}}
{\end{leftbar}\end{conv}}
\newtheorem{quest}[theo]{Question}
\newenvironment{question}[1][]
{\begin{quest}[#1]\begin{leftbar}}
{\end{leftbar}\end{quest}}
\newtheorem{warn}[theo]{Warning}
\newenvironment{warning}[1][]
{\begin{warn}[#1]\begin{leftbar}}
{\end{leftbar}\end{warn}}
\newtheorem{conj}[theo]{Conjecture}
\newenvironment{conjecture}[1][]
{\begin{conj}[#1]\begin{leftbar}}
{\end{leftbar}\end{conj}}
\newtheorem{exam}[theo]{Example}
\newenvironment{example}[1][]
{\begin{exam}[#1]\begin{leftbar}}
{\end{leftbar}\end{exam}}
\newtheorem{exmp}[exer]{Exercise}
\newenvironment{exercise}[1][]
{\begin{exmp}[#1]\begin{leftbar}}
{\end{leftbar}\end{exmp}}
\newenvironment{statement}{\begin{quote}}{\end{quote}}
\newenvironment{fineprint}{\medskip \begin{small}}{\end{small} \medskip}
\iffalse
\newenvironment{proof}[1][Proof]{\noindent\textbf{#1.} }{\ \rule{0.5em}{0.5em}}
\newenvironment{question}[1][Question]{\noindent\textbf{#1.} }{\ \rule{0.5em}{0.5em}}
\newenvironment{warning}[1][Warning]{\noindent\textbf{#1.} }{\ \rule{0.5em}{0.5em}}
\newenvironment{teachingnote}[1][Teaching note]{\noindent\textbf{#1.} }{\ \rule{0.5em}{0.5em}}
\fi
\let\sumnonlimits\sum
\let\prodnonlimits\prod
\let\cupnonlimits\bigcup
\let\capnonlimits\bigcap
\renewcommand{\sum}{\sumnonlimits\limits}
\renewcommand{\prod}{\prodnonlimits\limits}
\renewcommand{\bigcup}{\cupnonlimits\limits}
\renewcommand{\bigcap}{\capnonlimits\limits}
\setlength\tablinesep{3pt}
\setlength\arraylinesep{3pt}
\setlength\extrarulesep{3pt}
\voffset=0cm
\hoffset=-0.7cm
\setlength\textheight{22.5cm}
\setlength\textwidth{15.5cm}
\newcommand\arxiv[1]{\href{http://www.arxiv.org/abs/#1}{\texttt{arXiv:#1}}}
\newenvironment{verlong}{}{}
\newenvironment{vershort}{}{}
\newenvironment{noncompile}{}{}
\newenvironment{teachingnote}{}{}
\excludecomment{verlong}
\includecomment{vershort}
\excludecomment{noncompile}
\excludecomment{teachingnote}
\newcommand{\CC}{\mathbb{C}}
\newcommand{\RR}{\mathbb{R}}
\newcommand{\QQ}{\mathbb{Q}}
\newcommand{\NN}{\mathbb{N}}
\newcommand{\ZZ}{\mathbb{Z}}
\newcommand{\KK}{\mathbb{K}}
\newcommand{\id}{\operatorname{id}}
\newcommand{\lcm}{\operatorname{lcm}}
\newcommand{\rev}{\operatorname{rev}}
\newcommand{\powset}[2][]{\ifthenelse{\equal{#2}{}}{\mathcal{P}\left(#1\right)}{\mathcal{P}_{#1}\left(#2\right)}}
\newcommand{\set}[1]{\left\{ #1 \right\}}
\newcommand{\abs}[1]{\left| #1 \right|}
\newcommand{\tup}[1]{\left( #1 \right)}
\newcommand{\ive}[1]{\left[ #1 \right]}
\newcommand{\floor}[1]{\left\lfloor #1 \right\rfloor}
\newcommand{\lf}[2]{#1^{\underline{#2}}}
\newcommand{\underbrack}[2]{\underbrace{#1}_{\substack{#2}}}
\newcommand{\horrule}[1]{\rule{\linewidth}{#1}}
\newcommand{\are}{\ar@{-}}
\newcommand{\nnn}{\nonumber\\}
\newcommand{\sslash}{\mathbin{/\mkern-6mu/}}
\newcommand{\numboxed}[2]{\underbrace{\boxed{#1}}_{\text{box } #2}}
\newcommand{\ig}[2]{\includegraphics[scale=#1]{#2.png}}
\newcommand{\UNFINISHED}{\begin{center} \Huge{\textbf{Unfinished material begins here.}} \end{center} }
\iffalse
\NOEXPAND{\today}{\today}
\NOEXPAND{\sslash}{\sslash}
\NOEXPAND{\numboxed}[2]{\numboxed}
\NOEXPAND{\UNFINISHED}{\UNFINISHED}
\fi
\ihead{Math 504 notes}
\ohead{page \thepage}
\cfoot{\today}
\begin{document}

\title{Math 504: Advanced Linear Algebra}
\author{Hugo Woerdeman, with edits by Darij Grinberg\thanks{Drexel University, Korman
Center, 15 S 33rd Street, Philadelphia PA, 19104, USA}}
\date{\today\ (unfinished!)}
\maketitle
\tableofcontents

\section*{Math 504 Lecture 12}

\section{Jordan canonical (aka normal) form (cont'd)}

\subsection{The minimal polynomial (cont'd)}

Recall from last lecture:

\begin{definition}
Let $A\in\mathbb{C}^{n\times n}$ be an $n\times n$-matrix. The preceding
theorem shows that there is a \textbf{unique} monic polynomial $q_{A}\left(
t\right)  $ of minimum degree that annihilates $A$. This unique polynomial
will be denoted $q_{A}\left(  t\right)  $ and will be called the
\textbf{minimal polynomial} of $A$.
\end{definition}

\begin{theorem}
Let $A\in\mathbb{C}^{n\times n}$ be an $n\times n$-matrix. Let $f\left(
t\right)  \in\mathbb{C}\left[  t\right]  $ be any polynomial. Then, $f$
annihilates $A$ if and only if $f$ is a multiple of $q_{A}$ (that is,
$f\left(  t\right)  =q_{A}\left(  t\right)  \cdot g\left(  t\right)  $ for
some polynomial $g\left(  t\right)  \in\mathbb{C}\left[  t\right]  $).
\end{theorem}

\begin{corollary}
Let $A\in\mathbb{C}^{n\times n}$ be a matrix. Then, $q_{A}\left(  t\right)
\mid p_{A}\left(  t\right)  $.
\end{corollary}

\begin{proposition}
If $\lambda\in\sigma\left(  A\right)  $, then $q_{A}\left(  \lambda\right)
=0$.
\end{proposition}

Combining the Corollary with the Proposition, we see that the roots of
$q_{A}\left(  t\right)  $ are precisely the eigenvalues of $A$; we just don't
know yet with which multiplicities they appear as roots. In other words, we
have
\[
q_{A}\left(  t\right)  =\left(  t-\lambda_{1}\right)  ^{k_{1}}\left(
t-\lambda_{2}\right)  ^{k_{2}}\cdots\left(  t-\lambda_{p}\right)  ^{k_{p}},
\]
where $\lambda_{1},\lambda_{2},\ldots,\lambda_{p}$ are the distinct
eigenvalues of $A$, and the $k_{1},k_{2},\ldots,k_{p}$ are positive integers;
but we don't know these $k_{1},k_{2},\ldots,k_{p}$ yet. So let us find them.
We will use some lemmas for this.

\begin{lemma}
Let $A$ and $B$ be two similar $n\times n$-matrices. Then, $q_{A}\left(
t\right)  =q_{B}\left(  t\right)  $.
\end{lemma}

\begin{proof}
This is obvious from the viewpoint of endomorphisms. For a pedestrian proof,
you can just argue that a polynomial $f$ annihilates $A$ if and only if it
annihilates $B$. But this is easy: We have $A=SBS^{-1}$ for some invertible
$S$ (since $A$ and $B$ are similar), and therefore every polynomial $f$
satisfies
\[
f\left(  A\right)  =f\left(  SBS^{-1}\right)  =Sf\left(  B\right)  S^{-1}%
\]
and therefore $f\left(  A\right)  =0$ holds if and only if $f\left(  B\right)
=0$.
\end{proof}

We recall the notion of the lcm (= least common multiple) of several
polynomials. It is defined as one would expect: If $p_{1},p_{2},\ldots,p_{m}$
are $m$ nonzero polynomials (in a single indeterminate $t$), then
$\operatorname{lcm}\left(  p_{1},p_{2},\ldots,p_{m}\right)  $ is the monic
polynomial of smallest degree that is a common multiple of $p_{1},p_{2}%
,\ldots,p_{m}$. For example,%
\begin{align*}
\operatorname{lcm}\left(  t^{2}-1,\ t^{3}-1\right)    & =\operatorname{lcm}%
\left(  \left(  t-1\right)  \left(  t+1\right)  ,\ \left(  t-1\right)  \left(
t^{2}+t+1\right)  \right)  \\
& =\allowbreak\left(  t-1\right)  \left(  t+1\right)  \left(  t+t^{2}%
+1\right)  =\allowbreak t^{4}+t^{3}-t-1.
\end{align*}
(Again, the lcm of several polynomials is unique. This can be shown in the
same way that we used to prove uniqueness of the minimal polynomial.)

\begin{lemma}
Let $A_{1},A_{2},\ldots,A_{m}$ be $m$ square matrices. Let%
\[
A=\left(
\begin{array}
[c]{cccc}%
A_{1} &  &  & \\
& A_{2} &  & \\
&  & \ddots & \\
&  &  & A_{m}%
\end{array}
\right)  .
\]
Then,%
\[
q_{A}=\operatorname{lcm}\left(  q_{A_{1}},\ q_{A_{2}},\ \ldots,\ q_{A_{m}%
}\right)  .
\]

\end{lemma}

\begin{proof}
For any polynomial $f\in\mathbb{C}\left[  t\right]  $, we have%
\[
f\left(  A\right)  =f\left(
\begin{array}
[c]{cccc}%
A_{1} &  &  & \\
& A_{2} &  & \\
&  & \ddots & \\
&  &  & A_{m}%
\end{array}
\right)  =\left(
\begin{array}
[c]{cccc}%
f\left(  A_{1}\right)   &  &  & \\
& f\left(  A_{2}\right)   &  & \\
&  & \ddots & \\
&  &  & f\left(  A_{m}\right)
\end{array}
\right)
\]
(indeed, the last equality follows from%
\[
\left(
\begin{array}
[c]{cccc}%
A_{1} &  &  & \\
& A_{2} &  & \\
&  & \ddots & \\
&  &  & A_{m}%
\end{array}
\right)  ^{k}=\left(
\begin{array}
[c]{cccc}%
A_{1}^{k} &  &  & \\
& A_{2}^{k} &  & \\
&  & \ddots & \\
&  &  & A_{m}^{k}%
\end{array}
\right)
\]
and from the fact that a polynomial $f$ is just a $\mathbb{C}$-linear
combination of $t^{k}$s). Thus, $f\left(  A\right)  =0$ holds if and only if%
\[
f\left(  A_{1}\right)  =0\text{ and }f\left(  A_{2}\right)  =0\text{ and
}\cdots\text{ and }f\left(  A_{m}\right)  =0.
\]
However, $f\left(  A\right)  =0$ holds if and only if $f$ is a multiple of
$q_{A}$, whereas $f\left(  A_{i}\right)  =0$ holds if and only if $f$ is a
multiple of $q_{A_{i}}$. Thus, the previous sentence says that $f$ is a
multiple of $q_{A}$ if and only if $f$ is a multiple of all of the $q_{A_{i}}%
$s. In other words, the multiples of $q_{A}$ are precisely the common multiple
of all the $q_{A_{i}}$s. But this is the universal property of the lcm. So
$q_{A}$ is the lcm of the $q_{A_{i}}$s.
\end{proof}

\begin{lemma}
Let $k>0$ and $\lambda\in\mathbb{C}$. Let $A=J_{k}\left(  \lambda\right)  $.
Then,%
\[
q_{A}=\left(  t-\lambda\right)  ^{k}.
\]

\end{lemma}

\begin{proof}
It is easy to see that $q_{A}=q_{A-\lambda I_{k}}\left(  t-\lambda\right)  $,
because for a polynomial $f$ to annihilate $A-\lambda I_{k}$ is the same as
for the polynomial $f\left(  t-\lambda\right)  $ to annihilate $A$. So we need
to find $q_{A-\lambda I_{k}}$. Recall that
\[
A-\lambda I_{k}=J_{k}\left(  0\right)  =\left(
\begin{array}
[c]{ccccc}
& 1 &  &  & \\
&  & 1 &  & \\
&  &  &  & \\
&  &  & \ddots & \\
&  &  &  &
\end{array}
\right)  .
\]
Therefore, for any polynomial $f=f_{0}t^{0}+f_{1}t^{1}+f_{2}t^{2}+\cdots$, we
have%
\[
f\left(  A-\lambda I_{k}\right)  =\left(
\begin{array}
[c]{ccccc}%
f_{0} & f_{1} & f_{2} & \cdots & f_{k-1}\\
& f_{0} & f_{1} & \cdots & f_{k-2}\\
&  & f_{0} & \cdots & f_{k-3}\\
&  &  & \ddots & \vdots\\
&  &  &  & f_{0}%
\end{array}
\right)  .
\]
So $f\left(  A-\lambda I_{k}\right)  =0$ if and only if $f_{0}=f_{1}%
=\cdots=f_{k-1}=0$, i.e., if and only if the first $k$ coefficients of $f$ are
$0$. Now, the monic polynomial of smallest degree whose first $k$ coefficients
are $0$ is the polynomial $t^{k}$. So the monic polynomial $f$ of smallest
degree that satisfies $f\left(  A-\lambda I_{k}\right)  =0$ is $t^{k}$. In
other words, $q_{A-\lambda I_{k}}=t^{k}$.

Now, recall that $q_{A}=q_{A-\lambda I_{k}}\left(  t-\lambda\right)  =\left(
t-\lambda\right)  ^{k}$ (since $q_{A-\lambda I_{k}}=t^{k}$). Qed.
\end{proof}

\begin{theorem}
Let $A\in\mathbb{C}^{n\times n}$ be an $n\times n$-matrix. Let $J$ be the
Jordan canonical form of $A$. Let $\lambda_{1},\lambda_{2},\ldots,\lambda_{p}$
be the distinct eigenvalues of $A$. Then,%
\[
q_{A}=\left(  t-\lambda_{1}\right)  ^{k_{1}}\left(  t-\lambda_{2}\right)
^{k_{2}}\cdots\left(  t-\lambda_{p}\right)  ^{k_{p}},
\]
where $k_{i}$ is the size of the largest Jordan cell at eigenvalue
$\lambda_{i}$ in $J$.
\end{theorem}

\begin{example}
Let $A$ have Jordan canonical form%
\[
J=\left(
\begin{array}
[c]{cccccccc}%
5 & 1 &  &  &  &  &  & \\
& 5 & 1 &  &  &  &  & \\
&  & 5 &  &  &  &  & \\
&  &  & 5 & 1 &  &  & \\
&  &  &  & 5 &  &  & \\
&  &  &  &  & 2 &  & \\
&  &  &  &  &  & 2 & 1\\
&  &  &  &  &  &  & 2
\end{array}
\right)  .
\]
Then,%
\[
q_{A}=\left(  t-5\right)  ^{3}\left(  t-2\right)  ^{2}.
\]

\end{example}

\begin{proof}
[Proof of the Theorem.] We have $A\sim J$, so that $q_{A}=q_{J}$ (by our first lemma).

Recall that $J$ is a Jordan matrix, i.e., a block-diagonal matrix whose
diagonal blocks are Jordan cells $J_{1},J_{2},\ldots,J_{m}$. Thus, by our
second lemma, we have%
\begin{align*}
q_{J}  & =\operatorname{lcm}\left(  q_{J_{1}},\ q_{J_{2}},\ \ldots,\ q_{J_{m}%
}\right)  \\
& =\operatorname{lcm}\left(  \left(  t-\lambda_{J_{1}}\right)  ^{k_{J_{1}}%
},\ \left(  t-\lambda_{J_{2}}\right)  ^{k_{J_{2}}},\ \ldots,\ \left(
t-\lambda_{J_{m}}\right)  ^{k_{J_{m}}}\right)  ,
\end{align*}
where each $J_{i}$ has eigenvalue $\lambda_{J_{i}}$ and size $k_{J_{i}}$ (by
our third lemma). This lcm must be divisible by each $t-\lambda$ at least as
often as each of the $\left(  t-\lambda_{J_{i}}\right)  ^{k_{J_{i}}}$s is;
i.e., it must be divisible by $\left(  t-\lambda\right)  ^{k}$, where $k$ is
the largest size of a Jordan cell of $J$ at eigenvalue $\lambda$. So the lcm
is the product of these $\left(  t-\lambda\right)  ^{k}$s. But this is
precisely our claim.
\end{proof}

\subsection{Application of functions to matrices}

Consider a square matrix $A\in\mathbb{C}^{n\times n}$. We have already defined
what it means to apply a polynomial $f$ to $A$: We just write $f$ as $\sum
_{i}f_{i}t^{i}$, and substitute $A$ for $t$.

Can we do the same with non-polynomial functions $f$ ? For example, can we
define $\exp A$ or $\sin A$ ?

One option to do so is to follow the same rule as for polynomials, but using
the Taylor series for $f$. For example, since $\exp$ has Taylor series $\exp
t=\sum_{i\in\mathbb{N}}\dfrac{t^{i}}{i!}$, we can set%
\[
\exp A=\sum_{i\in\mathbb{N}}\dfrac{A^{i}}{i!}.
\]
This indeed works for $\exp$ and for $\sin$, as the sums you get always
converge. But it doesn't generally work, e.g., for $f=\tan t$, since its
Taylor series only converges in a certain neighborhood of $0$. Is this the
best we can do?

There is a different approach that gives a more general definition.

\begin{lemma}
Let $k>0$ and $\lambda\in\mathbb{C}$. Let $A=J_{k}\left(  \lambda\right)  $.
Then, for any polynomial $f\in\mathbb{C}\left[  t\right]  $, we have%
\[
f\left(  A\right)  =\left(
\begin{array}
[c]{ccccc}%
\dfrac{f\left(  \lambda\right)  }{0!} & \dfrac{f^{\prime}\left(
\lambda\right)  }{1!} & \dfrac{f^{\prime\prime}\left(  \lambda\right)  }{2!} &
\cdots & \dfrac{f^{\left(  k-1\right)  }\left(  \lambda\right)  }{\left(
k-1\right)  !}\\
& \dfrac{f\left(  \lambda\right)  }{0!} & \dfrac{f^{\prime}\left(
\lambda\right)  }{1!} & \cdots & \dfrac{f^{\left(  k-2\right)  }\left(
\lambda\right)  }{\left(  k-2\right)  !}\\
&  & \dfrac{f\left(  \lambda\right)  }{0!} & \cdots & \dfrac{f^{\left(
k-3\right)  }\left(  \lambda\right)  }{\left(  k-3\right)  !}\\
&  &  & \ddots & \vdots\\
&  &  &  & \dfrac{f\left(  \lambda\right)  }{0!}%
\end{array}
\right)
\]

\end{lemma}

\begin{proof}
Exercise.
\end{proof}

Now, we aim to define $f\left(  A\right)  $ by the above formula, at least
when $A$ is a Jordan cell. This only requires $f$ to be $\left(  k-1\right)
$-times differentiable at $\lambda$. 

\begin{definition}
Let $A\in\mathbb{C}^{n\times n}$ be an $n\times n$-matrix that has minimal
polynomial%
\[
q_{A}\left(  t\right)  =\left(  t-\lambda_{1}\right)  ^{k_{1}}\left(
t-\lambda_{2}\right)  ^{k_{2}}\cdots\left(  t-\lambda_{p}\right)  ^{k_{p}},
\]
where the $\lambda_{1},\lambda_{2},\ldots,\lambda_{p}$ are the distinct
eigenvalues of $A$.

Let $f$ be a function from $\mathbb{C}$ to $\mathbb{C}$ that is defined at
each of the numbers $\lambda_{1},\lambda_{2},\ldots,\lambda_{p}$ and is
holomorphic at each of them, or at least $\left(  k_{i}-1\right)  $-times
differentiable at each $\lambda_{i}$ if $\lambda_{i}$ is real. Then, we can
define an $n\times n$-matrix $f\left(  A\right)  \in\mathbb{C}^{n\times n}$ as
follows: Write $A=SJS^{-1}$, where $J$ is a Jordan matrix and $S$ is
invertible. Write $J$ as $\left(
\begin{array}
[c]{cccc}%
J_{1} &  &  & \\
& J_{2} &  & \\
&  & \ddots & \\
&  &  & J_{m}%
\end{array}
\right)  $, where the $J_{1},J_{2},\ldots,J_{m}$ are Jordan cells. Then, we
set%
\begin{align*}
f\left(  A\right)    & :=Sf\left(  J\right)  S^{-1}%
,\ \ \ \ \ \ \ \ \ \ \text{where}\\
f\left(  J\right)    & :=\left(
\begin{array}
[c]{cccc}%
f\left(  J_{1}\right)   &  &  & \\
& f\left(  J_{2}\right)   &  & \\
&  & \ddots & \\
&  &  & f\left(  J_{m}\right)
\end{array}
\right)  ,\ \ \ \ \ \ \ \ \ \ \text{where}\\
f\left(  J_{k}\left(  \lambda\right)  \right)    & :=\left(
\begin{array}
[c]{ccccc}%
\dfrac{f\left(  \lambda\right)  }{0!} & \dfrac{f^{\prime}\left(
\lambda\right)  }{1!} & \dfrac{f^{\prime\prime}\left(  \lambda\right)  }{2!} &
\cdots & \dfrac{f^{\left(  k-1\right)  }\left(  \lambda\right)  }{\left(
k-1\right)  !}\\
& \dfrac{f\left(  \lambda\right)  }{0!} & \dfrac{f^{\prime}\left(
\lambda\right)  }{1!} & \cdots & \dfrac{f^{\left(  k-2\right)  }\left(
\lambda\right)  }{\left(  k-2\right)  !}\\
&  & \dfrac{f\left(  \lambda\right)  }{0!} & \cdots & \dfrac{f^{\left(
k-3\right)  }\left(  \lambda\right)  }{\left(  k-3\right)  !}\\
&  &  & \ddots & \vdots\\
&  &  &  & \dfrac{f\left(  \lambda\right)  }{0!}%
\end{array}
\right)  .
\end{align*}

\end{definition}

\begin{theorem}
This definition is actually well-defined. That is, the value $f\left(
A\right)  $ does not depend on the choice of $S$ and $J$.
\end{theorem}

\begin{exercise}
Prove this.
\end{exercise}

\subsection{The companion matrix}

For each $n\times n$-matrix $A$, we have defined its characteristic polynomial
$p_{A}$ and its minimal polynomial $q_{A}$. What variety of polynomials do we
get this way? Do all characteristic polynomials share some property, or can
any monic polynomial be a characteristic polynomial? The same question for
minimal polynomials?


\end{document}