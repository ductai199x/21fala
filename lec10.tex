\documentclass[numbers=enddot,12pt,final,onecolumn,notitlepage]{scrartcl}%
\usepackage[headsepline,footsepline,manualmark]{scrlayer-scrpage}
\usepackage[all,cmtip]{xy}
\usepackage{amssymb}
\usepackage{amsmath}
\usepackage{amsthm}
\usepackage{framed}
\usepackage{comment}
\usepackage{color}
\usepackage{hyperref}
\usepackage[sc]{mathpazo}
\usepackage[T1]{fontenc}
\usepackage{tikz}
\usepackage{needspace}
\usepackage{tabls}
\usepackage{wasysym}
\usepackage{easytable}
\usepackage{pythonhighlight}
%TCIDATA{OutputFilter=latex2.dll}
%TCIDATA{Version=5.50.0.2960}
%TCIDATA{LastRevised=Wednesday, October 13, 2021 11:47:13}
%TCIDATA{SuppressPackageManagement}
%TCIDATA{<META NAME="GraphicsSave" CONTENT="32">}
%TCIDATA{<META NAME="SaveForMode" CONTENT="1">}
%TCIDATA{BibliographyScheme=Manual}
%TCIDATA{Language=American English}
%TCIDATA{ComputeDefs=
%$A=\left(
%\begin{array}
%[c]{cccccc}%
%0 & 1 & 0 & -1 & 1 & -1\\
%0 & 1 & 1 & -2 & 2 & -2\\
%0 & 1 & 0 & -1 & 2 & -2\\
%0 & 1 & 0 & -1 & 2 & -2\\
%0 & 1 & 0 & -1 & 1 & -1\\
%0 & 1 & 0 & -1 & 1 & -1
%\end{array}
%\right)  $
%$S=\left(
%\begin{array}
%[c]{ccc}%
%0 & 1 & 0\\
%1 & 0 & 0\\
%-1 & 0 & 1
%\end{array}
%\right)  $
%}
%BeginMSIPreambleData
\providecommand{\U}[1]{\protect\rule{.1in}{.1in}}
%EndMSIPreambleData
\usetikzlibrary{arrows.meta}
\usetikzlibrary{chains}
\newcounter{exer}
\newcounter{exera}
\numberwithin{exer}{subsection}
\theoremstyle{definition}
\newtheorem{theo}{Theorem}[subsection]
\newenvironment{theorem}[1][]
{\begin{theo}[#1]\begin{leftbar}}
{\end{leftbar}\end{theo}}
\newtheorem{lem}[theo]{Lemma}
\newenvironment{lemma}[1][]
{\begin{lem}[#1]\begin{leftbar}}
{\end{leftbar}\end{lem}}
\newtheorem{prop}[theo]{Proposition}
\newenvironment{proposition}[1][]
{\begin{prop}[#1]\begin{leftbar}}
{\end{leftbar}\end{prop}}
\newtheorem{defi}[theo]{Definition}
\newenvironment{definition}[1][]
{\begin{defi}[#1]\begin{leftbar}}
{\end{leftbar}\end{defi}}
\newtheorem{remk}[theo]{Remark}
\newenvironment{remark}[1][]
{\begin{remk}[#1]\begin{leftbar}}
{\end{leftbar}\end{remk}}
\newtheorem{coro}[theo]{Corollary}
\newenvironment{corollary}[1][]
{\begin{coro}[#1]\begin{leftbar}}
{\end{leftbar}\end{coro}}
\newtheorem{conv}[theo]{Convention}
\newenvironment{convention}[1][]
{\begin{conv}[#1]\begin{leftbar}}
{\end{leftbar}\end{conv}}
\newtheorem{quest}[theo]{Question}
\newenvironment{question}[1][]
{\begin{quest}[#1]\begin{leftbar}}
{\end{leftbar}\end{quest}}
\newtheorem{warn}[theo]{Warning}
\newenvironment{warning}[1][]
{\begin{warn}[#1]\begin{leftbar}}
{\end{leftbar}\end{warn}}
\newtheorem{conj}[theo]{Conjecture}
\newenvironment{conjecture}[1][]
{\begin{conj}[#1]\begin{leftbar}}
{\end{leftbar}\end{conj}}
\newtheorem{exam}[theo]{Example}
\newenvironment{example}[1][]
{\begin{exam}[#1]\begin{leftbar}}
{\end{leftbar}\end{exam}}
\newtheorem{exmp}[exer]{Exercise}
\newenvironment{exercise}[1][]
{\begin{exmp}[#1]\begin{leftbar}}
{\end{leftbar}\end{exmp}}
\newenvironment{statement}{\begin{quote}}{\end{quote}}
\newenvironment{fineprint}{\medskip \begin{small}}{\end{small} \medskip}
\iffalse
\newenvironment{proof}[1][Proof]{\noindent\textbf{#1.} }{\ \rule{0.5em}{0.5em}}
\newenvironment{question}[1][Question]{\noindent\textbf{#1.} }{\ \rule{0.5em}{0.5em}}
\newenvironment{warning}[1][Warning]{\noindent\textbf{#1.} }{\ \rule{0.5em}{0.5em}}
\newenvironment{teachingnote}[1][Teaching note]{\noindent\textbf{#1.} }{\ \rule{0.5em}{0.5em}}
\fi
\let\sumnonlimits\sum
\let\prodnonlimits\prod
\let\cupnonlimits\bigcup
\let\capnonlimits\bigcap
\renewcommand{\sum}{\sumnonlimits\limits}
\renewcommand{\prod}{\prodnonlimits\limits}
\renewcommand{\bigcup}{\cupnonlimits\limits}
\renewcommand{\bigcap}{\capnonlimits\limits}
\setlength\tablinesep{3pt}
\setlength\arraylinesep{3pt}
\setlength\extrarulesep{3pt}
\voffset=0cm
\hoffset=-0.7cm
\setlength\textheight{22.5cm}
\setlength\textwidth{15.5cm}
\newcommand\arxiv[1]{\href{http://www.arxiv.org/abs/#1}{\texttt{arXiv:#1}}}
\newenvironment{verlong}{}{}
\newenvironment{vershort}{}{}
\newenvironment{noncompile}{}{}
\newenvironment{teachingnote}{}{}
\excludecomment{verlong}
\includecomment{vershort}
\excludecomment{noncompile}
\excludecomment{teachingnote}
\newcommand{\CC}{\mathbb{C}}
\newcommand{\RR}{\mathbb{R}}
\newcommand{\QQ}{\mathbb{Q}}
\newcommand{\NN}{\mathbb{N}}
\newcommand{\ZZ}{\mathbb{Z}}
\newcommand{\KK}{\mathbb{K}}
\newcommand{\id}{\operatorname{id}}
\newcommand{\lcm}{\operatorname{lcm}}
\newcommand{\rev}{\operatorname{rev}}
\newcommand{\powset}[2][]{\ifthenelse{\equal{#2}{}}{\mathcal{P}\left(#1\right)}{\mathcal{P}_{#1}\left(#2\right)}}
\newcommand{\set}[1]{\left\{ #1 \right\}}
\newcommand{\abs}[1]{\left| #1 \right|}
\newcommand{\tup}[1]{\left( #1 \right)}
\newcommand{\ive}[1]{\left[ #1 \right]}
\newcommand{\floor}[1]{\left\lfloor #1 \right\rfloor}
\newcommand{\lf}[2]{#1^{\underline{#2}}}
\newcommand{\underbrack}[2]{\underbrace{#1}_{\substack{#2}}}
\newcommand{\horrule}[1]{\rule{\linewidth}{#1}}
\newcommand{\are}{\ar@{-}}
\newcommand{\nnn}{\nonumber\\}
\newcommand{\sslash}{\mathbin{/\mkern-6mu/}}
\newcommand{\numboxed}[2]{\underbrace{\boxed{#1}}_{\text{box } #2}}
\newcommand{\ig}[2]{\includegraphics[scale=#1]{#2.png}}
\newcommand{\UNFINISHED}{\begin{center} \Huge{\textbf{Unfinished material begins here.}} \end{center} }
\iffalse
\NOEXPAND{\today}{\today}
\NOEXPAND{\sslash}{\sslash}
\NOEXPAND{\numboxed}[2]{\numboxed}
\NOEXPAND{\UNFINISHED}{\UNFINISHED}
\fi
\ihead{Math 504 notes}
\ohead{page \thepage}
\cfoot{\today}
\begin{document}

\title{Math 504: Advanced Linear Algebra}
\author{Hugo Woerdeman, with edits by Darij Grinberg\thanks{Drexel University, Korman
Center, 15 S 33rd Street, Philadelphia PA, 19104, USA}}
\date{\today\ (unfinished!)}
\maketitle
\tableofcontents

\section*{Math 504 Lecture 10}

\section{Jordan canonical (aka normal) form (cont'd)}

\subsection{Step 3: Strictly upper-triangular matrices redux}

Let us fill in what we couldn't back in Lecture 9.

Consider a strictly upper-triangular $n\times n$-matrix $A$ (that is, an
upper-triangular matrix whose diagonal entries are $0$ as well).

We want to find a basis $\left(  s_{1},s_{2},\ldots,s_{n}\right)  $ of
$\mathbb{C}^{n}$ such that for each $i\in\left[  n\right]  $, the vector
$As_{i}$ is either $s_{i-1}$ or $0$. (When $i=1$, this vector has to be $0$,
since there is no $s_{0}$.) In fact, if $\left(  s_{1},s_{2},\ldots
,s_{n}\right)  $ is such a basis, then the matrix $S:=\left(
\begin{array}
[c]{cccc}%
s_{1} & s_{2} & \cdots & s_{n}%
\end{array}
\right)  \in\mathbb{C}^{n\times n}$ is invertible and satisfies%
\begin{align*}
AS  & =\left(  \text{a matrix whose }i\text{-th column is either }%
s_{i-1}\text{ or }0\text{ for each }i\right)  \\
& =S\left(
\begin{array}
[c]{ccccc}%
0 & 1 & 0 & 0 & 0\\
0 & 0 & 0 & 0 & 0\\
0 & 0 & 0 & 1 & 0\\
0 & 0 & 0 & 0 & 1\\
0 & 0 & 0 & 0 & 0
\end{array}
\right)  \ \ \ \ \ \ \ \ \ \ \left(  \text{for example}\right)  ,
\end{align*}
so that%
\[
S^{-1}AS=\left(
\begin{array}
[c]{ccccc}%
0 & 1 & 0 & 0 & 0\\
0 & 0 & 0 & 0 & 0\\
0 & 0 & 0 & 1 & 0\\
0 & 0 & 0 & 0 & 1\\
0 & 0 & 0 & 0 & 0
\end{array}
\right)  =\left(  \text{a Jordan matrix}\right)  .
\]
So $A$ is similar to a Jordan matrix.

How do we find such a basis $\left(  s_{1},s_{2},\ldots,s_{n}\right)  $ ?

(The following proof is due to Terence Tao.)

We define an \textbf{orbit} to be a tuple of the form $\left(  v,Av,A^{2}%
v,\ldots,A^{k}v\right)  $, where $v\in\mathbb{C}^{n}$ satisfies $A^{k+1}v=0$.
Note that for each $v\in\mathbb{C}^{n}$, there is an orbit that starts with
$v$, since $A^{n}=0$.

The \textbf{concatenation} of some tuples $\left(  a_{1},a_{2},\ldots
,a_{k}\right)  $ and $\left(  b_{1},b_{2},\ldots,b_{\ell}\right)  $ and
$\left(  c_{1},c_{2},\ldots,c_{m}\right)  $ is $\left(  a_{1},a_{2}%
,\ldots,a_{k},b_{1},b_{2},\ldots,b_{\ell},c_{1},c_{2},\ldots,c_{m}\right)  $.

Now, I claim:

\begin{lemma}
[orbit basis lemma]There exists a basis of $\mathbb{C}^{n}$ that is a
concatenation of orbits.
\end{lemma}

Once this lemma is proved, we will be done, because reading such a basis
backwards gives us exactly the basis  $\left(  s_{1},s_{2},\ldots
,s_{n}\right)  $ we are looking for. For example, if our basis that is a
concatenation of orbits is%
\[
\left(  u,Au,A^{2}u,\ \ v,Av,A^{2}v,A^{3}v,\ \ w,Aw\right)
\]
(with $A^{3}u=0$ and $A^{4}v=0$ and $A^{2}w=0$), then reading it backwards
gives
\[
\left(  Aw,w,\ \ A^{3}v,A^{2}v,Av,v,\ \ A^{2}u,Au,u\right)  ,
\]
which is a basis $\left(  s_{1},s_{2},\ldots,s_{n}\right)  $ of $\mathbb{C}%
^{n}$ such that for each $i\in\left[  n\right]  $, the vector $As_{i}$ is
either $s_{i-1}$ or $0$.

\begin{proof}
[Proof of the Lemma.] It is easy to find a finite \textbf{spanning set} of
$\mathbb{C}^{n}$ that is a concatenation of orbits. Indeed, we can start with
the standard basis $\left(  e_{1},e_{2},\ldots,e_{n}\right)  $, and extend it
to the list%
\begin{align*}
& (e_{1},Ae_{1},A^{2}e_{1},\ldots,A^{n-1}e_{1},\\
& \ e_{2},Ae_{2},A^{2}e_{2},\ldots,A^{n-1}e_{2},\\
& \ \ldots,\\
& \ e_{n},Ae_{n},A^{2}e_{n},\ldots,A^{n-1}e_{n}).
\end{align*}
This is clearly a spanning set of $\mathbb{C}^{n}$ (since $e_{1},e_{2}%
,\ldots,e_{n}$ already span $\mathbb{C}^{n}$), and also a concatenation of
orbits (since $A^{n}=0$).

Now, we will gradually shorten this spanning set (i.e., replace it by smaller
ones) until we get a basis. We have to do this in such a way that it remains a
spanning set throughout the process, and that it remains a concatenation of
orbits throughout the process.

For the sake of concreteness, let us assume that our spanning set is%
\[
\left(  x,Ax,\ \ y,Ay,A^{2}y,\ \ z,Az,A^{2}z,A^{3}z,\ \ w\right)  ,
\]
with $A^{2}x=0$ and $A^{3}y=0$ and $A^{4}z=0$ and $Aw=0$. If this spanning set
is linearly independent, then it is already a basis, and we are done. So
assume that it isn't. Thus, there exists some linear dependence relation --
say,%
\[
3x+4Ax+5Ay+6A^{2}y+7A^{2}z+8w=0.
\]
Apply $A$ to this relation:%
\begin{align*}
3Ax+4A^{2}x+5A^{2}y+6A^{3}y+7A^{3}z+8Aw  & =0,\ \ \ \ \ \ \ \ \ \ \text{i.e.}%
\\
3Ax+5A^{2}y+7A^{3}z  & =0.
\end{align*}
Apply $A$ to this relation:%
\begin{align*}
3A^{2}x+5A^{3}y+7A^{4}z  & =0,\ \ \ \ \ \ \ \ \ \ \text{i.e.}\\
0  & =0.
\end{align*}
We have gone too far, so let us revert to the previous equation:%
\[
3Ax+5A^{2}y+7A^{3}z=0.
\]
So this is a linear dependence relation between the \textbf{final} vectors of
the orbits in our spanning set. (\textquotedblleft Final\textquotedblright%
\ means the last vector in the orbit.) Factoring out an $A$ in this relation,
we obtain%
\[
A\left(  3x+5Ay+7A^{2}z\right)  =0.
\]
So the $1$-tuple $\left(  3x+5Ay+7A^{2}z\right)  $ is an orbit.

Now, let us replace the orbit $\left(  x,Ax\right)  $ in our spanning set
$\left(  x,Ax,\ \ y,Ay,A^{2}y,\ \ z,Az,A^{2}z,A^{3}z,\ \ w\right)  $ by the
orbit $\left(  3x+5Ay+7A^{2}z\right)  $. We get
\[
\left(  3x+5Ay+7A^{2}z,\ \ y,Ay,A^{2}y,\ \ z,Az,A^{2}z,A^{3}z,\ \ w\right)  .
\]
This is still a concatenation of orbits, since the $1$-tuple $\left(
3x+5Ay+7A^{2}z\right)  $ is an orbit. Furthermore, this is still a spanning
set of $\mathbb{C}^{n}$; why? Because we removed the dependent vector $Ax$
(this is a combination of the other vectors, because $3Ax+5A^{2}y+7A^{3}z=0$)
and we replaced $x$ by $3x+5Ay+7A^{2}z$ (which does not change the span,
because $Ay$ and $A^{2}z$ are still in the spanning set).

This example generalizes. In the general case, you have a spanning set that is
a concatenation of orbits:%
\[
\left(  v_{1},Av_{1},\ldots,A^{m_{1}}v_{1},\ v_{2},Av_{2},\ldots,A^{m_{2}%
}v_{2},\ \ldots,\ v_{k},Av_{k},\ldots,A^{m_{k}}v_{k}\right)  .
\]
If it is a basis, you are done. If not, you pick a linear dependence relation:%
\[
\sum_{i,j}\lambda_{i,j}A^{j}v_{i}=0.
\]
By multiplying this by $A$ an appropriate amount of times (namely, you keep
multiplying until it becomes $0=0$, and then you take a step back), you obtain
a linear dependence relation that involves only the \textbf{final} vectors of
the orbits (i.e., the vectors $A^{m_{1}}v_{1},\ A^{m_{2}}v_{2},\ \ldots
,\ A^{m_{k}}v_{k}$). So it will look like this:%
\[
\mu_{1}A^{m_{1}}v_{1}+\mu_{2}A^{m_{2}}v_{2}+\cdots+\mu_{k}A^{m_{k}}v_{k}=0.
\]
Assume WLOG that the first $p$ of the $\mu_{1},\mu_{2},\ldots,\mu_{k}$ are
nonzero, while the remaining $k-p$ are $0$. So the relation becomes%
\[
\mu_{1}A^{m_{1}}v_{1}+\mu_{2}A^{m_{2}}v_{2}+\cdots+\mu_{p}A^{m_{p}}v_{p}=0,
\]
with $\mu_{1},\mu_{2},\ldots,\mu_{p}$ being nonzero. Assume WLOG that
$m_{1}=\min\left\{  m_{1},m_{2},\ldots,m_{p}\right\}  $, and factor out
$A^{m_{1}}$ from this relation. This yields%
\[
A^{m_{1}}\left(  \mu_{1}v_{1}+\mu_{2}A^{m_{2}-m_{1}}v_{2}+\cdots+\mu
_{p}A^{m_{p}-m_{1}}v_{p}\right)  =0.
\]
Now, set $w_{1}=\mu_{1}v_{1}+\mu_{2}A^{m_{2}-m_{1}}v_{2}+\cdots+\mu
_{p}A^{m_{p}-m_{1}}v_{p}$. Thus, $A^{m_{1}}w_{1}=0$. Hence, $\left(
w_{1},Aw_{1},A^{2}w_{1},\ldots,A^{m_{1}-1}w_{1}\right)  $ is an orbit of
length $m_{1}$. Now, replace the orbit $\left(  v_{1},Av_{1},\ldots,A^{m_{1}%
}v_{1}\right)  $ in the spanning set%
\[
\left(  v_{1},Av_{1},\ldots,A^{m_{1}}v_{1},\ v_{2},Av_{2},\ldots,A^{m_{2}%
}v_{2},\ \ldots,\ v_{k},Av_{k},\ldots,A^{m_{k}}v_{k}\right)
\]
by the shorter orbit $\left(  w_{1},Aw_{1},A^{2}w_{1},\ldots,A^{m_{1}-1}%
w_{1}\right)  $. The resulting list%
\[
\left(  w_{1},Aw_{1},A^{2}w_{1},\ldots,A^{m_{1}-1}w_{1},\ v_{2},Av_{2}%
,\ldots,A^{m_{2}}v_{2},\ \ldots,\ v_{k},Av_{k},\ldots,A^{m_{k}}v_{k}\right)
\]
is still a concatenation of orbits. Also, it still spans $\mathbb{C}^{n}$,
because%
\begin{align*}
w_{1}  & =\underbrace{\mu_{1}}_{\neq0}v_{1}+\mu_{2}A^{m_{2}-m_{1}}v_{2}%
+\cdots+\mu_{p}A^{m_{p}-m_{1}}v_{p};\\
Aw_{1}  & =\underbrace{\mu_{1}}_{\neq0}Av_{1}+\mu_{2}A^{m_{2}-m_{1}+1}%
v_{2}+\cdots+\mu_{p}A^{m_{p}-m_{1}+1}v_{p};\\
& \ldots;\\
A^{m_{1}}v_{1}  & =-\left(  \mu_{2}A^{m_{2}}v_{2}+\cdots+\mu_{p}A^{m_{p}}%
v_{p}\right)  \\
& \ \ \ \ \ \ \ \ \ \ \left(  \text{since }\mu_{1}A^{m_{1}}v_{1}+\mu
_{2}A^{m_{2}}v_{2}+\cdots+\mu_{p}A^{m_{p}}v_{p}=0\right)  .
\end{align*}


So we have found a spanning set of $\mathbb{C}^{n}$ that is still a
concatenation of orbits, but is shorter (it has one less vector). Doing this
repeatedly, we will eventually obtain a basis (since we cannot keep making a
finite list shorter and shorter indefinitely). This proves the lemma.
\end{proof}

As we said, the lemma gives us a basis $\left(  s_{1},s_{2},\ldots
,s_{n}\right)  $ such that $As_{i}$ is either $s_{i-1}$ or $0$; and that shows
that $A$ is similar to a Jordan matrix. This completes the proof of the
existence part of the Jordan canonical form.

\begin{example}
Let
\[
A=\left(
\begin{array}
[c]{cccccc}%
0 & 1 & 0 & -1 & 1 & -1\\
0 & 1 & 1 & -2 & 2 & -2\\
0 & 1 & 0 & -1 & 2 & -2\\
0 & 1 & 0 & -1 & 2 & -2\\
0 & 1 & 0 & -1 & 1 & -1\\
0 & 1 & 0 & -1 & 1 & -1
\end{array}
\right)  .
\]
This is not strictly upper-triangular, but it is nilpotent, with $A^{3}=0$, so
the above argument goes equally well with this $A$. 

Let us try to find a basis of $\mathbb{C}^{6}$ that is a concatenation of orbits.

We begin with the spanning set%
\[
\left(  e_{1},Ae_{1},A^{2}e_{1},\ \ e_{2},Ae_{2},A^{2}e_{2},\ \ \ldots
,\ \ e_{6},Ae_{6},A^{2}e_{6}\right)  .
\]
It has lots of linear dependencies. For one, $Ae_{1}=0$. Multiplying it by $A$
gives $A^{2}e_{1}=0$, so we can replace $\left(  e_{1},Ae_{1},A^{2}%
e_{1}\right)  $ by $\left(  e_{1},Ae_{1}\right)  $. So our spanning set
becomes%
\[
\left(  e_{1},Ae_{1},\ \ e_{2},Ae_{2},A^{2}e_{2},\ \ \ldots,\ \ e_{6}%
,Ae_{6},A^{2}e_{6}\right)  .
\]
One more step of the same form gives%
\[
\left(  e_{1},\ \ e_{2},Ae_{2},A^{2}e_{2},\ \ \ldots,\ \ e_{6},Ae_{6}%
,A^{2}e_{6}\right)  .
\]


Now, observe that $Ae_{3}=e_{2}$. That is, $e_{2}-Ae_{3}=0$. Multiplying it by
$A^{2}$, we obtain $A^{2}e_{2}=0$ (since $A^{2}\cdot Ae_{3}=A^{3}e_{3}=0$). So
we replace the orbit $\left(  e_{2},Ae_{2},A^{2}e_{2}\right)  $ by $\left(
e_{2},Ae_{2}\right)  $. So we get the spanning set
\[
\left(  e_{1},\ \ e_{2},Ae_{2},\ \ e_{3},Ae_{3},A^{2}e_{3},\ \ e_{4}%
,Ae_{4},A^{2}e_{4},\ \ e_{5},Ae_{5},A^{2}e_{5},\ \ e_{6},Ae_{6},A^{2}%
e_{6}\right)  .
\]


We observe that%
\[
Ae_{2}=e_{1}+e_{2}+e_{3}+e_{4}+e_{5}+e_{6}.
\]
In other words,%
\[
Ae_{2}-e_{1}-e_{2}-e_{3}-e_{4}-e_{5}-e_{6}=0.
\]
Multiplying this by $A^{2}$, we obtain%
\[
-A^{2}e_{3}-A^{2}e_{4}-A^{2}e_{5}-A^{2}e_{6}=0.
\]
In other words,%
\[
A^{2}\left(  -e_{3}-e_{4}-e_{5}-e_{6}\right)  =0.
\]
Thus, we set $w_{1}:=-e_{3}-e_{4}-e_{5}-e_{6}$, and we replace $\left(
e_{3},Ae_{3},A^{2}e_{3}\right)  $ by $\left(  w_{1},Aw_{1}\right)  $. So we
get the spanning set
\[
\left(  e_{1},\ \ e_{2},Ae_{2},\ \ w_{1},Aw_{1},\ \ e_{4},Ae_{4},A^{2}%
e_{4},\ \ e_{5},Ae_{5},A^{2}e_{5},\ \ e_{6},Ae_{6},A^{2}e_{6}\right)  .
\]
Keep making these steps. Eventually, there will be no more linear
dependencies, so we will have a basis.
\end{example}


\end{document}