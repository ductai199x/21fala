\documentclass[numbers=enddot,12pt,final,onecolumn,notitlepage]{scrartcl}%
\usepackage[headsepline,footsepline,manualmark]{scrlayer-scrpage}
\usepackage[all,cmtip]{xy}
\usepackage{amssymb}
\usepackage{amsmath}
\usepackage{amsthm}
\usepackage{framed}
\usepackage{comment}
\usepackage{color}
\usepackage{hyperref}
\usepackage[sc]{mathpazo}
\usepackage[T1]{fontenc}
\usepackage{tikz}
\usepackage{needspace}
\usepackage{tabls}
\usepackage{wasysym}
\usepackage{easytable}
\usepackage{pythonhighlight}
%TCIDATA{OutputFilter=latex2.dll}
%TCIDATA{Version=5.50.0.2960}
%TCIDATA{LastRevised=Monday, November 15, 2021 11:50:55}
%TCIDATA{SuppressPackageManagement}
%TCIDATA{<META NAME="GraphicsSave" CONTENT="32">}
%TCIDATA{<META NAME="SaveForMode" CONTENT="1">}
%TCIDATA{BibliographyScheme=Manual}
%TCIDATA{Language=American English}
%BeginMSIPreambleData
\providecommand{\U}[1]{\protect\rule{.1in}{.1in}}
%EndMSIPreambleData
\usetikzlibrary{arrows.meta}
\usetikzlibrary{chains}
\newcounter{exer}
\newcounter{exera}
\numberwithin{exer}{subsection}
\theoremstyle{definition}
\newtheorem{theo}{Theorem}[subsection]
\newenvironment{theorem}[1][]
{\begin{theo}[#1]\begin{leftbar}}
{\end{leftbar}\end{theo}}
\newtheorem{lem}[theo]{Lemma}
\newenvironment{lemma}[1][]
{\begin{lem}[#1]\begin{leftbar}}
{\end{leftbar}\end{lem}}
\newtheorem{prop}[theo]{Proposition}
\newenvironment{proposition}[1][]
{\begin{prop}[#1]\begin{leftbar}}
{\end{leftbar}\end{prop}}
\newtheorem{defi}[theo]{Definition}
\newenvironment{definition}[1][]
{\begin{defi}[#1]\begin{leftbar}}
{\end{leftbar}\end{defi}}
\newtheorem{remk}[theo]{Remark}
\newenvironment{remark}[1][]
{\begin{remk}[#1]\begin{leftbar}}
{\end{leftbar}\end{remk}}
\newtheorem{coro}[theo]{Corollary}
\newenvironment{corollary}[1][]
{\begin{coro}[#1]\begin{leftbar}}
{\end{leftbar}\end{coro}}
\newtheorem{conv}[theo]{Convention}
\newenvironment{convention}[1][]
{\begin{conv}[#1]\begin{leftbar}}
{\end{leftbar}\end{conv}}
\newtheorem{quest}[theo]{Question}
\newenvironment{question}[1][]
{\begin{quest}[#1]\begin{leftbar}}
{\end{leftbar}\end{quest}}
\newtheorem{warn}[theo]{Warning}
\newenvironment{warning}[1][]
{\begin{warn}[#1]\begin{leftbar}}
{\end{leftbar}\end{warn}}
\newtheorem{conj}[theo]{Conjecture}
\newenvironment{conjecture}[1][]
{\begin{conj}[#1]\begin{leftbar}}
{\end{leftbar}\end{conj}}
\newtheorem{exam}[theo]{Example}
\newenvironment{example}[1][]
{\begin{exam}[#1]\begin{leftbar}}
{\end{leftbar}\end{exam}}
\newtheorem{exmp}[exer]{Exercise}
\newenvironment{exercise}[1][]
{\begin{exmp}[#1]\begin{leftbar}}
{\end{leftbar}\end{exmp}}
\newenvironment{statement}{\begin{quote}}{\end{quote}}
\newenvironment{fineprint}{\medskip \begin{small}}{\end{small} \medskip}
\iffalse
\newenvironment{proof}[1][Proof]{\noindent\textbf{#1.} }{\ \rule{0.5em}{0.5em}}
\newenvironment{question}[1][Question]{\noindent\textbf{#1.} }{\ \rule{0.5em}{0.5em}}
\newenvironment{warning}[1][Warning]{\noindent\textbf{#1.} }{\ \rule{0.5em}{0.5em}}
\newenvironment{teachingnote}[1][Teaching note]{\noindent\textbf{#1.} }{\ \rule{0.5em}{0.5em}}
\fi
\let\sumnonlimits\sum
\let\prodnonlimits\prod
\let\cupnonlimits\bigcup
\let\capnonlimits\bigcap
\renewcommand{\sum}{\sumnonlimits\limits}
\renewcommand{\prod}{\prodnonlimits\limits}
\renewcommand{\bigcup}{\cupnonlimits\limits}
\renewcommand{\bigcap}{\capnonlimits\limits}
\setlength\tablinesep{3pt}
\setlength\arraylinesep{3pt}
\setlength\extrarulesep{3pt}
\voffset=0cm
\hoffset=-0.7cm
\setlength\textheight{22.5cm}
\setlength\textwidth{15.5cm}
\newcommand\arxiv[1]{\href{http://www.arxiv.org/abs/#1}{\texttt{arXiv:#1}}}
\newenvironment{verlong}{}{}
\newenvironment{vershort}{}{}
\newenvironment{noncompile}{}{}
\newenvironment{teachingnote}{}{}
\excludecomment{verlong}
\includecomment{vershort}
\excludecomment{noncompile}
\excludecomment{teachingnote}
\newcommand{\CC}{\mathbb{C}}
\newcommand{\RR}{\mathbb{R}}
\newcommand{\QQ}{\mathbb{Q}}
\newcommand{\NN}{\mathbb{N}}
\newcommand{\ZZ}{\mathbb{Z}}
\newcommand{\KK}{\mathbb{K}}
\newcommand{\id}{\operatorname{id}}
\newcommand{\lcm}{\operatorname{lcm}}
\newcommand{\rev}{\operatorname{rev}}
\newcommand{\powset}[2][]{\ifthenelse{\equal{#2}{}}{\mathcal{P}\left(#1\right)}{\mathcal{P}_{#1}\left(#2\right)}}
\newcommand{\set}[1]{\left\{ #1 \right\}}
\newcommand{\abs}[1]{\left| #1 \right|}
\newcommand{\tup}[1]{\left( #1 \right)}
\newcommand{\ive}[1]{\left[ #1 \right]}
\newcommand{\floor}[1]{\left\lfloor #1 \right\rfloor}
\newcommand{\lf}[2]{#1^{\underline{#2}}}
\newcommand{\underbrack}[2]{\underbrace{#1}_{\substack{#2}}}
\newcommand{\horrule}[1]{\rule{\linewidth}{#1}}
\newcommand{\are}{\ar@{-}}
\newcommand{\nnn}{\nonumber\\}
\newcommand{\sslash}{\mathbin{/\mkern-6mu/}}
\newcommand{\numboxed}[2]{\underbrace{\boxed{#1}}_{\text{box } #2}}
\newcommand{\ig}[2]{\includegraphics[scale=#1]{#2.png}}
\newcommand{\UNFINISHED}{\begin{center} \Huge{\textbf{Unfinished material begins here.}} \end{center} }
\iffalse
\NOEXPAND{\today}{\today}
\NOEXPAND{\sslash}{\sslash}
\NOEXPAND{\numboxed}[2]{\numboxed}
\NOEXPAND{\UNFINISHED}{\UNFINISHED}
\fi
\ihead{Math 504 notes}
\ohead{page \thepage}
\cfoot{\today}
\begin{document}

\title{Math 504: Advanced Linear Algebra}
\author{Hugo Woerdeman, with edits by Darij Grinberg\thanks{Drexel University, Korman
Center, 15 S 33rd Street, Philadelphia PA, 19104, USA}}
\date{\today\ (unfinished!)}
\maketitle
\tableofcontents

\section*{Math 504 Lecture 19}

\section{Hermitian matrices (cont'd)}

\subsection{Consequences of the interlacing theorem}

Recall: If $A\in\mathbb{C}^{n\times n}$ is a Hermitian matrix (i.e., a square
matrix satisfying $A^{\ast}=A$), then we denote its eigenvalues by
$\lambda_{1}\left(  A\right)  ,\lambda_{2}\left(  A\right)  ,\ldots
,\lambda_{n}\left(  A\right)  $ in weakly increasing order (with
multiplicities). This makes sense, since we know that these eigenvalues are reals.

Last time, Hugo proved:

\begin{theorem}
[Cauchy's interlacing theorem, aka eigenvalue interlacing theorem]Let
$A\in\mathbb{C}^{n\times n}$ be a Hermitian matrix. Let $j\in\left[  n\right]
$. Let $B\in\mathbb{C}^{\left(  n-1\right)  \times\left(  n-1\right)  }$ be
the matrix obtained from $A$ by removing the $j$-th row and the $j$-th column.
Then,%
\[
\lambda_{1}\left(  A\right)  \leq\lambda_{1}\left(  B\right)  \leq\lambda
_{2}\left(  A\right)  \leq\lambda_{2}\left(  B\right)  \leq\cdots\leq
\lambda_{n-1}\left(  A\right)  \leq\lambda_{n-1}\left(  B\right)  \leq
\lambda_{n}\left(  A\right)  .
\]
In other words,%
\[
\lambda_{i}\left(  A\right)  \leq\lambda_{i}\left(  B\right)  \leq
\lambda_{i+1}\left(  A\right)  \ \ \ \ \ \ \ \ \ \ \text{for each }i\in\left[
n-1\right]  .
\]

\end{theorem}

A converse of this theorem also holds:

\begin{proposition}
Let $\lambda_{1},\lambda_{2},\ldots,\lambda_{n}$ and $\mu_{1},\mu_{2}%
,\ldots,\mu_{n-1}$ be real numbers satisfying
\[
\lambda_{1}\leq\mu_{1}\leq\lambda_{2}\leq\mu_{2}\leq\cdots\leq\lambda
_{n-1}\leq\mu_{n-1}\leq\lambda_{n}.
\]
Then, there exist $n-1$ reals $y_{1},y_{2},\ldots,y_{n-1}\in\mathbb{R}$ and a
real $a\in\mathbb{R}$ such that the matrix%
\[
A:=\left(
\begin{array}
[c]{ccccc}%
\mu_{1} &  &  &  & y_{1}\\
& \mu_{2} &  &  & y_{2}\\
&  & \ddots &  & \vdots\\
&  &  & \mu_{n-1} & y_{n-1}\\
y_{1} & y_{2} & \cdots & y_{n-1} & a
\end{array}
\right)
\]
(where all empty cells are supposed to be filled with $0$s) has eigenvalues
$\lambda_{1},\lambda_{2},\ldots,\lambda_{n}$. (This matrix is, of course,
Hermitian, since it is real symmetric.)
\end{proposition}

\begin{proof}
Omitted. (Exercise?)
\end{proof}

Now, let us derive some consequences from Cauchy's interlacing theorem. We
begin with a straightforward generalization:

\begin{corollary}
[Cauchy's interlacing theorem for multiple deletions]Let $A\in\mathbb{C}%
^{n\times n}$ be a Hermitian matrix. Let $r\in\left\{  0,1,\ldots,n\right\}
$. Let $C\in\mathbb{C}^{r\times r}$ be the result of removing $n-r$ rows and
the corresponding $n-r$ columns from $A$. (That is, we pick some $j_{1}%
<j_{2}<\cdots<j_{n-r}$, and we remove the $j_{1}$-st, $j_{2}$-nd, $\ldots$,
$j_{n-r}$-th rows from $A$, and we remove the $j_{1}$-st, $j_{2}$-nd, $\ldots
$, $j_{n-r}$-th columns from $A$.) Then, for each $j\in\left[  r\right]  $, we
have%
\[
\lambda_{j}\left(  A\right)  \leq\lambda_{j}\left(  C\right)  \leq
\lambda_{j+n-r}\left(  A\right)  .
\]

\end{corollary}

\begin{proof}
Induction on $n-r$.

The \textit{base case} ($n-r=0$) is trivial, since $C=A$ in this case.

In the \textit{induction step}, we obtain $C$ from $B$ by removing a single
row and the corresponding column. Thus, by the original Cauchy interlacing
theorem, we get $\lambda_{j}\left(  B\right)  \leq\lambda_{j}\left(  C\right)
$. However, by the induction hypothesis, we get $\lambda_{j}\left(  A\right)
\leq\lambda_{j}\left(  B\right)  $. Combining these inequalities, we get
$\lambda_{j}\left(  A\right)  \leq\lambda_{j}\left(  C\right)  $. The
remaining inequality $\lambda_{j}\left(  C\right)  \leq\lambda_{j+n-r}\left(
A\right)  $ is proved similarly: By the original Cauchy interlacing theorem,
we get $\lambda_{j}\left(  C\right)  \leq\lambda_{j+1}\left(  B\right)  $.
However, by the induction hypothesis, we get $\lambda_{j+1}\left(  B\right)
\leq\lambda_{j+1+\left(  n-r-1\right)  }\left(  A\right)  =\lambda
_{j+n-r}\left(  A\right)  $.
\end{proof}

The next corollary provides a minimum/maximum description of the sum of the
first $m$ smallest/largest eigenvalues of a Hermitian matrix:

\begin{corollary}
Let $A\in\mathbb{C}^{n\times n}$ be a Hermitian matrix. Let $m\in\left\{
0,1,\ldots,n\right\}  $. Then,%
\[
\lambda_{1}\left(  A\right)  +\lambda_{2}\left(  A\right)  +\cdots+\lambda
_{m}\left(  A\right)  =\min\limits_{\text{isometries }V\in\mathbb{C}^{n\times
m}}\operatorname*{Tr}\left(  V^{\ast}AV\right)
\]
and%
\[
\lambda_{n-m+1}\left(  A\right)  +\lambda_{n-m+2}\left(  A\right)
+\cdots+\lambda_{n}\left(  A\right)  =\max\limits_{\text{isometries }%
V\in\mathbb{C}^{n\times m}}\operatorname*{Tr}\left(  V^{\ast}AV\right)  .
\]

\end{corollary}

\begin{proof}
First of all, it suffices to show the first equality, because the second
follows by applying the first to $-A$ instead of $A$.

First, we shall show that%
\[
\lambda_{1}\left(  A\right)  +\lambda_{2}\left(  A\right)  +\cdots+\lambda
_{m}\left(  A\right)  \leq\operatorname*{Tr}\left(  V^{\ast}AV\right)
\ \ \ \ \ \ \ \ \ \ \text{for every isometry }V\in\mathbb{C}^{n\times m}.
\]


Indeed, let $V\in\mathbb{C}^{n\times m}$ be an isometry. Thus, $V$ is an
$n\times m$-matrix whose columns are orthonormal. As we have seen in the first
chapter(?), we can extend each orthonormal tuple of vectors to an orthonormal
basis. Doing this to the columns of $V$, we thus obtain an orthonormal basis
of $\mathbb{C}^{n}$ whose first $m$ entries are the columns of $V$. Let $U$ be
the matrix whose columns are the entries of this basis. Then,%
\[
U=\left(
\begin{array}
[c]{cc}%
V & \widetilde{V}%
\end{array}
\right)  \ \ \ \ \ \ \ \ \ \ \left(  \text{in block-matrix notation}\right)
\]
by construction of this basis, and furthermore the matrix $U$ is unitary since
its columns form an orthonormal basis.

Since $U$ is unitary, we have $U^{\ast}AU\sim A$ and therefore%
\[
\lambda_{j}\left(  U^{\ast}AU\right)  =\lambda_{j}\left(  A\right)
\ \ \ \ \ \ \ \ \ \ \text{for all }j\in\left[  n\right]  .
\]


However, $U=\left(
\begin{array}
[c]{cc}%
V & \widetilde{V}%
\end{array}
\right)  $ entails%
\[
U^{\ast}AU=\left(
\begin{array}
[c]{cc}%
V & \widetilde{V}%
\end{array}
\right)  ^{\ast}A\left(
\begin{array}
[c]{cc}%
V & \widetilde{V}%
\end{array}
\right)  =\left(
\begin{array}
[c]{c}%
V^{\ast}\\
\widetilde{V}^{\ast}%
\end{array}
\right)  A\left(
\begin{array}
[c]{cc}%
V & \widetilde{V}%
\end{array}
\right)  =\left(
\begin{array}
[c]{cc}%
V^{\ast}AV & \ast\\
\ast & \ast
\end{array}
\right)  ,
\]
where the three $\ast$s mean blocks that we don't care about. So the matrix
$V^{\ast}AV$ is obtained from $U^{\ast}AU$ by removing a bunch of rows and the
corresponding columns. Hence, the previous corollary yields%
\[
\lambda_{j}\left(  U^{\ast}AU\right)  \leq\lambda_{j}\left(  V^{\ast
}AV\right)  \ \ \ \ \ \ \ \ \ \ \text{for all }j\in\left[  m\right]
\]
(since $U^{\ast}AU$ is Hermitian (because $A$ is Hermitian)). In other words,%
\[
\lambda_{j}\left(  A\right)  \leq\lambda_{j}\left(  V^{\ast}AV\right)
\ \ \ \ \ \ \ \ \ \ \text{for all }j\in\left[  m\right]
\]
(since $\lambda_{j}\left(  U^{\ast}AU\right)  =\lambda_{j}\left(  A\right)
$). Adding these inequalities together, we obtain%
\begin{align*}
& \lambda_{1}\left(  A\right)  +\lambda_{2}\left(  A\right)  +\cdots
+\lambda_{m}\left(  A\right)  \\
& \leq\lambda_{1}\left(  V^{\ast}AV\right)  +\lambda_{2}\left(  V^{\ast
}AV\right)  +\cdots+\lambda_{m}\left(  V^{\ast}AV\right)  \\
& =\left(  \text{the sum of all eigenvalues of }V^{\ast}AV\right)  \\
& \ \ \ \ \ \ \ \ \ \ \ \ \ \ \ \ \ \ \ \ \left(
\begin{array}
[c]{c}%
\text{since }V^{\ast}AV\text{ is an }m\times m\text{-matrix}\\
\text{and thus has }m\text{ eigenvalues}%
\end{array}
\right)  \\
& =\operatorname*{Tr}\left(  V^{\ast}AV\right)
\end{align*}
(since the sum of all eigenvalues of a matrix is the trace of this matrix).

Now, we need to show that there exists a unitary matrix $V\in\mathbb{C}%
^{n\times m}$ such that%
\[
\lambda_{1}\left(  A\right)  +\lambda_{2}\left(  A\right)  +\cdots+\lambda
_{m}\left(  A\right)  =\operatorname*{Tr}\left(  V^{\ast}AV\right)  .
\]
To do this, we construct $V$ as follows: We pick an eigenvector $x_{i}$ of $A$
at eigenvalue $\lambda_{i}\left(  A\right)  $ for each $i\in\left[  n\right]
$ in such a way that $\left(  x_{1},x_{2},\ldots,x_{n}\right)  $ is an
orthonormal basis of $\mathbb{C}^{n}$. (This is possible because of Theorem
2.6.1 \textbf{(b)}.) Now, let $V\in\mathbb{C}^{n\times m}$ be the matrix whose
columns are $x_{1},x_{2},\ldots,x_{m}$. This matrix $V$ is an isometry, since
$x_{1},x_{2},\ldots,x_{m}$ are orthonormal. Moreover,%
\begin{align*}
V^{\ast}AV  & =\left(
\begin{array}
[c]{c}%
x_{1}^{\ast}\\
x_{2}^{\ast}\\
\vdots\\
x_{m}^{\ast}%
\end{array}
\right)  A\left(
\begin{array}
[c]{cccc}%
x_{1} & x_{2} & \cdots & x_{m}%
\end{array}
\right)  \\
& =\left(
\begin{array}
[c]{cccc}%
x_{1}^{\ast}Ax_{1} & x_{1}^{\ast}Ax_{2} & \cdots & x_{1}^{\ast}Ax_{m}\\
x_{2}^{\ast}Ax_{1} & x_{2}^{\ast}Ax_{2} & \cdots & x_{2}^{\ast}Ax_{m}\\
\vdots & \vdots & \ddots & \vdots\\
x_{m}^{\ast}Ax_{1} & x_{m}^{\ast}Ax_{2} & \cdots & x_{m}^{\ast}Ax_{m}%
\end{array}
\right)  ,
\end{align*}
so that%
\begin{align*}
\operatorname*{Tr}\left(  V^{\ast}AV\right)    & =\sum_{j=1}^{m}x_{j}^{\ast
}\underbrace{Ax_{j}}_{\substack{=\lambda_{j}\left(  A\right)  x_{j}%
\\\text{(since }x_{j}\text{ is an eigenvector of }A\\\text{at eigenvalue
}\lambda_{j}\left(  A\right)  \text{)}}}=\sum_{j=1}^{m}\lambda_{j}\left(
A\right)  \underbrace{x_{j}^{\ast}x_{j}}_{\substack{=\left\vert \left\vert
x_{j}\right\vert \right\vert ^{2}=1\\\text{(since }\left(  x_{1},x_{2}%
,\ldots,x_{n}\right)  \\\text{is an orthonormal basis)}}}\\
& =\sum_{j=1}^{m}\lambda_{j}\left(  A\right)  =\lambda_{1}\left(  A\right)
+\lambda_{2}\left(  A\right)  +\cdots+\lambda_{m}\left(  A\right)  .
\end{align*}
This is precisely what we needed. Thus, we conclude that%
\[
\lambda_{1}\left(  A\right)  +\lambda_{2}\left(  A\right)  +\cdots+\lambda
_{m}\left(  A\right)  =\min\limits_{\text{isometries }V\in\mathbb{C}^{n\times
m}}\operatorname*{Tr}\left(  V^{\ast}AV\right)  .
\]
As we said above, this completes the proof.
\end{proof}

\begin{corollary}
Let $A\in\mathbb{C}^{n\times n}$ be a Hermitian $n\times n$-matrix. Let
$m\in\left\{  0,1,\ldots,n\right\}  $. Let $i_{1},i_{2},\ldots,i_{m}\in\left[
n\right]  $ be distinct. Then,%
\begin{align*}
\lambda_{1}\left(  A\right)  +\lambda_{2}\left(  A\right)  +\cdots+\lambda
_{m}\left(  A\right)    & \leq A_{i_{1},i_{1}}+A_{i_{2},i_{2}}+\cdots
+A_{i_{m},i_{m}}\\
& \leq\lambda_{n-m+1}\left(  A\right)  +\lambda_{n-m+2}\left(  A\right)
+\cdots+\lambda_{n}\left(  A\right)  .
\end{align*}

\end{corollary}

In words: For a Hermitian matrix $A$, each sum of $m$ distinct diagonal
entries of $A$ is sandwiched between the sum of the $m$ smallest eigenvalues
of $A$ and the sum of the $m$ largest eigenvalues of $A$.

\begin{proof}
Let $C$ be the matrix obtained from $A$ by removing all but the $i_{1}$-st,
$i_{2}$-nd, $\ldots$, $i_{m}$-th rows and the corresponding columns of $A$.
Then,%
\[
\operatorname*{Tr}C=A_{i_{1},i_{1}}+A_{i_{2},i_{2}}+\cdots+A_{i_{m},i_{m}}.
\]


However, Cauchy's interlacing theorem for multiple deletions yields%
\[
\lambda_{j}\left(  A\right)  \leq\lambda_{j}\left(  C\right)
\ \ \ \ \ \ \ \ \ \ \text{for each }j\in\left[  m\right]  .
\]
Summing these up, we obtain%
\begin{align*}
\lambda_{1}\left(  A\right)  +\lambda_{2}\left(  A\right)  +\cdots+\lambda
_{m}\left(  A\right)    & \leq\lambda_{1}\left(  C\right)  +\lambda_{2}\left(
C\right)  +\cdots+\lambda_{m}\left(  C\right)  \\
& =\left(  \text{the sum of all eigenvalues of }C\right)  \\
& =\operatorname*{Tr}C=A_{i_{1},i_{1}}+A_{i_{2},i_{2}}+\cdots+A_{i_{m},i_{m}}.
\end{align*}
So we have proved the first of the required two inequalities. The second
follows by applying the first to $-A$ instead of $A$.
\end{proof}

The above corollary has a bunch of consequences that are obtained by restating
it in terms of something called \textbf{majorization}. Let us define this
concept and see what it entails.

\subsection{Introduction to majorization theory}

\begin{convention}
Let $x=\left(  x_{1},x_{2},\ldots,x_{n}\right)  ^{T}\in\mathbb{R}^{n}$ be a
column vector with real entries. Then, for each $i\in\left[  n\right]  $, we
let $x_{i}^{\downarrow}$ denote the $i$-th largest entry of $x$. So $\left(
x_{1}^{\downarrow},x_{2}^{\downarrow},\ldots,x_{n}^{\downarrow}\right)  $ is
the unique permutation of the tuple $\left(  x_{1},x_{2},\ldots,x_{n}\right)
$ that satisfies
\[
x_{1}^{\downarrow}\geq x_{2}^{\downarrow}\geq\cdots\geq x_{n}^{\downarrow}.
\]

\end{convention}

For example, if $x=\left(  3,5,2\right)  ^{T}$, then $x_{1}^{\downarrow}=5$
and $x_{2}^{\downarrow}=3$ and $x_{3}^{\downarrow}=2$.

Similarly, we define $x_{i}^{\uparrow}$ to be the $i$-th smallest entry of $x$.

\begin{definition}
Let $x\in\mathbb{R}^{n}$ and $y\in\mathbb{R}^{n}$ be two column vectors with
real entries. Then, we say that $x$ \textbf{majorizes} $y$ (and we write
$x\succcurlyeq y$) if and only if we have%
\[
\sum_{i=1}^{m}x_{i}^{\downarrow}\geq\sum_{i=1}^{m}y_{i}^{\downarrow
}\ \ \ \ \ \ \ \ \ \ \text{for each }m\in\left[  n\right]  ,
\]
with equality for $m=n$ (and possibly for other $m$'s). In other words, $x$
majorizes $y$ if and only if%
\begin{align*}
x_{1}^{\downarrow}  & \geq y_{1}^{\downarrow};\\
x_{1}^{\downarrow}+x_{2}^{\downarrow}  & \geq y_{1}^{\downarrow}%
+y_{2}^{\downarrow};\\
x_{1}^{\downarrow}+x_{2}^{\downarrow}+x_{3}^{\downarrow}  & \geq
y_{1}^{\downarrow}+y_{2}^{\downarrow}+y_{3}^{\downarrow};\\
& \ldots;\\
x_{1}^{\downarrow}+x_{2}^{\downarrow}+\cdots+x_{n-1}^{\downarrow}  & \geq
y_{1}^{\downarrow}+y_{2}^{\downarrow}+\cdots+y_{n-1}^{\downarrow};\\
x_{1}^{\downarrow}+x_{2}^{\downarrow}+\cdots+x_{n}^{\downarrow}  &
=y_{1}^{\downarrow}+y_{2}^{\downarrow}+\cdots+y_{n}^{\downarrow}.
\end{align*}

\end{definition}

\begin{example}
We have%
\[
\left(
\begin{array}
[c]{c}%
1\\
3\\
5\\
7
\end{array}
\right)  \geq\left(
\begin{array}
[c]{c}%
2\\
2\\
6\\
6
\end{array}
\right)  ,
\]
since%
\begin{align*}
7  & \geq6;\\
7+5  & \geq6+6;\\
7+5+3  & \geq6+6+2;\\
7+5+3+1  & =6+6+2+2.
\end{align*}

\end{example}

\begin{example}
We don't have%
\[
\left(
\begin{array}
[c]{c}%
1\\
3\\
5\\
7
\end{array}
\right)  \geq\left(
\begin{array}
[c]{c}%
0\\
2\\
6\\
8
\end{array}
\right)  ,
\]
since we don't have $7\geq8$.
\end{example}

The intuition behind majorization is the following: $x$ majorizes $y$ if and
only if you can obtain $y$ from $x$ by \textquotedblleft having the entries
come closer together (while keeping the average equal)\textquotedblright.

\begin{proposition}
Majorization is a partial order: i.e., a reflexive, antisymmetric and
transitive relation.
\end{proposition}

However, it is not a total order: For example, the sequences%
\[
x:=\left(
\begin{array}
[c]{c}%
2\\
2\\
4\\
6
\end{array}
\right)  \ \ \ \ \ \ \ \ \ \ \text{and}\ \ \ \ \ \ \ \ \ \ y=\left(
\begin{array}
[c]{c}%
1\\
3\\
5\\
5
\end{array}
\right)
\]
satisfy neither $x\succcurlyeq y$ nor $y\succcurlyeq x$, since we have $6>5$
but $6+4+2<5+5+3$.

Now we can restate our last corollary as follows:

\begin{corollary}
[Schur's theorem]Let $A\in\mathbb{C}^{n\times n}$ be a Hermitian $n\times
n$-matrix. Then,%
\[
\left(  A_{1,1},A_{2,2},\ldots,A_{n,n}\right)  ^{T}\succcurlyeq\left(
\lambda_{1}\left(  A\right)  ,\lambda_{2}\left(  A\right)  ,\ldots,\lambda
_{n}\left(  A\right)  \right)  ^{T}.
\]
In words: The tuple of diagonal entries of $A$ majorizes the tuple of
eigenvalues of $A$.
\end{corollary}

\begin{proof}
We need to show that

\begin{itemize}
\item the sum of the $m$ largest diagonal entries of $A$ is $\geq$ to the sum
of the $m$ largest eigenvalues of $A$ for each $m\in\left[  n\right]  $;

\item the sum of all diagonal entries of $A$ equals the sum of all eigenvalues
of $A$.
\end{itemize}

But the first of these two statements is the first inequality in the above
corollary, whereas the second statement is the well-known theorem that the
trace of a matrix is the sum of its eigenvalues.
\end{proof}

Now, what can we do with majorizing vectors? Here is probably the most
important property:

\begin{theorem}
[Karamata's inequality]Let $I\subseteq\mathbb{R}$ be an interval. Let
$f:I\rightarrow\mathbb{R}$ be a convex function. Let $x=\left(  x_{1}%
,x_{2},\ldots,x_{n}\right)  ^{T}\in\mathbb{R}^{n}$ and $y=\left(  y_{1}%
,y_{2},\ldots,y_{n}\right)  ^{T}\in\mathbb{R}^{n}$ be two vectors such that
$x\succcurlyeq y$. Then,%
\[
f\left(  x_{1}\right)  +f\left(  x_{2}\right)  +\cdots+f\left(  x_{n}\right)
\geq f\left(  y_{1}\right)  +f\left(  y_{2}\right)  +\cdots+f\left(
y_{n}\right)  .
\]

\end{theorem}

For example, applying this to $f\left(  t\right)  =t^{2}$, we obtain%
\[
x_{1}^{2}+x_{2}^{2}+\cdots+x_{n}^{2}\geq y_{1}^{2}+y_{2}^{2}+\cdots+y_{n}^{2}.
\]


Next time, we will prove Karamata's inequality.
\end{document}